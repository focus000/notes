\documentclass[11pt]{amsart}

\usepackage{amssymb}
\usepackage{mathtools}
\usepackage{cleveref}
% \usepackage{hyperref}
% \bibliographystyle{plain}
\usepackage{biblatex}
\addbibresource{Remote.bib}
\addbibresource{DynamicalSystem.bib}
\addbibresource{background.bib}
\addbibresource{pre.bib}

\newtheorem{theorem}{Theorem}[section]
\newtheorem{corollary}[theorem]{Corollary}
\newtheorem{lemma}[theorem]{Lemma}
\newtheorem{proposition}[theorem]{Proposition}
\newtheorem{remark}[theorem]{Remark}
\theoremstyle{definition}
\newtheorem{definition}[theorem]{Definition}
\newtheorem{proofpart}{part}
\makeatletter
\@addtoreset{proofpart}{theorem}
\makeatother

\numberwithin{equation}{section}

% \providecommand*{\lemmaautorefname}{Lemma}
% \providecommand*{\corollaryautorefname}{Corollary}
% \providecommand*{\propositionautorefname}{Proposition}
% \providecommand*{\remarkautorefname}{Remark}
% \providecommand*{\definitionautorefname}{Definition}

\newcommand*\abs[1]{\lvert#1\rvert}
\newcommand*\norm[1]{\lVert#1\rVert}
\newcommand*\Brace[1]{\lbrace#1\rbrace}
% \newcommand*\innerproduct[1]{\langle#1\rangle}

\newcommand\R{\mathbb{R}}

\DeclareMathOperator{\Div}{div}

\title[Attractor of $p$-laplace equation]{Global attractors of weighted $p$-laplace equation with damping term}

\author[Y. Li]{Yunfang Li}
\address[Y. Li]{School of Mathematics and Statistics, Lanzhou University, Lanzhou, 730000, China}
\email{{\tt liyf17@lzu.edu.cn}}

\keywords{Global attractor, global solution, weighted $p$-laplacian}
\subjclass[2010]{74H40, 35B41, 35K65}

\begin{document}

\begin{abstract}
	We study the long-time behavior of the solutions to a nonlinear damped and weighted $p$-laplace equation. We prove the
	existence and uniqueness of weak solutions and the existence
	of the global attractors.
\end{abstract}

\maketitle

\section{Introduction}
This article is devoted to study the asymptotic dynamics for a dynamical system generated by a nonlinear
weighted $p$-laplace equation that reads
\begin{equation}\label{eq:main}
\begin{alignedat}{2}
& u_t = \Div(a(x)\abs{\nabla u}^{p-2}\nabla u) - b(x)\abs{\nabla u}^2 \quad &\text{in } \Omega \times \R^+,\\
& u(x,0) = u_0 \quad &\text{in } \Omega,\\
& u = 0 \quad &\text{on } \partial\Omega,
\end{alignedat}
\end{equation}
where $\Omega$ is a bounded open domain in $\R^{n}$ with a sufficiently smooth boundary $\partial\Omega$, $p>1$.
$ a(x)$, $b(x) \in C^1(\bar{\Omega}) $, $b(x) \geq 0$, and $a(x) > 0$ in $\Omega$, $a(x) = 0$ on $\partial\Omega$.

The PDEs which are conservative or have some dissipative property
usually could generate dynamical systems. For the last case, one can
hope all the trajectories could be attracted into a bound sets even
global attractor, which has a very complicated geometry reflects the
complexity of the long-time behavior of the system. In order to
understand the asymptotic behavior of dynamical systems, we could
analyse the existence and structure of its global attractor
(see \autocite{cholewaGlobalAttractorsAbstract2000a,chueshovIntroductionTheoryInfinitedimensional2002,robinsonInfiniteDimensionalDynamicalSystems2001a,temamInfiniteDimensionalDynamicalSystems1997} and reference therein).

In order to prove the existence of the attractor the main assumption is
the compactness of the solution operator associated with the system,
which is usually available for parabolic systems in bounded domain.
However, this kind of compactness does not hold in general for the
lack of compact embedding. Instead, we often have some kind of
asymptotic compactness.

Before restricting our focus on features of the main evolution problem,
we should emphasize that the class of weighted $p$-laplace type equations
has widely used in many branches of applied sciences such as image
denoising, model growing/collapsing sandpiles, model Newtonian and non Newtonian Fluids, type-II superconductivity theory etc (see \autocite{aronssonFastSlowDiffusion1996,aubertMathematicalProblemsImage2006,mastorakisSolutionPLaplacianNonNewtonian2009,yinLaplacianTypeEvolution2001} and reference therein).

Now let's go back to the statement of the main results in the present paper. In order to prove the existence of the attractor, firstly, we need to define a semigroup in some suitable space. In Zhan's paper~\cite{Zhan2019Uniquenessa}, the author has proved the existence and uniqueness of the weak solution to \cref{eq:main} with positive initial data in $L^{\infty}(\Omega) \cap W_0^{1,p}(a,\Omega)$, which is
\begin{theorem}\cite{Zhan2019Uniquenessa}\label{thm:zhan}
	If $p>4$, $a(x)$, $b(x)$ satisfies
	\begin{equation}
		\int_{\Omega} b^{\frac{2p}{p-4}}a^{-\frac{4}{p-4}} \leq c,
	\end{equation}
	and $u_0$ satisfies
	\begin{equation}
		u_0 \in L^{\infty}(\Omega), a(x)u_0 \in W_0^{1,p}(\Omega),
	\end{equation}
	then \cref{eq:main} has unique nonnegative weak solution satisfies
	\begin{equation}
		u \in L^{\infty}(Q_T), a(x)\abs{\nabla u}^p \in L^1(Q_T).
	\end{equation}
	The initial value is satisfied in the sense of that
	\begin{equation}
		\lim_{t \to 0}\int_{\Omega}\abs{u(x,t) - u_0(x)}dx = 0.
	\end{equation}
\end{theorem}
In this paper we release the condition of initial data as
$u_0 \in L^2(\Omega)$, and get the existence and
uniqueness of the weak solution. Futhermore, we use
these solutions to define a semigroup $\Brace{S(t)}_{t \geq 0} $ in $L^2(\Omega) $ and study the asymptotic behavior of this equation. However,
the damping term brings a lot of trouble on some estimates, firstly in
the proof of existence of solution, we lack of Gr\"onwall's inequality
to get convergence, and then the lackness of embedding theorem when we
estimate $\norm{u_t(s)}_2^2$ in \cref{thm:ut_L2_bd} which is important
to proof the existence of global attractor in $W_0^{1,p}(\Omega)$.

The content of the paper is as follows.
In \cref{sec:preliminaries}
we state some basic results that will be used later. In
\cref{sec:Existence_and_uniqueness_of_the_weak_solution}
we proof the existence and uniqueness of the weak solution.
In \cref{sec:existence_of_the_global_attractors}
we give the proof of the existence of global attractors in
$L^2(\Omega)$ and $W_0^{1,p}(\Omega)$.
\section{Preliminaries and hypotheses}\label{sec:preliminaries}
In this section, we will introduce the functional spaces and some
useful lemmas, useful results in other papers will also put in here.

Now we give the definition of weighted sobolev spaces.
Let $\Omega$ be a bounded domain in $\R^n$ and
$a \colon \R^n \to [0, \infty)$
be a locally summable nonnegative function, i.e.\ a weight.
\begin{definition}
	A weighted Lebesgue space $L^p(a, \Omega)$, $1 \leq p < \infty$,
	as a Banach space of locally summable functions
	$u \colon \Omega \to \R$ equipped with the following norm:
	\begin{equation}
		\norm{u}_{L^p(a,\Omega)} =
		\left( \int_{\Omega}a\abs{u}^p \right)^{\frac{1}{p}}.
	\end{equation}
\end{definition}
\begin{definition}
	A weighted Sobolev space $W^{k,p}(a,\Omega)$,
	$1 \leq k < \infty$, $1 \leq p < \infty$,
	as a normed space of locally summable, $k$ times weakly
	differentiable functions $u \colon \Omega \to \R$ equipped with the following norm:
	\begin{equation}
		\norm{u}_{W^{k,p}(a,\Omega)} =
		\left( \int_{\Omega}a\abs{u}^p \right)^{\frac{1}{p}}
		+ \sum_{\abs{\alpha}=k}
		\left( \int_{\Omega}a\abs{D^{\alpha}u}^p \right)^{\frac{1}{p}},
	\end{equation}
	where $\alpha$ is a multi-index.
	
	Furthermore,
	$W_0^{k,p}(a,\Omega)$ is defined as the completion of
	$C_0^{\infty}(\Omega)$ with respect to the norm
	\begin{equation}
		\norm{u}_{L^p(a,\Omega)} =
		\left( \int_{\Omega}a\abs{D^{\alpha}u}^p \right)^{\frac{1}{p}}.
	\end{equation}
\end{definition}
\begin{remark}
	Generally $W_0^{k,p}(a,\Omega)$ is not dense in
	$W^{k,p}(a,\Omega)$ unless the Muckenhoupt condition $a \in A_{p^-}$ is satisfied. See~\cite{goldshteinWeightedSobolevSpaces2009} for detailed discussing.
\end{remark}
Next we give a lemma which is helpful to deal with $p$-laplacian.
\begin{lemma}\label{lem:VecIneq}
	Let $a$ and $b$ denoting vectors in $\R^{n}$, if $p \geq 4$, then we have
	\begin{equation}
		\abs{a^2 - b^2}^{\frac{p}{2}}
		\leq C \langle \abs{a}^{p-2}a - \abs{b}^{p-2}b, a-b\rangle
	\end{equation}
\end{lemma}
\begin{proof}
	when $p \geq 1 $
	\begin{equation}\label{eq:p_ineq}
		\abs{a-b}^p \leq C\abs{\abs{a}^{p-1}a - \abs{b}^{p-1}b},
	\end{equation}
	to prove this, firstly, for $n = 1$ and $a$, $b \geq 0$, we have
	\begin{equation}
		\abs{a-b}^p \leq \abs{a^p - b^p}.
	\end{equation}
	Without loss of generality, we assume $a \leq b$, let
	\begin{equation}
		f(x) = (b-a+x)^p - x^p,
	\end{equation}
	then
	\begin{equation}
		f'(x) = p\left((b-a+x)^{p-1} - x^{p-1}\right) \geq 0,
	\end{equation}
	hence
	\begin{equation}
		\left(b-a\right)^p = f(0) \leq f(a) = b^p - a^p.
	\end{equation}
	Applying jensen's inequality, we also have
	\begin{equation}
		\left(a+b\right)^p \leq 2^{p-1}\left(a^p + b^p\right),
	\end{equation}
	hence we proved \cref{eq:p_ineq} while $n = 1$, for $n > 1$, we need use
	the law of cosines, assume $\gamma$ is the angle contained between sides
	of lengths $a$ and $b$, we have
	\begin{equation}
		\begin{split}
			\left(\abs{a-b}^p\right)^2
			&= \left(a^2 + b^2 - 2 \abs{a}\abs{b}\cos{\gamma}\right)^p\\
			&\leq \left(a^2+b^2\right)^p - 2^p\abs{a}^p\abs{b}^p\cos^p{\gamma}\\
			&\leq 2^{p-1}\left(a^{2p} + b^{2p} - 2\abs{a}^p\abs{b}^p\cos^p{\gamma}\right)\\
			&\leq \left(2^{p-1}+C\right)\left(a^{2p} + b^{2p}\right)
			- \left(2^{p-1}\cos^p{\gamma}+C\right)2\abs{a}^p\abs{b}^p\\
			&\leq \left(2^{p-1}+C\right)\left(a^{2p} + b^{2p} - 2\abs{a}^p\abs{b}^p\cos{\gamma}\right)\\
			&= \left(2^{p-1}+C\right)\abs{\abs{a}^{p-1}a - \abs{b}^{p-1}b}^2,
		\end{split}
	\end{equation}
	when $C$ large enough, indeed
	\begin{equation}
		\begin{split}
			& 2^{p-1}\cos^p{\gamma} + C - \left(2^{p-1}+C\right)\cos{\gamma}\\
			={} & 2^{p-1}\left(\cos^p{\gamma} - \cos{\gamma}\right) + C\left(1-\cos{\gamma}\right)
			\geq 0
		\end{split}
	\end{equation}
	when $C$ large enough. Hence we proved \cref{eq:p_ineq}.

	On the other hand, we have
	\begin{equation}
		\abs{\abs{b}^{p-1}b - \abs{a}^{p-1}a} \leq p\abs{b-a}\int_0^1 \abs{a + t(b - a)}^{p-1}dt.
	\end{equation}
	Note that, if $p \geq 2$, we have
	\begin{equation}\label{eq:VecIneq_1}
		\langle \abs{a}^{p-2}a - \abs{b}^{p-2}b, a-b\rangle
		\geq \abs{a-b}^2\int_0^1 \abs{b + t(a - b)}^{p-2}dt,
	\end{equation}
	the proof of~\cref{eq:VecIneq_1} could find in~\cite{lindqvistNotesStationaryPLaplace2019}.
	Hence when $p \geq 4$ we have
	\begin{equation}
		\begin{split}
			\abs{a^2 - b^2}^{\frac{p}{2}}
			&\leq 2^{\frac{p}{2}}\abs{a-b}^{\frac{p}{2}}
			\left(\int_0^1 \abs{b + t(a - b)}dt\right)^{\frac{p}{2}}\\
			&\leq C\frac{p-2}{2}2^{\frac{p}{2}}\abs{a-b}^2
			\left(\int_0^1 \abs{b + t(a - b)}^{\frac{p-4}{2}}dt\right)\\
			&\times\left(\int_0^1 \abs{b + t(a - b)}dt\right)^{\frac{p}{2}}\\
			&\leq C\frac{p-2}{2}2^{\frac{p}{2}}\abs{a-b}^2
			\int_0^1 \abs{b + t(a - b)}^{p-2}dt\\
			&\leq C \langle \abs{a}^{p-2}a - \abs{b}^{p-2}b, a-b\rangle
		\end{split}
	\end{equation}
\end{proof}

\section{Existence and uniqueness of the weak solution}\label{sec:Existence_and_uniqueness_of_the_weak_solution}
Now we prove the existence of the weak solution.
Since we already have the well-posedness when initial data
satisfied strictly conditions (see \cref{thm:zhan}),
we can use these
initial data to approach $u_0 \in L^2$.
\begin{theorem}\label{thm:absorb}
If~\cref{eq:main} satisfy conditions in~\cite[thm 1.3]{Zhan2019Uniquenessa}, and $u_0 \in L^2(\Omega) $, then it has a weak solution satisfies
\begin{equation}
u \in L^p(0, T; W_0^{1,p}(a,\Omega)), \quad u \in C([0, T]; L^2(\Omega)).
\end{equation}
\end{theorem}

\begin{proof}
\begin{proofpart}[remove positive initial value from {\cite[thm 1.3]{Zhan2019Uniquenessa}}]
	let $u_{\epsilon,0} \in C_0^\infty(\Omega) $, $au_{\epsilon,0} \to au_0 $ in
	$W_0^{1,p}(\Omega) $, by maximum principle we have $\abs{u_{\epsilon}} \leq C$,
	where $C$ only depends on $\norm{u_0}_{\infty} $.
	\begin{equation}
		\begin{split}
			\abs{\int_0^T\int_{\Omega} b\abs{\nabla u_{\epsilon}}^2u_{\epsilon}}
			&\leq C\int_0^T\left( \int_{\Omega} b^{\frac{p}{p-2}}a^{-\frac{2}{p-2}}\abs{u_{\epsilon}}^{\frac{p}{p-2}} \right)^{\frac{p-2}{p}}
			\left(  \int_{\Omega} a\abs{\nabla u_{\epsilon}}^p \right)^{\frac{2}{p}}\\
			&\leq C\int_0^T\left(\int_{\Omega}\abs{u_{\epsilon}}^2\right)^{\frac{p}{2(p-2)}} + \frac{1}{2}\int_0^T\int_{\Omega} a\abs{\nabla u_{\epsilon}}^p\\
			&\leq \eta\int_0^T\int_{\Omega}\abs{u_{\epsilon}}^2 + \frac{1}{2}\int_0^T\int_{\Omega} a\abs{\nabla u_{\epsilon}}^p + C(\eta)
		\end{split}
	\end{equation}
	then just as the prove in~\cite[thm 1.3]{Zhan2019Uniquenessa}, we have the weak solution.
\end{proofpart}
\begin{proofpart}[$u_0 \in L^2(\Omega)$]
	Choose $C_c^{\infty}(\Omega) \supset \Brace{u_{n, 0}}_{n=1}^{\infty} $
	convergence to $u_0$ in $L^2(\Omega) $ as $n \to \infty $. $\forall u_{n, 0}$,
	by the first part of this proof, there exist a unique $u_n$ as a weak solution satisfy \cref{eq:main}.
	Hence we have
	\begin{equation}
		\begin{split}
			& \frac{1}{2}\int_{\Omega}\left(u_n-u_m\right)^2(t)\\
			+{} & \int_{0}^{t}\int_{\Omega}a(x)
			\left(\abs{\nabla u_n}^{p-2}\nabla u_n
			- \abs{\nabla u_m}^{p-2}\nabla u_m\right)
			\left(\nabla u_n - \nabla u_m\right)\\
			={} & \int_{0}^{t}\int_{\Omega}b(x)\left(\abs{\nabla u_n}^2
			- \abs{\nabla u_m}^2\right)\left(u_n - u_m\right)
			+ \frac{1}{2}\int_{\Omega}\left(u_n-u_m\right)^2(0).
		\end{split}
	\end{equation}
	Use \cref{lem:VecIneq}, we have
	\begin{equation}
		\begin{split}
			& \int_{0}^{t}\int_{\Omega}b\left(\abs{\nabla u_n}^2
			- \abs{\nabla u_m}^2\right)\left(u_n - u_m\right)\\
			\leq{} & \left(\int_0^t\int_{\Omega}a\left(\abs{\nabla u_n}^2
			- \abs{\nabla u_m}^2\right)^{\frac{p}{2}}\right)^{\frac{2}{p}}
			\left(\int_0^t\int_{\Omega}\left(ba^{-\frac{2}{p}}
			\left(u_n-u_m\right)\right)^{\frac{p}{p-2}}\right)^{\frac{p-2}{p}}\\
			\leq{} & C\left(\int_0^t\int_{\Omega}a
			\left(\abs{\nabla u_n}^{p-2}\nabla u_n
			- \abs{\nabla u_m}^{p-2}\nabla u_m\right)
			\left(\nabla u_n - \nabla u_m\right)\right)^{\frac{2}{p}}\\
			\times{} & \left(\int_0^t\int_{\Omega}b^{\frac{2p}{p-4}}a^{-\frac{4}{p-4}}\right)^{\frac{p-4}{2p}}
			\left(\int_0^t\int_{\Omega}\left(u_n-u_m\right)^2\right)^{\frac{1}{2}}\\
			\leq{} & \frac{1}{2}\int_0^t\int_{\Omega}a
			\left(\abs{\nabla u_n}^{p-2}\nabla u_n
			- \abs{\nabla u_m}^{p-2}\nabla u_m\right)
			\left(\nabla u_n - \nabla u_m\right)\\
			+{} & C\left(\int_0^t\int_{\Omega}\left(u_n-u_m\right)^2\right)^{\frac{p}{2(p-2)}}.
		\end{split}
	\end{equation}
	Combine above inequalities we get
	\begin{equation}\label{un-umL2_0TW1p_bd_0TL2_L20}
		\begin{split}
			& \int_{\Omega}\left(u_n-u_m\right)^2(t)\\
			+{} & \int_{0}^{t}\int_{\Omega}a
			\left(\abs{\nabla u_n}^{p-2}\nabla u_n
			- \abs{\nabla u_m}^{p-2}\nabla u_m\right)
			\left(\nabla u_n - \nabla u_m\right)\\
			\leq{} & C\left(\int_0^t\int_{\Omega}
			\left(u_n-u_m\right)^2\right)^{\frac{p}{2(p-2)}}
			+ \int_{\Omega}\left(u_n-u_m\right)^2(0).
		\end{split}
	\end{equation}
	For $p=4$, we use Gr\"onwall's inequality have
	\begin{equation}
		\begin{split}
			\int_{\Omega}\left( u_n-u_m \right)^2(t)
			\leq \int_{\Omega}\left(u_n-u_m\right)^2(0)e^{Ct} ,
		\end{split}
	\end{equation}
	for a.e.\ $0 \leq t \leq T$. For $p>4$,
	we can't use Gr\"onwall's inequality directly,
	since $\frac{p}{2(p-2)}<1$. But we can use Young's inequality up
	exponential to $1$,
	indeed we can deduce \cref{un-umL2_0TW1p_bd_0TL2_L20} as
	\begin{equation}
		\begin{split}
			& \int_{\Omega}\left(u_n-u_m\right)^2(t)\\
			+{} & \int_{0}^{t}\int_{\Omega}a
			\left(\abs{\nabla u_n}^{p-2}\nabla u_n
			- \abs{\nabla u_m}^{p-2}\nabla u_m\right)
			\left(\nabla u_n - \nabla u_m\right)\\
			\leq{} & \epsilon\int_0^t\int_{\Omega}
			\left(u_n-u_m\right)^2
			+ \epsilon^{-\frac{p}{p-4}}C
			+ \int_{\Omega}\left(u_n-u_m\right)^2(0),
		\end{split}
	\end{equation}
	hence we can use Gr\"onwall's inequality to get
	\begin{equation}\label{un-umL2_gronwall}
		\begin{split}
			\int_{\Omega}\left( u_n-u_m \right)^2(t)
			\leq \left( 
				\epsilon^{-\frac{p}{p-4}}C
				+ \int_{\Omega}\left(u_n-u_m\right)^2(0)
				\right)e^{\epsilon t}
		\end{split}
	\end{equation}
	for a.e.\ $0 \leq t \leq T$. Choose $\epsilon = \delta t^{-1}$,
	where $\delta > 0$ is arbitrary, for a.e.\ fixed $0 \leq t \leq T$,
	let $n,m \to \infty$ and $\delta \to 0$, we have
	\begin{equation}\label{cauchy_in_L2}
		\norm{u_n(t,\cdot)-u_m(t,\cdot)}_{L^2(\Omega)} \to 0,
	\end{equation}
	hence for a.e.\ fixed $0 \leq t \leq T$, there exists
	$u(t,\cdot) \in L^2(\Omega)$ such that
	\begin{equation}
		\norm{u_n(t,\cdot)-u(t,\cdot)}_{L^2(\Omega)} \to 0
	\end{equation}
	as $n \to \infty$, applying continuous of $u_n(t, \cdot)$
	in $L^2(\Omega)$, we have
	\begin{equation}
		\lim_{t' \to t}\norm{u(t,\cdot)-u(t',\cdot)}_{L^2(\Omega)}=0.
	\end{equation}
	Applying \cref{cauchy_in_L2} in \cref{un-umL2_0TW1p_bd_0TL2_L20},
	we have
	\begin{equation}\label{cauchy_in_W1pa}
		\int_{0}^{t}\int_{\Omega}a
		\left(\abs{\nabla u_n}^{p-2}\nabla u_n
		- \abs{\nabla u_m}^{p-2}\nabla u_m\right)
		\left(\nabla u_n - \nabla u_m\right)
		\to 0,
	\end{equation}
	note that
	\begin{equation}
		\begin{split}
			& \int_0^T\int_{\Omega}a\abs{\nabla u_n - \nabla u_m}^p\\
			\leq{} & 2^{p-2}\int_{0}^{T}\int_{\Omega}a
			\left(\abs{\nabla u_n}^{p-2}\nabla u_n
			- \abs{\nabla u_m}^{p-2}\nabla u_m\right)
			\left(\nabla u_n - \nabla u_m\right),
		\end{split}
	\end{equation}
	hence there exists $u \in L^p(0, T; W_0^{1,p}(a,\Omega))
	\cap C([0, T]; L^2(\Omega))$ such that
	\begin{equation}
		u_{n} \to u
	\end{equation}
	in $L^p(0, T; W_0^{1,p}(a,\Omega))\cap C([0, T]; L^2(\Omega))$
	as $n \to \infty$.

	Finally, we consider the damping term, applying H\"older's inequality,
	for any $v \in C_c^\infty$,
	\begin{equation}
		\begin{split}
			& \int_0^T\int_{\Omega}b\left(\abs{\nabla u_n}^2
			- \abs{\nabla u}^2\right)v\\
			\leq{} & C_p\left(\int_0^T\int_{\Omega}a
			\left(\abs{\nabla u_n}^{p-2}\nabla u_n
			- \abs{\nabla u}^{p-2}\nabla u\right)
			\left(\nabla u_n - \nabla u\right)\right)^{\frac{2}{p}}\\
			\times{} & \left(\int_0^T\int_{\Omega}b^{\frac{2p}{p-4}}a^{-\frac{4}{p-4}}\right)^{\frac{p-4}{2p}}
			\left(\int_0^T\int_{\Omega}v^2\right)^{\frac{1}{2}} \to 0
		\end{split}
	\end{equation}
	as $n \to \infty$,
\end{proofpart}
hence we have completed the proof.
\end{proof}
\begin{remark}
	For $p>4$, we also could estimate \cref{un-umL2_0TW1p_bd_0TL2_L20} as
	\begin{equation}
		\begin{split}
			& \int_{\Omega}\left( u_n-u_m \right)(t)\\
			+{} & \int_{0}^{t}\int_{\Omega}a
			\left(\abs{\nabla u_n}^{p-2}\nabla u_n
			- \abs{\nabla u_m}^{p-2}\nabla u_m\right)
			\left(\nabla u_n - \nabla u_m\right)\\
			\leq{} & C\int_0^t\int_{\Omega}
			\left( u_n-u_m \right)^{\frac{p}{p-2}}
			+ \int_{\Omega}\left( u_n-u_m \right)^2(0),
		\end{split}
	\end{equation}
	since $\frac{p}{p-2}>1$, we could use Gr\"onwall's inequality,
	then applying similar steps we could get \cref{cauchy_in_W1pa}.
	By weak compactness theorem we could also get the existence of solutions.
\end{remark}
Uniqueness can use the same way in \cite{Zhan2019Uniquenessa}
to get. Now we can use these solutions to define a semigroup $\Brace{S(t)}_{t>0}$,
by setting
\begin{equation}
	S(t)u_0 = u(t),
\end{equation}
which is continuous on $u_{0}$ in $L^2(\Omega)$.
\section{Existence of the global attractors}\label{sec:existence_of_the_global_attractors}
\begin{theorem}
	The semigroup $\Brace{S(t)}_{t \geq 0} $ possesses a bounded absorbing set in
	$L^2$ and $W_0^{1,p}(a,\Omega)$ respectively, i.e.\ for any bounded subset
	$B \subset L^2(\Omega)$, there exist constants $T(\norm{u_0}_2)$ and $\rho > 0$, such that
	\begin{equation}
		\norm{u(t)}_2^2 + \int_{\Omega}a\abs{\nabla u}^p \leq \rho,
	\end{equation}
	for all $t \geq T$ and $u_0 \in B$, where $u(t) = S(t)u_0$.
\end{theorem}
\begin{proof}
	If we multiply \cref{eq:main} by $u$ and integrate over $\Omega$,
	then we have
	\begin{equation}\label{eq:operate_u}
		\frac{1}{2}\frac{d}{dt}\norm{u}_2^2
		+ \int_\Omega a\abs{\nabla u}^p
		+ \int_\Omega b\abs{\nabla u}^2u = 0,
	\end{equation}
	since $p \geq 4$, use H\"older's inequality,
	\begin{equation}
			\int_{\Omega}\abs{\nabla u}^2
			= \int_{\Omega}a^{-\frac{2}{p}}a^{\frac{2}{p}}\abs{\nabla u}^2
			\leq \left(\int_{\Omega}a^{-\frac{2}{p-2}}\right)^{\frac{p-2}{p}}
			\left(\int_{\Omega}a\abs{\nabla u}^p\right)^{\frac{2}{p}}
	\end{equation}
	and use Poincar\`e's inequality, we have
	\begin{equation}
		\left(\int_{\Omega}\abs{u}^2\right)^{\frac{p}{2}}
		\leq C\int_{\Omega}a\abs{\nabla u}^p,
	\end{equation}
	where $C$ only related with $\Omega$ and $p$. For the 3rd term, we could use
	H\"older's inequality and Young's inequality to get the bound respectively,
	\begin{equation}\label{eq:absorb_damping_u}
		\begin{split}
			\abs{\int_{\Omega}b\abs{\nabla u}^2 u}
			&\leq \left(\int_{\Omega}a\abs{\nabla u}^p\right)^{\frac{2}{p}}
			\left(\int_{\Omega}\abs{b a^{-\frac{2}{p}} u}^{\frac{p}{p-2}}\right)^{\frac{p-2}{p}}\\
			&\leq \frac{1}{2}\int_{\Omega}a\abs{\nabla u}^p
			+ 2^{\frac{p-2}{2}}\int_{\Omega}\abs{b a^{-\frac{2}{p}} u}^{\frac{p}{p-2}}\\
			&\leq \frac{1}{2}\int_{\Omega}a\abs{\nabla u}^p
			+ 2^{\frac{p-2}{2}}\left(\int_{\Omega}b^{\frac{2p}{p-4}}a^{-\frac{4}{p-4}}\right)^{\frac{p-4}{2(p-2)}}
			\left(\int_{\Omega}u^2\right)^{\frac{p}{2(p-2)}}\\
			&\leq \frac{1}{2}\int_{\Omega}a\abs{\nabla u}^p
			+ \epsilon \left(\int_{\Omega}u^2\right)^{\frac{p}{2}}
			+ C_{\epsilon}\left(\int_{\Omega}b^{\frac{2p}{p-4}}a^{-\frac{4}{p-4}}\right)^{\frac{p-4}{2(p-3)}},
		\end{split}
	\end{equation}
	choose $\epsilon$ small enough and combine above estimates, we get
	\begin{equation}\label{eq:DL2_L2_bd}
		\frac{d}{dt}\norm{u}_2^2 + \norm{u}_2^p \leq C
	\end{equation}
	where $C$ independents on $u$. By the Gr\"onwall's inequality, we get the
	bounded absorbing set in $L^2(\Omega)$, i.e.\ $\exists \rho_0$ and
	$T_0 = T_0(\norm{u_0}_2)$ such that
	\begin{equation}\label{eq:L2_bd}
		\norm{u(t)}_2^2 \leq \rho_0 \text{ for } t \geq T_0.
	\end{equation}
	Applying \cref{eq:operate_u,eq:absorb_damping_u,eq:L2_bd}, we have
	\begin{equation}\label{eq:DL2_W1pa_bd}
		\frac{d}{dt}\norm{u}_2^2 + \int_{\Omega}a \abs{\nabla u}^p \leq C.
	\end{equation}
	If we multiply \cref{eq:main} by $u_{t}$ and integrate over $\Omega$,
	then we have
	\begin{equation}\label{eq:operate_ut}
		\norm{u_t}_2^2
		+ \frac{1}{p}\frac{d}{dt}\int_{\Omega}a \abs{\nabla u}^p
		+ \int_{\Omega}b \abs{\nabla u}^2 u_t
		= 0,
	\end{equation}
	estimate the 3rd term like \cref{eq:absorb_damping_u}, then we could get
	\begin{equation}\label{eq:utL2_DW1pa_bd_W1pa}
		C\norm{u_t}_2^2
		+ \frac{d}{dt}\int_{\Omega}a \abs{\nabla u}^p
		\leq \int_{\Omega}a \abs{\nabla u}^p
		+ C,
	\end{equation}
	\cref{eq:DL2_W1pa_bd} $+$ \cref{eq:utL2_DW1pa_bd_W1pa} implies that
	\begin{equation}\label{eq:utL2_DW1pa_DL2}
		C\norm{u_t}_2^2
		+ \frac{d}{dt}\left(\int_{\Omega}a \abs{\nabla u}^p
		+ \norm{u}_2^2\right)
		\leq C.
	\end{equation}
	Applying \cref{eq:DL2_L2_bd}, we have
	\begin{equation}
		\frac{d}{dt}\norm{u}_2^2 + \norm{u}_2^2 \leq C
	\end{equation}
	hence
	\begin{equation}\label{eq:int_DL2_L2_bd}
		\norm{u(t+1)}_2^2
		+ \int_t^{t+1}\norm{u}_2^2
		\leq C + \norm{u(t)}_2^2,
	\end{equation}
	similarity, applying \cref{eq:DL2_W1pa_bd}, we have
	\begin{equation}\label{eq:int_DL2_W1pa_bd}
		\norm{u(t+1)}_2^2
		+ \int_t^{t+1}\int_{\Omega}a \abs{\nabla u}^p
		\leq C + \norm{u(t)}_2^2,
	\end{equation}
	combine \cref{eq:int_DL2_L2_bd,eq:int_DL2_W1pa_bd}, we have
	\begin{equation}\label{eq:int_W1pa_L2}
		\int_t^{t+1}\int_{\Omega}a \abs{\nabla u}^p
		+ \norm{u}_2^2 \leq C \text{ for } t \geq T_0.
	\end{equation}
	Applying \cref{eq:utL2_DW1pa_DL2,eq:int_W1pa_L2}, use Uniform Gr\"onwall's Lemma we get the bounded
	absorbing set in $W_0^{1,p}(a,\Omega)$, i.e\ $\exists \rho$ and $T \geq T_{0}$ such that
	\begin{equation}\label{eq:uL2_W1pa_bd}
		\norm{u(t)}_2^2 + \int_{\Omega}a\abs{\nabla u}^p \leq \rho \text{ for } t \geq T,
	\end{equation}
	hence we completed this proof.
\end{proof}
Since $p \geq 4$, by the compact embedding
$W_0^{1,p}(\Omega) \subset\subset L^2(\Omega)$
and \cref{thm:absorb} yield the existence of a global attractor in $L^2(\Omega)$
immediately.
\begin{theorem}\label{thm:attractor_L2}
	The semigroup $\Brace{S(t)}_{t \geq 0}$ generated by the weak solution of
	\cref{eq:main} possesses a global attractor $\mathcal{A}_2$ in $L^2(\Omega)$.
\end{theorem}
Next we prove the existence of attractor in $W_0^{1,p}(\Omega)$. Firstly, we have
to obtain $u_t$ bounded in $L^2(\Omega)$. Unfortunately, we cannot get
uniform boundedness of $u_t$ when $\abs{\nabla u}$ small enough, hence we deal $\abs{\nabla u}$ as $\abs{\nabla u}+1$.
\begin{theorem}\label{thm:ut_L2_bd}
	For any bounded subset $B$ in $L^2(\Omega)$, if $\abs{\nabla u} > 1$,
	there exists a constant $T' = T'(B) > 0$, such that
	\begin{equation}
		\norm{u_t(s)}_2^2 \leq M, \quad \forall u_0 \in B \text{ and } s \geq T'.
	\end{equation}
\end{theorem}
\begin{proof}
	% \begin{proofpart}
	% 	First we consider the situation of $\abs{\nabla u} \to 0$ as $t \to t_0$,
	% 	where $t_0 \in \R^{+}\cup\Brace{+\infty}$, without loss of generality,
	% 	we assume $\abs{\nabla u} \leq 1$.
	% 	If we multiply \cref{eq:main} by any fixed $\phi \in C_c^{\infty}(\Omega)$
	% 	and integrate over $\Omega$, then take limits, we have,
	% 	\begin{equation}
	% 		\begin{split}
	% 			\lim_{t \to t_0}\int_{\Omega}u_t \phi
	% 			= - \lim_{t \to t_0}\int_{\Omega}a\abs{\nabla u}^{p-2} \nabla u
	% 			\cdot \nabla \phi
	% 			- \lim_{t \to t_0}\int_{\Omega}b\abs{\nabla u}^2 \phi
	% 			= 0
	% 		\end{split}
	% 	\end{equation}
	% \end{proofpart}
	By differentiating \cref{eq:main} in time and denoting $v = u_t$, we have
	\begin{equation}
		\begin{split}
			v_t
			= \Div\left(a\abs{\nabla u}^{p-2}\nabla v\right)
			+ \Div\left(\left(p-2\right)a \abs{\nabla u}^{p-4}\left(\nabla u \cdot \nabla v\right)\nabla u\right)
			- 2b\nabla u \cdot \nabla v.
		\end{split}
	\end{equation}
	We multiply above equation by $v$ and integrate over $\Omega$, then we have
	\begin{equation}
		\begin{split}
			\frac{1}{2}\frac{d}{dt}\norm{v}_2^2
			+ \int_{\Omega}a\abs{\nabla u}^{p-2}\abs{\nabla v}^2
			&+ \int_{\Omega}\left(p-2\right)a\abs{\nabla u}^{p-4}\left(\nabla u
			\cdot \nabla v\right)^2\\
			&+ 2\int_{\Omega}b\nabla u \cdot \nabla v v
			= 0.
		\end{split}
	\end{equation}
	Now we estimate the damping term,
	\begin{equation}
		\begin{split}
			\int_{\Omega}b\nabla u \cdot \nabla v v
			&\leq \left(\int_{\Omega}b^2\left(\nabla u
			\cdot \nabla v\right)^2\right)^{\frac{1}{2}}
			\*\norm{v}_2\\
			&\leq \norm{b^2 a^{-1}}_{\infty}^{\frac{1}{2}}
			\left(\int_{\Omega}a\abs{\nabla u}^2\abs{\nabla v}^2\right)^{\frac{1}{2}}\norm{v}_2\\
			&\leq \norm{b^2 a^{-1}}_{\infty}^{\frac{1}{2}}
			\left(\int_{\Omega}a\left(\abs{\nabla u}+1\right)^2\abs{\nabla v}^2\right)^{\frac{1}{2}}\norm{v}_2\\
			% &= \norm{b^2 a^{-1}}_{\infty}^{\frac{1}{2}}
			% \left(
			% \int_{\Omega}a\left(
			% \left(\abs{\nabla u}+1\right)^2 + 1 - 2\left(\abs{\nabla u}+1\right)
			% \right)\abs{\nabla v}^2
			% \right)^{\frac{1}{2}}\norm{v}_2\\
			&\leq \norm{b^2 a^{-1}}_{\infty}^{\frac{1}{2}}
			\left(\int_{\Omega}a\left(\abs{\nabla u}+1\right)^{p-2}\abs{\nabla v}^2\right)^{\frac{1}{2}}\norm{v}_2\\
			&\leq \epsilon \int_{\Omega}a\left(\abs{\nabla u}+1\right)^{p-2}\abs{\nabla v}^2
			+ C(\epsilon) \norm{v}_2^2,
		\end{split}
	\end{equation}
	for the 2nd term
	\begin{equation}
		\begin{split}
			\int_{\Omega}a\abs{\nabla u}^{p-2}\abs{\nabla v}^2
			\geq 2^{3-p}\int_{\Omega}a\left(\abs{\nabla u}+1\right)^{p-2}\abs{\nabla v}^2
			- \int_{\Omega}a\abs{\nabla v}^2.
		\end{split}
	\end{equation}
	hence we have
	\begin{equation}\label{eq:DvL2_bd_vL2}
		\begin{split}
			& \frac{1}{2}\frac{d}{dt}\norm{v}_2^2
			+ C\int_{\Omega}a\left(\abs{\nabla u}+1\right)^{p-2}\abs{\nabla v}^2\\
			+{} & \int_{\Omega}\left(p-2\right)a\abs{\nabla u}^{p-4}\left(\nabla u
			\cdot \nabla v\right)^2
			\leq C\norm{v}_2^2 + C.
		\end{split}
	\end{equation}
	Now applying \cref{eq:utL2_DW1pa_bd_W1pa,eq:uL2_W1pa_bd,eq:DvL2_bd_vL2},
	we could get
	\begin{equation}
		\begin{split}
			\frac{d}{dt}\left(
			\norm{v}_2^2 + C\int_{\Omega}a\abs{\nabla u}^p
			\right)
			\leq C,
		\end{split}
	\end{equation}
	next we integrating \cref{eq:utL2_DW1pa_DL2} from $t$ to $t+1$, we have
	\begin{equation}
		\begin{split}
			\int_{\Omega}a\abs{\nabla u(t+1)}^p
			+ \norm{u(t+1)}_2^2
			+ C\int_t^{t+1}\norm{v}_2^2
			\leq C + \int_{\Omega}a\abs{\nabla u(t)}^p
			+ \norm{u(t)}_2^2,
		\end{split}
	\end{equation}
	hence for $t$ large enough, we have
	\begin{equation}
		\begin{split}
			\int_t^{t+1}\left(
			\norm{v}_2^2 + C\int_{\Omega}a\abs{\nabla u}^p
			\right) \leq C,
		\end{split}
	\end{equation}
	then we use Uniform Gr\"onwall's Lemma could get, there exists $M > 0$ such that
	\begin{equation}
		\begin{split}
			\norm{v(t)}_2^2
			+ \int_{\Omega}a\abs{\nabla u}^p \leq M \text{ for } t \geq T' \geq T,
		\end{split}
	\end{equation}
	hence we completed this proof.
\end{proof}
Now we can prove the existence of global attractor in
$W_0^{1,p}(\Omega)$.
\begin{theorem}
	The semigroup $\Brace{S(t)}_{t \geq 0}$ associated with \cref{eq:main} possesses
	a global attractor $\mathcal{A}_V$ in $W_0^{1,p}(\Omega)$, i.e.\
	$\mathcal{A}_V$ is compact in $W_0^{1,p}(\Omega)$, invariant and attracts the
	bounded sets of $L^2(\Omega)$ in the topology of $W_0^{1,p}(\Omega)$.
\end{theorem}
\begin{proof}
	In this prove, all we need to show is that $\Brace{S(t)}_{t \geq 0}$ is
	asymptotically compact in $W_0^{1,p}(\Omega)$.
	Let $B_0$ be a bounded absorbing set in $W_0^{1,p}(\Omega)$ obtained in
	\cref{thm:absorb}, then we need to verify for any sequence
	$\Brace{u_{0n}}_{n=1}^{\infty} \subset B_0$, $\Brace{u_n(t_n)}_{n=1}^{\infty}$
	possesses a convergence subsequence in $W_0^{1,p}(\Omega)$.

	Indeed thanks to \cref{thm:attractor_L2}, we have
	$\Brace{u_n(t_n)}_{n=1}^{\infty}$ is precompact in $L^2(\Omega)$.
	So we could take a subsequence in $\Brace{u_n(t_n)}_{n=1}^{\infty}$ is a
	Cauchy sequence in $L^2(\Omega)$
	and still denote as $\Brace{u_n(t_n)}_{n=1}^{\infty}$. Since $p \geq 4$ and
	applying H\"older's inequality, we have
	\begin{equation*}
		\begin{split}
			& \int_{\Omega}a\abs{\nabla u_m(t_m) - \nabla u_n(t_n)}^p\\
			\leq{} & C\int_{\Omega}a
			\left(\abs{\nabla u_m(t_m)}^{p-2}\nabla u_m(t_m)
			- \abs{\nabla u_n(t_n)}^{p-2}\nabla u_n(t_n)\right)
			\*\nabla\left(u_m(t_m) - u_n(t_n)\right)\\
			={} & C\left(\int_{\Omega}\left(u_n-u_m\right)_t\left(u_m-u_n\right)
			- \int_{\Omega}b\left(\abs{\nabla u_m}^2 - \abs{\nabla u_n}^2\right)
			\left(u_m - u_n\right)\right)\\
			\leq{} & C\norm{\left(u_n-u_m\right)_t}_2\norm{u_m-u_n}_2\\
			+{} & C\left(\int_{\Omega}a\left(\abs{\nabla u_m}^p
			+ \abs{\nabla u_n}^p\right)\right)^{\frac{2}{p}}
			\left(b^{\frac{2p}{p-4}}a^{-\frac{4}{p-4}}\right)^{\frac{p-4}{2p}}
			\norm{u_m-u_n}_2 \to 0
		\end{split}
	\end{equation*}
	as $m$, $n \to \infty$. Hence applying \cref{thm:absorb}, the semigroup $\Brace{S(t)}_{t \geq 0}$
	possesses a global attractor $\mathcal{A}_V$ in $W_0^{1,p}(\Omega)$.
\end{proof}
% \bibliographystyle{plain}
\printbibliography
\end{document}
