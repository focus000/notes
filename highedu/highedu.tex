\documentclass{ctexart}

\begin{document}

\section{发展现状}
\begin{itemize}
	\item 基本形成“基础学科拔尖学生培养试验计划”、国家理科基地、国家重点建设高校理科专业与地方高校理科专业等四类“金字塔形”的人才培养体系;
	\item 专业内部分流培养。
\end{itemize}


\section{存在的问题}
\subsection{高等理科教育的地位和作用}
高等理科教育过于强调对社会经济发展的适应性, 原始创新不足。
\subsection{资源配置问题}
\begin{itemize}
	\item 不同高校之间的经费投入不均衡;
	\item 经费分配到教学环节比例过低;
	\item 国家对理科经费投入增长缓慢, 总量不足;
	\item 经费使用和管理制度存在诸多弊端;
	\item 应加强对不发达地区的理科建设投入;
	\item 实验教学设施条件严重不足。
\end{itemize}
\subsection{师资队伍问题}
\begin{itemize}
	\item 教师队伍数量结构性不足;
	\item 教师教学激励机制不足;
	\item 教师培训和发展不足。
\end{itemize}
\subsection{人才培养质量}
学生的专业基础知识薄弱, 理想抱负的缺失, 学习的物质利益等外在学习动机比较强烈, 自主学习的主动性不足, 学习不够勤奋, 课堂参与不积极等问题, 缺乏专业兴趣, 毕业后从事职业与专业相关性不大。
\subsection{生源质量}
由于基础教育阶段“题海训练”的教育模式, 高考对学生学习兴趣、动机、精力的过渡压榨, 进入大学之后学生普遍存在学习兴趣不足, 自主性学习很差, 被动学习情况严重;灵活运用知识较为欠缺, 死记硬背较为普遍, 学生的创新思维能力受到了很大的限制, 磨灭了学生的创造性, 没有形成提出问题、思考问题、解决问题的学习习惯。
\subsection{人才培养模式}
人才培养模式僵化单一、重知识传授的教学模式、流于形式的科研训练、“教而不育”
\subsection{专业转换现状}
具有转专业意愿的学生远远多于有资格转专业的学生,并且这类学生可能由于学习行为更不频繁,学习成效更低而被学校转专业政策排除在外。
\subsection{可就业能力发展}
理科毕业生在沟通能力、专业素养、问题解决能力等三个维度所感知到的自我提高程度相对较低,这一趋势会影响到理科毕业生在劳动力市场的就业竞争力。


\section{未来展望}
\begin{itemize}
	\item 客观理性看待历史发展成就, 充分肯定成绩, 清醒认识挑战的严峻性;在深化人才培养原则的基础上, 广泛深入探讨高等理科教育人才培养目标, 明晰理科本科生人才培养目标的具体内容;
	\item 以“激发学习兴趣, 培养终身自主学习能力”为核心目标;
	\item 平衡“灌输”教学和以学生自主学习为中心的教学模式;
	\item 随着奖学金额度的提升,其对于学生发展的促进作用效率逐渐降低,高校可根据自身实际情况,适度降低部分高额奖学金的额度,拓宽奖学金的投放面,使奖学金覆盖面适度扩大。
\end{itemize}

\end{document}