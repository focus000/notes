\documentclass[a4paper]{book}

%  math support
\usepackage{mathtools}
\usepackage{autobreak}
\usepackage{extarrows}
\usepackage[thmmarks,amsmath]{ntheorem}
\usepackage{amssymb}
% \usepackage{amsmath}
\usepackage{hyperref}

% theorems
\newtheorem{thm}{Theorem}[chapter]
\newtheorem{defi}[thm]{Definition}
\newtheorem{lemma}[thm]{Lemma}
\newtheorem{prop}[thm]{Proposition}
\newtheorem{coro}[thm]{Corollary}
{
    \theoremstyle{nonumberplain}
    \theoremheaderfont{\normalfont}
    \theorembodyfont{\normalfont}
    % auto add \QED
    \theoremsymbol{\mbox{$\Box$}}
    \newtheorem{proof}{proof}
}
{
    \theoremstyle{nonumberplain}
    \newtheorem{myDef}{Definition}
}
% abbr
\newcommand\st{\quad \text{s.t.} \quad}
\newcommand\diff{\,\mathrm{d}}
\newcommand\compact{\subset \subset}
\newcommand\xiff\xLongleftrightarrow
\newcommand\ximpliedby\xLongleftarrow
\newcommand\ximplies\xLongrightarrow
\newcommand\xeq\xlongequal
\newcommand\nto\nrightarrow

% declare \norm macro
\DeclarePairedDelimiter{\norm}\lVert\rVert
\DeclarePairedDelimiter{\set}\lbrace\rbrace
\DeclarePairedDelimiter{\abs}\lvert\rvert
% Functional Analysis Symbols
\def\L{\mathcal{L}}
\def\K{\mathcal{K}}
\def\R{\mathbb{R}}
\DeclareMathOperator{\Ima}{Im}
\DeclareMathOperator{\Ker}{Ker}
\DeclareMathOperator{\spann}{span}
\newcommand\spanset[1]{\ensuremath\spann\{#1\}}
% \DeclareMathOperator{\dim}{dim}
% domain range
\DeclareMathOperator{\domain}{D}
\DeclareMathOperator{\range}{R}

\begin{document}
\part{Functional Analysis}
\chapter{Banach and Hilbert Spaces}
\chapter{ODEs}
\chapter{Linear Operators}
\begin{myDef}[Linear]
    An operator $A$ on a vector space $V$ is \emph{linear} if
    \[
        A(x+\lambda y)=Ax+\lambda Ay, \forall x, y \in V \text{ and } \lambda \in \R.
    \]
\end{myDef}
\section{Bounded Linear Operators on Banach Spaces}
\begin{myDef}[Bounded]
    A linear operator $A \colon X \to Y$ is bounded if $\exists M$ s.t.
    \begin{equation}\label{eq:linear opeartor bounded}
        \norm{Ax}_Y \leq M \norm{x}_X, \forall x \in X.
    \end{equation}
    Def $A \in \L(X,Y)$
    \begin{align}
        \norm{A}_{\L(X,Y)}=\inf\set{M \colon \eqref{eq:linear opeartor bounded} \text{ holds}} \iff \\
        \norm{A}_{\L(X,Y)}=\sup_{x \neq 0}\frac{\norm{Ax}_Y}{\norm{x}_X}=\sup_{\norm{x}_X=1}\norm{Ax}_Y
    \end{align}
\end{myDef}
\begin{proof}
    \begin{align*}
        \eqref{eq:linear opeartor bounded} \iff \frac{\norm{Ax}_Y}{\norm{x}_X} \leq M, x \neq 0 \\
        \implies \inf{M} = \sup_{x \neq 0}\frac{\norm{Ax}_Y}{\norm{x}_X}                \\
        = \sup_{x \neq 0}\norm{A\frac{x}{\norm{x}_X}}_Y = \sup_{\norm{x}_X=1}\norm{Ax}_Y
    \end{align*}
\end{proof}
\begin{prop}\label{completeness_of_linear_operator}
    $Y \in$ Banach space, and $X \in$ norm space $\implies \L(X,Y) \in$ Banach space.
\end{prop}
\begin{proof}
    Let $\set{A_n}$ be Cauchy $\subset \L(X,Y). \implies$
    \begin{gather}
        \norm{A_n - A_m}_{op} \leq \epsilon, \forall n,m \geq N.
        \intertext{$\forall$ fixed $x \in X$.}
        \norm{A_nx-A_mx}_Y=\norm{(A_n-A_m)x}_Y \leq \norm{A_n-A_m}_{op}\norm{x}_X,
        \intertext{$\implies \set{A_nx}$ is Cauchy. Since $Y$ is complete $\implies$}
        A_nx \to y(x), \notag
        \intertext{def $A \colon X \to Y$, $Ax \mapsto y$. $A \in \L(X,Y)$: }
        A(x+\lambda y) = \lim_{n \to \infty}A_n(x + \lambda y)
        = \lim_{n \to \infty}A_nx + \lambda \lim_{n \to \infty}A_ny = Ax+\lambda Ay. \notag
        \intertext{$n,m \geq N, n \to \infty, \implies$}
        \norm{Ax-A_nx}_Y \leq \epsilon(x) \leq \epsilon \norm{x}_X.
        \intertext{$A$ is bounded, and}
        \norm{A_n - A}_{op} \leq \epsilon,
    \end{gather}
    $\implies A_n \to A \in \L(X,Y)$.
\end{proof}

\begin{prop}
    $L \colon X \to Y$ be a linear map, $L$ continuous $\iff$ $L$ bounded.
\end{prop}
\begin{proof}
    ($\impliedby$)
    \begin{gather*}
        \norm{L(x_n - x)}_Y \leq \norm{L}_{op}\norm{x_n - x}_X,
    \end{gather*}
    ($\implies$) If $L$ continuous but unbounded then $\forall n, \exists y_n$ s.t. $\norm{Ly_n}_Y > n^2\norm{y_n}_X$. Then
    \begin{gather*}
        x_n = y_n/(n\norm{y_n}_X) \to 0,
    \end{gather*}
    but $\norm{Lx_n}_Y > 0, \implies L$ is not continuous, a contradiction.
\end{proof}

\begin{lemma}\label{boundness_of_integral_operator}
    Let $k(x,y) \in L^2(\Omega \times \Omega)$:
    \begin{gather*}
        \int_{\Omega}\int_{\Omega}\abs{k(x,y)}^2 \diff x \diff y = C^2 < \infty.
    \end{gather*}
    Integral operator $K \colon L^2(\Omega) \to L^2(\Omega)$ defined by
    \begin{gather}
        [Ku](x)=\int_{\Omega}k(x,y)u(y) \diff y
    \end{gather}
    is bounded.
\end{lemma}
\begin{proof}
    Use C-S inequality
    \begin{align*}
        \abs{Ku}^2 & = \int_{\Omega}(\int_{\Omega}k(x,y)u(y) \diff y)^2 \diff x                                        \\
                   & \leq \int_{\Omega}(\int_{\Omega}\abs{k(x,y)}^2 \diff y)(\int_{\Omega}\abs{u(y)}^2 \diff y)\diff x \\
                   & = (\int_{\Omega}\int_{\Omega}\abs{k(x,y)}^2 \diff x \diff y)(\int_{\Omega}\abs{u(y)}^2 \diff y)   \\
                   & = C^2 \abs{u}^2
    \end{align*}
\end{proof}
\section{Domain, Range, Kernel, and the Inverse Operator}
% TODO: bounded linear operator : D(A)=X by hahn-banach thm from chapter 4, while the domain is an intrinsic part of the def of unbounded operators
\begin{myDef}[inverse]
    $A$ is inverse if
    \begin{gather*}
        Ax = y
    \end{gather*}
    has a unique solution, $\forall y \in \range(A)$. Define $A^{-1}y=x$. Easy check:
    \begin{gather*}
        AA^{-1}u=u, \forall u \in \range(A) \text{ and } A^{-1}Au=u, \forall u \in \domain(A).
    \end{gather*}
    If $A$ is linear and $A^{-1}$ exist then it is linear too.
\end{myDef}
\begin{proof}
    $\forall x,y \in \range(A), \exists x', y' \in \domain(A)$ s.t.$Ax'=x, Ay'=y$.
    \begin{align*}
        A^{-1}(x+\lambda y) & =A^{-1}(Ax'+\lambda Ay')=A^{-1}A(x'+\lambda y') \\
                            & =x'+\lambda y'=A^{-1}x+\lambda A^{-1}y.
    \end{align*}
\end{proof}
\begin{lemma}\label{A_invertible_iff_kerA=0}
    $A$ is invertible iff $\Ker (A) = \set{0}$.
\end{lemma}
\begin{proof}
    ($\implies$), $Ax=y$ has unique solution $\forall y \in \range(A)$.
    If $\exists 0 \neq z \in \Ker(A)$, then $A(x+z)=y$ also, so $\Ker(A)$ must be $\set{0}$.\\
    ($\impliedby$), if $A$ is not invertible
    $\implies \exists y \in \range(A), x_1 \neq x_2 \in \domain(A)$ s.t.
    $Ax_1=y, Ax_2=y. \implies A(x_1-x_2)=0 \implies 0 \neq x_1-x_2 \in \Ker(A)$.
\end{proof}
\section{The Baire Category Theorem}
\begin{thm}[Baire Category Theorem]
    If $G_i$ is a countable family of dense open subsets of a Banach space $X$ then
    \begin{equation*}
        G = \bigcap_{n=1}^\infty G_n
    \end{equation*}
    is dense in $X$.
\end{thm}
\begin{proof}
    $\overline G = X \iff \forall x \in X$, and $r>0$, $B(x,r) \cap G \neq \emptyset$\\
    since $\forall G_n$ is dense and open, $\exists y \in G_n, s>0$,
    \begin{equation*}
        B(x,r)\cap G_n \supset B(y,2s) \supset \overline{B}(y,s).
    \end{equation*}
    $\exists x_1 \in X, r_1 < 2^{-1} \text{ s.t. } \overline{B}(x_1,r_1) \subset G_1 \cap B(x,r)$\\
    $\exists x_2 \in X, r_2 < 2^{-2} \text{ s.t. } \overline{B}(x_2,r_2) \subset G_2 \cap B(x_1,r_1)$\\
    $\cdots$\\
    $\exists x_n \in X, r_n < 2^{-n} \text{ s.t. } \overline{B}(x_n,r_n) \subset G_n \cap B(x_{n-1},r_{n-1})$.\\
    $\implies$
    \begin{gather}
        \overline{B}(x_1,r_1) \supset \overline{B}(x_2,r_2) \supset \cdots .
    \end{gather}
    Since the space is complete,
    \begin{gather*}
        \bigcap_{n=1}^\infty \overline{B}(x_n,r_n)=x_0.
    \end{gather*}
    $\implies x_0 \in \overline{B}(x_1,r_1) \subset B(x,r)$ and $x_0 \in \overline{B}(x_n,r_n) \subset G_n, \forall n. $\\
    $\implies x_0 \in B(x,r) \cap G$.
\end{proof}
\begin{myDef}[nowhere dense]
    $W$ is nowhere dense if $\overline{W}^\circ = \emptyset$.
\end{myDef}
\begin{coro}\label{Baire Category thm coro}
    Let $X$ be a Banach space and $F_j$ be a countable sequence of nowhere dense subsets. Then
    \begin{gather*}
        \bigcup_{j=1}^\infty F_j \neq X.
    \end{gather*}
\end{coro}
\begin{proof}
    Suppose $\bigcup_{j=1}^\infty F_j = X$,
    \begin{gather*}
        X=\overline{\bigcap_{j=1}^\infty{X \setminus \overline{F_j}}}=\overline{X \setminus \overline{\bigcup_{j=1}^\infty F_j}} \subset \overline{X \setminus \bigcup_{j=1}^\infty F_j} = \emptyset.
    \end{gather*}
    Contradiction!
\end{proof}
\begin{thm}[Uniform Boundedness Principle]
    Let $X$ be a Banach space and $Y$ a normed space. Let $S \subset \L(X,Y)$, and let
    \begin{equation*}
        \sup_{T \in S}\norm{Tx}_Y < \infty \text{ for all } x \in X.
    \end{equation*}
    Then
    \begin{equation*}
        \sup_{T \in S}\norm{T}_{\L(X,Y)} < \infty.
    \end{equation*}
\end{thm}
\begin{proof}
    Define
    \begin{equation*}
        F_j=\set{x \in X \colon \norm{Tx}_Y \leq j, \forall T \in S}.
    \end{equation*}
    $\implies$ $\bigcup_j{F_j}=X$.
    \begin{gather*}
        X \setminus F_j = \set{x \in X  \colon \norm{T_x x}_Y > j, \exists T_x \in S},
        \intertext{$\forall x_0 \in X \setminus F_j$, $\forall \epsilon > 0, \text{ let } \delta = \frac{\epsilon}{\norm{T}_{\L(X,Y)}},\forall x \in B(x_0, \delta)$ s.t.}
        \norm{T_{x_0}x-T_{x_0}x_0}=\norm{T_{x_0}(x-x_0)} \leq \norm{T}_{\L(X,Y)}\norm{x-x_0} <\epsilon.
    \end{gather*}
    let $\epsilon = \epsilon_0 = \norm{j - x_0}/2$, $\implies B(x_0,\delta_0) \subset X \setminus F_j$, \\
    $\implies F_j$ is closed.\\
    $\overset{\ref{Baire Category thm coro}}{\implies} F_n^\circ \neq \emptyset$ for some $n$\\
    Then $\exists y \in X, r > 0$ s.t.
    \begin{equation*}
        B(y,r) \subset F_n.
    \end{equation*}
    Therefore, if $\norm{x}_X \leq r$,
    \begin{equation*}
        \norm{Tx}_Y=\norm{T(x+y)+T(-y)} \leq n +\norm{Ty}_Y \leq 2n,
    \end{equation*}
    since $\sup_{T \in S}\norm{Ty}_Y < \infty. \forall x  \colon \norm{x}_X=r \implies$
    \begin{equation*}
        \norm{Tx}_Y \leq \frac{2n}{r}\norm{x}_X, \text{ for all }T \in S,
    \end{equation*}
    since $T$ is linear, this follows for all $x$.
\end{proof}
% TODO: app ex 3.4
\section{Compact Operators}
\begin{defi}\label{compact_opeartor}
    An opeartor $K \colon X \to Y$ is compact if
    \begin{equation*}
        \overline{K((W < \infty) \subset X)} \compact Y.
    \end{equation*}
    $\xiff{\text{usually used}} \forall (\set{x_n} < \infty) \subset X, \exists \set{Kx_{n_j}}$ converge.
\end{defi}
\begin{lemma}
    A compact operator is bounded.
\end{lemma}
\begin{proof}
    $(B_X(0,1)<\infty) \subset X \ximplies[\text{def } \ref{compact_opeartor}]{K \text{ is compact}} \overline{K(B)} \compact{Y} \implies \overline{K(B)} \subset B_Y(0,R)$, thus
    \begin{equation*}
        \sup_{\norm{x}_X \leq 1}\norm{K(x)}_Y \leq R.
    \end{equation*}
\end{proof}

The space of all compact linear opeartors from norm space $X$ into Banach space $Y$, $(\K(X,Y),\norm{\cdot}_{op})$ is a Banach space (cf. Proposition \ref{completeness_of_linear_operator}):
\begin{thm}\label{completeness_of_compact_linear_operator_s1}
    Let $X$ be a norm space and $Y$ be a Banach space, if $({K_n} \compact{\L(X,Y)}) \to K \in \L(X,Y)$, i.e.
    \begin{equation*}
        \sup_{\norm{x}_X = 1}\norm{K_n{x} - K{x}}_Y \to 0 \text{ as } n \to 0,
    \end{equation*}
    then $K \compact{\L(X,Y)}$.
\end{thm}
\begin{proof}
    we could use Cantor's diagonal argument to find $(\set{y_j} < \infty) \subset X$ s.t. $\forall n$, $K_n(y_j)$ converge. Now show that $K(y_j)$ is Cauchy,
    \begin{gather*}
        \norm{K(y_i)-K(y_j)}_Y \leq \norm{K(y_i) - K_n(y_i)} + \norm{K_n(y_i)-K_n(y_j)} + \norm{K_n(y_j) - K(y_j)},
        \intertext{since $\set{y_j} < \infty$ and $K_n \to K$, pick $n$ large enough that}
        \norm{K(y_j)-K_n(y_j)}_Y \leq \epsilon /3, \quad \forall y_j.
        \intertext{For such an $n$, $\set{K_n(y_j)}$ is Cauchy, so $\exists N$, $i,j \geq N$ s.t.}
        \norm{K_n(y_i)-K_n(y_j)}_Y \leq \epsilon /3.
        \intertext{$\implies$}
        \norm{K(y_i)-K(y_j)}_Y \leq \epsilon, \quad i,j \geq N,
    \end{gather*}
    and $\set{K(y_j)}$ is Cauchy. Since $Y$ is complete, $K \compact{\L(X,Y)}$.
\end{proof}
\begin{coro}
    $X$ norm space and $Y$ Banach space $\implies \K(X,Y)$ Banach space.
\end{coro}
\begin{proof}
    Let $((\set{K_n} \text{ is Cauchy with } \norm{\cdot}_{op}) \subset \K(X,Y) )\subset \L(X,Y), \\ \ximplies{\text{prop} \ref{completeness_of_linear_operator}} K_n \to K \subset \L(X,Y) \ximplies{\text{thm} \ref{completeness_of_compact_linear_operator_s1}} K \in \K(X,Y).$
\end{proof}
\begin{lemma}\label{bounded_op_with_finite-dim_range_is_compact}
    Let $(A \colon X \to Y) < \infty$, if $R(A)$ is finite-dimensional, then $A$ is compact.
\end{lemma}
% TODO: why not necessary linear
\begin{proof}
    Let $(\set{u_n} \subset D(A)) < \infty \ximplies{A < \infty} (\set{Au_n} \subset R(A)) < \infty \\ \ximplies[\text{finite-dimensional}]{R(A) \text{ is}} Au_{n_i}$ converge, and so $A$ is compact.
\end{proof}
\begin{prop}\label{K_compact}
    The integral opeartor $K \colon L^2(\Omega) \to L^2(\Omega)$ given by
    \begin{equation*}
        [Ku](x)=\int_{\Omega}k(x,y)u(y)dy,
    \end{equation*}
    where $k \in L^2(\Omega \times \Omega)$ is compact.
\end{prop}
\begin{proof}
    We need find $K_n$ s.t. $((\set{K_n} < \infty) \subset \K(L^2(\Omega),L^2(\Omega))) \to K$, use Lemma \ref{bounded_op_with_finite-dim_range_is_compact} prove $K_n$ is compact, then use Theorem \ref{completeness_of_compact_linear_operator_s1} prove $K$ is compact.
    Since $\dim(R(K_n)) < \infty$ while $\dim(R(K))$ is infinity, $\dim(R(K_n)) \to \infty$, for $n \to \infty$.\\
    Let $\set{\phi_j}$ be an orthonormal basis for $L^2(\Omega)$. Then $\set{\phi_i(x)\phi_j(y)}$ is an orthonormal basis for $L^2(\Omega \times \Omega)$,then
    \begin{equation*}
        k(x,y)=\sum_{i,j=1}^{\infty}k_{ij}\phi_i(x)\phi_j(y),
    \end{equation*}
    where
    % TODO: proof basis ex 3.5 fix x consider L^2(\Omega)
    \begin{equation*}
        k_{ij}=\int_{\Omega \times \Omega}k(x,y)\phi_i(x)\phi_j(y) \diff x \diff y.
    \end{equation*}
    Since $\set{\phi_i(x)\phi_j(y)}$ is a basis, we have
    \begin{equation}
        \norm{k}_{L^2(\Omega \times \Omega)}^2 = \int_{\Omega}\int_{\Omega}\abs{k(x,y)}^2 \diff x \diff y = \sum_{i,j=1}^{\infty}\abs{k_{ij}}^2.
    \end{equation}
    Set
    \begin{equation*}
        k_n(x,y)=\sum_{i,j=1}^{n}k_{ij}\phi_i(x)\phi_j(y),
    \end{equation*}
    and
    \begin{equation*}
        [K_n u](x)=\int_{\Omega}k_n(x,y)u(y)dy.
    \end{equation*}
    Let $u=\sum_{l=1}^{\infty}c_l\phi_l$, then
    \begin{align*}
        \begin{autobreak}
            K_nu=\int_{\Omega}\left(\sum_{i,j=1}^{n}k_{ij}\phi_i(x)\phi_j(y)\right)\left(\sum_{l=1}^{\infty}c_l\phi_l(y)\right) \diff y
            = \sum_{i,j=1}^n{k_{ij}c_{j}\phi_i(x)\int_{\Omega}\abs{\phi_j(y)}^2 \diff y}
            = \sum_{i,j=1}^n k_{ij}c_{j}\phi_i,
        \end{autobreak}
    \end{align*}
    so $K_n$ has rank $n$, then $K_n$ is compact by lemma \ref{bounded_op_with_finite-dim_range_is_compact}.\\
    By Lemma \ref{boundness_of_integral_operator},
    \begin{align*}
        \norm{K-K_n}^2 \leq \left(\int_{\Omega}\int_{\Omega}\abs(k(x,y)-k_n(x,y))^2 \diff x \diff y\right)\left(\int_{\Omega}\abs{u(y)}^2\right) = \sum_{i,j=n+1}^{\infty}\abs{k_{ij}}^2,
    \end{align*}
    since $\norm{k}_{L^2(\Omega \times \Omega)}^2 = \sum_{i,j=1}^{\infty}\abs{k_{ij}}^2$, $K_n \to K$, so $K$ is compact by Theorem \ref{completeness_of_compact_linear_operator_s1}.
\end{proof}
\section{Compact Symmetric Operators on Hilbert Spaces}
\begin{defi}
    A linear opeartor $A \in \L(H,H)$ is symmetric if
    \[
        (u,Av)=(Au,v), \quad \forall u,v \in H.
    \]
\end{defi}
\begin{prop}
    if $A$ a symmetric opeartor, then
    \begin{equation}\label{eq:symmetric_op_norm_inner_pro_to_norm}
        \norm{A}_{\L(H,H)}=\sup_{\norm{u}=1}\abs{(Au,u)}.
    \end{equation}
\end{prop}
\begin{proof}
    Let $\alpha = \sup_{\norm{u}=1}\abs{(Au,u)}$, then
    \begin{equation}
        \abs{(Au,u)}=\norm{u}^2\abs{(\frac{Au}{\norm{u}},\frac{u}{\norm{u}})} \leq \alpha \norm{u}^2 \quad \forall u \in H.
    \end{equation}
    $\forall u,v \in H$, use \ref{eq:symmetric_op_norm_inner_pro_to_norm} and the parallelogram law,
    \begin{align*}
        4(Au,v)=(A(u+v),u+v)-(A(u-v),u-v)
        \leq \alpha(\norm{u+v}^2+\norm{u-v}^2)
        =2\alpha(\norm{u}^2+\norm{v}^2),
    \end{align*}
    if $Au \neq 0$, choose
    \[ v=\norm{u}\frac{Au}{\norm{Au}}, \]
    since $\norm{v}=\norm{u}$,
    \[ \norm{u}\norm{Au} \leq \alpha \norm{u}^2. \]
    Therefore
    \[ \norm{Au} \leq \alpha \norm{u}, \]
    which holds as well if $Au=0$, so
    \[\norm{A}_{\L(H,H)} \leq \alpha.\]
    The other hand, use C-S inequality, for $\norm{u}=1$,
    \[\abs{(Au,u)} \leq \norm{Au}\norm{u} \leq \norm{A}_{op}. \]
\end{proof}
\begin{lemma}\label{K_symmetric}
    If $k(x,y)=k(y,x)$ then the integral opeartor is symmetric.
\end{lemma}
\begin{proof}
    \begin{align*}
        \begin{autobreak}
            (Ku,v) = \int_{\Omega}[Ku](x)v(x) \diff x
            = \int_{\Omega}\left(\int_{\Omega} k(x,y)u(y) \diff y \right)v(x) \diff x
            = \int_{\Omega}\int_{\Omega} k(x,y)u(y)v(x) \diff x \diff y
            = \int_{\Omega}\int_{\Omega} k(y,x)u(y)v(x) \diff x \diff y
            = \int_{\Omega}\left(\int_{\Omega} k(y,x)v(x) \diff x\right)u(y) \diff y
            = \int_{\Omega} u(y)[Kv](y) \diff y = (u, Kv).
        \end{autobreak}
    \end{align*}
\end{proof}
\section{Obtaining an Eigenbasis from a Compact Symmetric Operator}
\begin{lemma}\label{hilbert-schmidt_thms_lemma}
    If $A$ is a compact symmetric opeartor then at least one of $\pm \norm{A}_{op}$ is an eigenvalue of $A$.
\end{lemma}
\begin{proof}
    Assume $A \neq 0$, otherwise the result is trivial. Since $A$ is symmetric,
    \[\norm{A}_{op}=\sup_{\norm{x}=1}\abs{(Ax,x)}.\]
    Thus $\exists \abs{x_n}=1$ s.t.
    \[(Ax_n,x_n) \to \pm \norm{A}_{op} = \alpha.\]
    Since $A$ is compact, $\exists Ax_{n_j} \to y$, Relabel $x_{n_j}$ as $x_n$ again. Consider
    \[\norm{Ax_n-\alpha x_n}^2 = \norm{Ax_n}^2 + \alpha^2 -2\alpha(Ax_n,x_n) \to 0 \text{ as } n \to \infty;\]
    $\implies$
    \[\alpha x_n \to y \text{ as } Ax_n \to y,\]
    and since fixed $\alpha \neq 0$, we must have $x_n \to x \in H$. $\implies$
    \[Ax=\alpha x\]
    since $\norm{x}-\norm{x_n} \leq \norm{x-x_n} \to 0$, $x \neq 0$.
\end{proof}
\begin{thm}[Hilbert-Schmidt Theorem]\label{hilbert_schmidt_thm}
    Let $A$ be a linear, symmetric, compact operator acting on a infinite-dimensional Hilbert space $H$. Then all eigenvalues $\lambda_j \in \R$ of $A$, and if they are ordered so that
    \[\abs{\lambda_{n+1}} \leq \abs{\lambda_{n}},\]
    one has
    \[\lim_{n \to \infty}\lambda_n = 0.\]
    Furthermore, the eigenvalues $w_j$ can be chosen so that they form an orthonormal basis for $R(A)$, and the action of $A$ on any $u \in H$ is given by
    \[Au = \sum_{j=1}^{\infty} \lambda_j (u,w_j)w_j.\]
\end{thm}
\begin{proof}
    By lemma \ref{hilbert-schmidt_thms_lemma}, $\exists w_1$ s.t. $\norm{w_1}=1$ and $Aw_1 = (\lambda_1 = \pm \norm{A}) w_1$. Consider
    \[H_1 = w_1^{\bot}, \quad \left. A_1 = A \right|_{H_1}.\]
    Clearly, $A_1$ is a symmetric compact opeartor. Since if $u \perp w_1$ then
    \[(Au, w_1)=(u,Aw_1)=\lambda_1(u,w_1)=0.\]
    We know that $A_1 \colon H_1 \to H_1$.
    So we have $(A_1 \text{ or } A) w_2=(\lambda_2 = \pm \norm{A_1})w_2$ and $\norm{w_2} = 1$, and $\abs{\lambda_1} \geq \abs{\lambda_2}$ since $H_1 \subset H$.\\
    Continuing inductively in this way, we obtain $\set{w_j}$, with $Aw_j=\lambda_j w_j$ and $\abs{\lambda_{n+1}} \leq \abs{\lambda_{n}}$.\\
    Assume $\lambda_j \nto 0$, so that $\abs{\lambda_j} \geq \eta$ for some $\eta > 0$, thus $\forall j, \quad \eta w_j \in A(B(0, 1))$, but ${\eta w_j}$ has no convergent subsequence, contradicting the compactness of $A$.

    If $x \perp w_j$ for all $j$, then 
    \[\norm{Ax} \leq \abs{\lambda_j}\norm{x}, \quad \forall j\]
    we must have $Ax=0$, hence $x \in \Ker A$.

    If $\set{k_j}_{j \in \mathcal{I}}$ is an orthonormal basis for $\Ker A$, then $\set{k_j} \cup \set{w_j}$ is an orthonormal basis for $H$. So
    \[u = \sum_{j=1}^{\infty}(u,w_j)w_j + \sum_{j \in \mathcal{I}}(u, k_j)k_j\], 
    we obtain
    \[Au = \sum_{j=1}^{\infty}\lambda_j(u,w_j)w_j\].
\end{proof}
\begin{coro}\label{invertible_to_basis}
    If $A$ is invertible and satisfies the conditions of thm \ref{hilbert_schmidt_thm}, then there is a basis of $H$ consisting entirely of eigenfunctions of $A$.
\end{coro}
\begin{proof}
    $A$ is invertible $\xiff{\text{lemma \ref{A_invertible_iff_kerA=0}}} \Ker{A} =\set{0} \ximplies{\text{thm \ref{hilbert_schmidt_thm}}} u = \sum_{j=1}^{\infty}(u,w_j)w_j$
\end{proof}
Now we consider Sturm-Liouville boundary value problem. $L$ is the linear differential operator:
\begin{equation}
    L[u]=-\frac{\diff}{\diff x}\left(p(x)\frac{\diff u}{\diff x}\right) + q(x)u(x), 
\end{equation}
where $p,q \in C[a,b], \quad p,q > 0$, the Sturm-Liouville eigenvalue problem is to find all those $\lambda$ in
\begin{equation}\label{eq:sturm_liouville_eigenfunction}
    L[u] = \lambda w(x)u(x)
\end{equation}
has a corresponding solution $u(x)$, where "weight function" $w \in C[a,b], \quad w > 0$. For simplicity we study \ref{eq:sturm_liouville_eigenfunction} with the boundary conditions
\[u(a)=u(b)=0\].
\begin{thm}
    The eigenfunctions of problem \ref{eq:sturm_liouville_eigenfunction} form a complete orthonormal basis for $L_w^2(a,b)$:
    \begin{equation}\label{eq:Lw2_def}
        L_w^2(a,b)=\left\{u \colon \int_a^b w(x)\abs{u(x)}^2 \diff x < \infty \right\}.
    \end{equation}
    Inner product corresponds to the norm in definition \ref{eq:Lw2_def}:
    \[(u,v)_w = \int_a^b w(x)u(x)v(x) \diff x.\]
\end{thm}
 \begin{proof}
     First show that any $\lambda > 0$ in \ref{eq:sturm_liouville_eigenfunction}
     \begin{align*}
         \begin{autobreak}
             (Lu,u)=\int_a^b -(pu')'u + q\abs{u}^2 \diff x
             = \int_a^b -(p'u'u+pu''u)+q\abs{u}^2 \diff x
             = \int_a^b -(p'u'u+pu''u+pu'u')+p\abs{u'}^2 +q\abs{u}^2 \diff x
             = \int_a^b p\abs{u'}^2 +q\abs{u}^2 \diff x - \left[ pu'u \right]_a^b
             \xeq{u(a)=u(b)=0} \int_a^b p\abs{u'}^2 +q\abs{u}^2 \diff x
             \geq \int_a^b q\abs{u}^2 \diff x
             \geq \frac{\inf_{x \in [a,b]q(x)}}{\sup_{x \in [a,b]}w(x)}\int_a^b w(x)\abs{u(x)}^2 \diff x
             = \alpha (wu,u),
         \end{autobreak}
     \end{align*}
     with $\alpha > 0$, and so $\lambda \geq \alpha > 0.$

     The Green's function $G(x,y)$, the solution of
     \[L[G]=\delta(x-y), \quad G(x,a)=G(x,b)=0,\]
     is symmetric in $x$ and $y$ (see TODO: ex 3.7). Let
     \[L[u]=f(x), \quad u(a)=u(b)=0\]
     then
     \[u(x)=\int_a^b G(x,y)f(y) \diff y,\]
     since
     \[L[u]=L[\int_a^b G(x,y)f(y) \diff y]=\int_a^b L[G]f(y) \diff y = \int_a^b \delta(x-y)f(y) \diff y = f(x).\]
     The original Sturm-Liouville problem \eqref{eq:sturm_liouville_eigenfunction} $\iff$
     \begin{equation}\label{eq:sturm_liouville_eigenfunction_eq_form}
         \int_a^b G(x,y)w(y)u(y) \diff y = \frac{1}{\lambda} u(x).
     \end{equation}
     Set
     \[v(x)=\sqrt{w(x)}u(x), \quad k(x,y)=G(x,y)\sqrt{w(x)w(y)},\]
     then \eqref{eq:sturm_liouville_eigenfunction_eq_form} $\iff$
     \begin{equation}
         Kv = \frac{1}{\lambda}v,
     \end{equation}
     where $K$ is the integral opeartor given by
     \[Kv(x)=\int_a^b k(x,y)v(y) \diff y.\]
     $k(x,y)$ is symmetric and bounded $\ximplies[\text{prop \ref{K_compact}}]{\text{lemma \ref{K_symmetric}}}$ $K$ is compact and symmetric $\ximplies{\text{thm \ref{hilbert_schmidt_thm}}}$ $K$ has a set of orthonormal eigenfunctions $\set{\phi_j}.$

     Since $\lambda, w(x) > 0$, $\Ker L = \set{0}$ by \eqref{eq:sturm_liouville_eigenfunction} $\implies$ $\Ker K = \set{0}$ $\ximplies{\text{coro \ref{invertible_to_basis}}}$ the eigenfunctions $\set{\phi_j}$ form a basis for $L^2(a,b)$.

     Since $v(x)=\sqrt{w(x)}u(x)$, we have an eigenfunction $\varphi_j=\phi_j w^{-1/2}$ of the original Sturm-Liouville problem \eqref{eq:sturm_liouville_eigenfunction}, and
     \[(\varphi_j,\varphi_k)=\int_a^b w(x)\varphi_j(x)\varphi_k(x) \diff x = (\phi_j, \phi_k) = \delta_{jk}.\]
     Now for any $f \in L_w^2(a,b) \implies w^{1/2}f \in L^2(a,b) \implies$
     \[w^{1/2}f = \sum_{j=1}^{\infty} (w^{1/2}f,\phi_j)\phi_j.\]
     $\implies$
     \[f=\sum_{j=1}^{\infty}(w^{1/2}f, w^{1/2}\varphi_j)\varphi_j = \sum_{j=1}^{\infty}(f,\varphi_j)_w\varphi_j.\]
 \end{proof}
 \chapter{Dual Spaces}
 We study dual space is to circumventing the lack of compactness of bounded sets in infinite-dimensional Spaces.
 \begin{myDef}[linear functional]
     $\forall f \in \L(X, \R)$, where $X$ is a Banach space, called a linear functional on $X$
 \end{myDef}
 Because of its importance, we adopt a special notation for $\L(X, \R)$.
 \begin{defi}[dual space]
     $X$ a real (vector space $X$ over the field $\R$) Banach space,
     \[X^* := \L(X,\R)\]
     called the dual space of $X$.
 \end{defi}
 By Proposition \ref{completeness_of_linear_operator}, $(X^*, \norm{\cdot}_{op})$ is a Banach space, we will denote:
 \[\norm{f}_* = \norm{f}_{X^*} := \norm{f}_{\L(X,\R)}.\]
 \section{The Hahn-Banach Theorem}
 TODO: descrable hahn-banach thm.
 \begin{myDef}[partial order, chain, upper bound, maximal element]
     A partial order on a set $P$ is an opeartion $\leq$ that satisfies
     \begin{enumerate}
         \item (reflexive) $a \leq a$, $\forall a \in P$;
         \item (antisymmetric) if $a \leq b$ and $b \leq a$, then $a = b$;
         \item (transitive) if $a \leq b$ and $b \leq c$, then $a \leq c$.
     \end{enumerate}
     
     A subset $C \subset P$ is called a chain if $\forall a, b \in C$ either $a \leq b$ or $b \leq a$ (or both if $a = b$).

     An element $b$ is an upper bound for $S \subset P$ if $s \leq b$, $\forall s \in S.$

     $m$ is a maximal element of $P$ if $m \leq a \implies a = m.$
 \end{myDef}
 \begin{lemma}[Zorn's lemma]
     if $P$ is a partial order set and all chains in $P$ have an upper bound then $P$ has a maximal element.
 \end{lemma}
 TODO:an application of zorn's lemma ex 4.1
 \begin{thm}
     $X$ be a Banach space and $M$ a linear subspace of $X$. $f \in M^*$ s.t.
     \begin{equation}
         \abs{f(x)} \leq k\norm{x}, \quad \forall x \in M.
     \end{equation}
     Then $\exists$ an extension $F \in X^*$ of $f$, s.t.
     \begin{equation}
         \abs{F(x)} \leq k\norm{x}, \quad \forall x \in X.
     \end{equation}
 \end{thm}
 \begin{proof}
    We use Zorn's lemma to prove the existence of $F$. Consider
    \begin{equation}
        E = \{(g,G) \colon g \in G^* \text{ extends of } f, \abs{g(x)} \leq k\norm{x}, \forall x \in G\}
    \end{equation}
    Since $(f,E) \in E$, $E \neq \emptyset$.
    def an order on $E$: $(g,G) \geq (h,H)$ if $g$ extends $h$.
    For any chain $C=\set{(g_i, G_i)}$, set
    \[u(x) = g_i(x), \quad x \in G_i\]
    on
    \[U=\cup_i G_i,\]
    then $u$ is an upper bound for $C$.
    By Zorn's lemma, there exist a maximal element $(F,W)$ of $E$.

    Now we need proof $W = X$. If $W \neq X$, then $\exists z \in X-W$, $\forall x \in Z = \spanset{z, W}$, can be uniquely written as
    \[x = w +\alpha z,\]
    since
    \[w_1 + \alpha_1 z = w_2 +\alpha_2 z \implies (\alpha_1 - \alpha_2)z=w_1-w_2;\]
    since $z \notin W, w_1-w_2 \in w \implies \alpha_1 = \alpha_2$ and $w_1=w_2$. def $g \in Z^*$
    \[g(w+\alpha Z)=F(w)+\alpha c.\]
    we need let $g$ be extends of $f$ on $Z$, that is
    \begin{equation}
        \abs{F(w)+\alpha c} \leq k\norm{w+\alpha z}, \quad \forall w \in W, \alpha \in \R
    \end{equation}
    $\iff$
    \begin{equation}
        F(w)+\alpha c \leq k\norm{w+\alpha z}
    \end{equation}
    (
        $\iff$
        \begin{equation*}
            F(-w)+(-\alpha) c \leq k\norm{-w+(-\alpha) z}
        \end{equation*}
    )
    $\iff$
    \begin{gather*}
        c \leq \frac{1}{\alpha}(k\norm{w+\alpha z} - F(w)),\\
        -c \leq \frac{1}{\alpha}(k\norm{w-\alpha z} - F(w)),
    \end{gather*}
    with $w \in W$ and $\alpha \geq 0$, $\iff$
    \begin{equation}
        \frac{1}{\alpha_1}(F(w_1) - k\norm{w_1-\alpha_1 z}) \leq c \leq \frac{1}{\alpha_2}(k\norm{w_2+\alpha_2 z} - F(w_2)).
    \end{equation}
    The existence of $c$ equals
    \begin{equation*}
        \sup \frac{1}{\alpha_1}(F(w_1) - k\norm{w_1-\alpha_1 z}) \leq \inf \frac{1}{\alpha_2}(k\norm{w_2+\alpha_2 z} - F(w_2)).
    \end{equation*}
    $\implies$
    \begin{equation*}
        \alpha_2(F(w_1) - k\norm{w_1-\alpha_1 z}) \leq \alpha_1(k\norm{w_2+\alpha_2 z} - F(w_2))
    \end{equation*}
    $\implies$
    \begin{equation}
        F(\alpha_2 w_1 + \alpha_1 w_2) \leq k\norm{\alpha_1 w_2 +\alpha_1 \alpha_2 z} +k\norm{\alpha_2 w_1 +\alpha_1 \alpha_2 z}.
    \end{equation}
    Since
    \begin{equation*}
        \abs{F(\alpha_2 w_1 + \alpha_1 w_2)} \leq k\norm{\alpha_2 w_1 + \alpha_1 w_2} \leq k\norm{\alpha_1 w_2 +\alpha_1 \alpha_2 z} +k\norm{\alpha_2 w_1 +\alpha_1 \alpha_2 z}.
    \end{equation*}
    So we can find constant $c$.
\end{proof}
TODO:lemma 4.4 coro 4.5
\end{document}
