% !TEX TS-program = xelatex
% !BIB program = bibtex
% !TEX TS-program = xelatex
% !TEX TS-program = xelatex

\documentclass[UTF8]{ctexbeamer}
% \usepackage{zh_CN-Adobefonts_external}
\usepackage[T1]{fontenc}
\usepackage{fontspec}
\usepackage{indentfirst}
\setlength{\parindent}{2em}

% \defaultfontfeatures{Path = fonts/zh_CN-Adobe/, Mapping=tex-text}

% \setCJKmainfont[
% BoldFont=AdobeHeitiStd-Regular.otf,
% ItalicFont=AdobeKaitiStd-Regular.otf,
% SmallCapsFont=AdobeHeitiStd-Regular.otf
% ]{AdobeFangsongStd-Regular.otf}
% \setCJKsansfont{AdobeHeitiStd-Regular.otf}
% \setCJKmonofont{AdobeFangsongStd-Regular.otf}

% \setCJKfamilyfont{zhsong}{AdobeSongStd-Light.otf}
% \setCJKfamilyfont{zhhei}{AdobeHeitiStd-Regular.otf}
% \setCJKfamilyfont{zhfs}{AdobeFangsongStd-Regular.otf}
% \setCJKfamilyfont{zhkai}{AdobeKaitiStd-Regular.otf}

% \newcommand*{\songti}{\CJKfamily{zhsong}} % 宋体
% \newcommand*{\heiti}{\CJKfamily{zhhei}}   % 黑体
% \newcommand*{\kaishu}{\CJKfamily{zhkai}}  % 楷书
% \newcommand*{\fangsong}{\CJKfamily{zhfs}} % 仿宋

% \setCJKmainfont{Kaiti SC}
% \defaultfontfeatures{Path = fonts/zh_CN-Adobe/, Mapping=tex-text}
\usepackage{mathtools}
\usepackage{amssymb}
\usepackage{cleveref}
\crefname{equation}{式}{式}
\crefname{figure}{图}{图}
\crefname{table}{表}{表}
\crefname{appendix}{附录}{附录}
\crefname{chapter}{章}{章}
\crefname{theorem}{定理}{定理}
\crefname{lemma}{引理}{引理}
\usepackage[backend=bibtex, style=verbose]{biblatex}
\addbibresource{thesis.bib}
% \bibliographystyle{plain}

\newcommand*\abs[1]{\lvert#1\rvert}
\newcommand*\norm[1]{\lVert#1\rVert}
\newcommand*\Brace[1]{\lbrace#1\rbrace}
% \newcommand*\innerproduct[1]{\langle#1\rangle}

\newcommand\R{\mathbb{R}}
\newcommand*\Laplace{\mathop{}\!\mathbin\bigtriangleup}

\DeclareMathOperator{\Div}{div}
\DeclareMathOperator{\dist}{dist}

\setbeamertemplate{theorems}[numbered]

\setbeamertemplate{lemmas}[theorem]

\setbeamertemplate{definitions}[theorem]
% \addtobeamertemplate{frametitle}{
%     \let\insertframetitle\insertsectionhead}{}
% \addtobeamertemplate{frametitle}{
%     \let\insertframesubtitle\insertsubsectionhead}{}


% \makeatletter
%     \CheckCommand*\beamer@checkframetitle{\@ifnextchar\bgroup\beamer@inlineframetitle{}}
%     \renewcommand*\beamer@checkframetitle{\global\let\beamer@frametitle\relax\@ifnextchar\bgroup\beamer@inlineframetitle{}}
% \makeatother

\mode<presentation>
{
    \usetheme{Madrid}
    \setbeamertemplate{footline}[frame number]
    \expandafter\def\expandafter\insertshorttitle\expandafter{%
        \insertshorttitle\hfill%
        \insertframenumber\,/\,\inserttotalframenumber}
    \setbeamercovered{transparent}
    \setbeamersize{text margin left=6mm, text margin right=6mm}
}

\AtBeginDocument{
    \title{带有阻尼项的加权\texorpdfstring{$p$}--laplace方程全局吸引子的存在性}
    \author[李蕴方]{
        \begin{tabular}{c@{ : }cc}
            \makebox[3em][s]{答辩人}&\makebox[3em][s]{李蕴方}& \\
            \makebox[3em][s]{导师}&\makebox[3em][s]{马闪}&副教授
        \end{tabular}
    }
    \institute{兰州大学数学与统计学院}
    \date{\today}
}

\AtBeginSection[]
{
  \begin{frame}<beamer>{大纲}
    \tableofcontents[currentsection]
  \end{frame}
}

% \AtBeginSubsection[]
% {
%   \begin{frame}<beamer>{大纲}
%     \tableofcontents[currentsection,currentsubsection]
%   \end{frame}
% }

\begin{document}

\begin{frame}
    \titlepage
\end{frame}

\begin{frame}{大纲}
    \tableofcontents
    % You might wish to add the option [pausesections]
\end{frame}

\section{研究背景}
\subsection{无穷维动力系统理论简介}
\begin{frame}{\secname : \subsecname}
    无穷维动力系统理论是用来描述发展方程解的长时间行为的理论框架.
    随着吸引子概念的提出, 数学物理学家们如今通过研究吸引子的存在性, 收敛阶以及维数等来刻画可以由发展方程来描述的复杂系统的长时间行为.

    \begin{itemize}
        \item 流体力学:
        \begin{itemize}
            \item 大气、海洋动力学,
            \item 湍流理论……
        \end{itemize}
        \item 量子物理学: 
        \begin{itemize}
            \item Schrödinger 方程,
            \item Ginzburg–Landau 方程……
        \end{itemize}
        \item 波方程等
    \end{itemize}
\end{frame}
% \begin{frame}{应用背景}
%     \begin{itemize}
%         \item 流体力学:
%         \begin{itemize}
%             \item 大气、海洋动力学,
%             \item 湍流理论……
%         \end{itemize}
%         \item 量子物理学: 
%         \begin{itemize}
%             \item Schrödinger 方程,
%             \item Ginzburg–Landau 方程……
%         \end{itemize}
%         \item 波方程等
%     \end{itemize}
% \end{frame}
\subsection{带有 \texorpdfstring{$p$}--laplace 算子的发展方程研究概况}
\begin{frame}{\secname : \subsecname}
    $p$-laplace 发展方程具体形式为:
    \begin{equation*}
        u_t = \Div(\abs{\nabla u}^{p-2}\nabla u) + f(u, \nabla u, x, t),
    \end{equation*}
    具有广泛的应用背景:
    \begin{itemize}
        \item $1 < p < 4/3$ : 冰川运动;
        \item $p = 3/2$ : 多孔介质渗流现象;
        \item $p = 2$ : 牛顿流体反应扩散现象;
        \item $p > 2$ : 非线性耗散, 低功耗材料(超导体), 非牛顿流体运动等;
        \item 图像识别, 图像恢复, 图像去噪.
    \end{itemize}
\end{frame}
\begin{frame}
    \frametitle{\secname : \subsecname}

    \begin{itemize}
        \item 文献\footfullcite{carvalhoGlobalAttractorsProblems1999}考虑了方程$\frac{du}{dt}+A(u)+B(u)=0$全局吸引子的存在性, 其中$A$是极大单调算子.
        \item 文献\footfullcite{yangGlobalAttractorsPLaplacian2007} 通过验证渐近紧性证明了$p$-Laplace发展方程$W_0^{1, p}$上全局吸引子的存在性.
    \end{itemize}

\end{frame}
\begin{frame}
    \frametitle{\secname : \subsecname}

    \begin{itemize}
        \item 文献\footfullcite{simsenLaplacianDifferentialInclusions2009} 考虑了一类$p$-Laplace微分包含耦合系统全局解的存在性, 以及全局吸引子的存在性和上半连续性.
        \item 文献\footfullcite{simsenLaplacianParabolicProblems2014a}考虑了$p_s(x)$-Laplace发展方程全局吸引子的存在性以及关于$s$的上半连续性.
    \end{itemize}

\end{frame}
\begin{frame}{\secname : \subsecname}
    带权$p$-laplace 发展方程:
    \begin{equation*}
        u_t = \Div(a(x)\abs{\nabla u}^{p-2}\nabla u) + f(u, \nabla u, x, t),
    \end{equation*}
    这里$a(x)$更能刻画材质的非均匀分布导致局部具有不同的物理规律的现象\footfullcite{lionsMathematicalTopicsFluid1996}
\end{frame}
\begin{frame}
    \frametitle{\secname : \subsecname}

    \begin{itemize}
        \item 文献\footfullcite{maGlobalAttractorsWeighted2012a}考虑了加权$p$-Laplace发展方程解的适定性以及全局吸引子的存在性.
        \item 文献\footfullcite{simsenExistenceUpperSemicontinuity2014}考虑了非自治的加权$p$-Laplace发展方程拉回吸引子的上半连续性.
    \end{itemize}

\end{frame}
\subsection{本文研究的问题}
\begin{frame}{\secname : \subsecname}
    本文考虑如下方程
    \begin{equation}\label{eq:main}
        \begin{cases}
            u_t = \Div(a(x)\abs{\nabla u}^{p-2}\nabla u) - b(x)\abs{\nabla u}^2 \quad &(x, t) \in \Omega \times \R^+,\\
            u(x,0) = u_0 \quad &x \in \Omega,\\
            u(x, t) = 0 \quad &x \in \partial\Omega,
        \end{cases}
    \end{equation}
    其中 $\Omega$ 是在 $\R^{n}$ 里的有界光滑开区域, 边界记为 $\partial\Omega$, $p>4$.
    $ a(x)$, $b(x) \in C^1(\bar{\Omega}) $, $b(x) \geq 0$, 在 $\Omega$ 内部 $a(x) > 0$, 在边界 $\partial\Omega$ 上 $a(x) = 0$.
\end{frame}
\begin{frame}{\secname : \subsecname}
    \begin{itemize}
        \item 主要结果:
        \begin{itemize}
            \item 解的适定性;
            \item 全局吸引子在$L^2(\Omega)$和$W_0^{1, p}(\Omega)$中的存在性;
            \item 全局吸引子的上半连续性.
        \end{itemize}
        \item 难点:
        \begin{itemize}
            \item 解的适定性
            \item 用渐近紧性验证$W_0^{1, p}(\Omega)$ 上全局吸引子的存在性
        \end{itemize}
        \item 创新点: 相比于文献\footfullcite{Zhan2019Uniquenessa}讨论了非负初值
        $u_0 \in L^\infty(\Omega)$ 时正解的适定性, 本文考虑了
        \begin{itemize}
            \item 解的适定性: 正解 $\rightsquigarrow$ 变号解 $\rightsquigarrow$ 初值在更一般的空间
            \item 方程的全局吸引子在 $L^2(\Omega)$ 和 $W_0^{1, p}(\Omega)$ 上的存在性
            \item 扰动权函数和初值时方程的全局吸引子的上半连续性
        \end{itemize}
    \end{itemize}
\end{frame}
\section{弱解的存在唯一性}
\subsection{整体思路}
\begin{frame}{\secname : \subsecname}
    \begin{itemize}
        \item 广义 Galerkin 法
        \item 扰动方程作逼近
        \begin{itemize}
            \item 扰动方程非退化
            \begin{itemize}
                \item 非退化方程结论丰富
                \item 证明收敛性困难
            \end{itemize}
            \item 扰动方程退化
            \begin{itemize}
                \item 证明收敛性容易
                \item 退化方程结论相对较少\\
                ($L^\infty$ 估计\footfullcite{dibenedettoDegenerateParabolicEquations1993a}: De Giorgi 迭代技巧)
            \end{itemize}
        \end{itemize}
    \end{itemize}

    初值: $L^\infty(\Omega) \rightsquigarrow L^2(\Omega)$ ?
\end{frame}
\subsection{基本概念和定理}
\begin{frame}{\secname : \subsecname}
    下面给出带权 Lebesgue-Sobolev 空间的概念以定义方程的弱解:
    设 $\Omega$ 是 $\R^n$ 里的有界区域, 权重
    $a \colon \R^n \to [0, \infty)$
    是一个非负的局部可和函数.
    \begin{definition}
        带权 Lebesgue 空间 $L^p(a, \Omega)$, $1 \leq p < \infty$,
        是由局部可和函数类
        $u \colon \Omega \to \R$ 在如下范数下定义的 Banach 空间
        \begin{equation*}
            \norm{u}_{L^p(a,\Omega)} =
            \left( \int_{\Omega}a\abs{u}^p \right)^{\frac{1}{p}}.
        \end{equation*}
    \end{definition}
\end{frame}
\begin{frame}{\secname : \subsecname}
    \begin{definition}
        带权 Sobolev 空间 $W^{k,p}(a,\Omega)$,
        $1 \leq k < \infty$, $1 \leq p < \infty$,
        是由 $k$ 阶弱可微局部可和函数
        $u \colon \Omega \to \R$ 关于如下范数下的 Banach 空间
        \begin{equation*}
            \norm{u}_{W^{k,p}(a,\Omega)} =
            \left( \int_{\Omega}a\abs{u}^p \right)^{\frac{1}{p}}
            + \sum_{\abs{\alpha}=k}
            \left( \int_{\Omega}a\abs{D^{\alpha}u}^p \right)^{\frac{1}{p}},
        \end{equation*}
        这里 $\alpha$ 是一个多重指标.

        另外,
        $W_0^{k,p}(a,\Omega)$ 被定义为
        $C_0^{\infty}(\Omega)$ 在如下范数下的完备空间
        \begin{equation*}
            \norm{u}_{L^p(a,\Omega)} =
            \left( \int_{\Omega}a\abs{D^{\alpha}u}^p \right)^{\frac{1}{p}}.
        \end{equation*}
    \end{definition}
\end{frame}
\begin{frame}{\secname : \subsecname}
    下面给出方程\eqref{eq:main}弱解的定义,
    \begin{definition}
        函数 $u(x, t)$ 是方程\eqref{eq:main}在 $[0, T]$ 上的弱解, 如果
        \begin{equation*}
            u \in C([0, T]; L^2(\Omega))\cap L^p(0, T; W_0^{1,p}(a,\Omega))
        \end{equation*}
        且对于 $\forall \phi \in L^p(0, T; W_0^{1,p}(a,\Omega))\cap L^{\frac{p}{p-2}}(Q_T)$ 有
        \begin{equation*}
            \int_0^T <u_t, \phi> + \int_0^T\int_\Omega a(x)\abs{\nabla u}^{p - 2}\nabla u \cdot \nabla \phi
            + \int_0^T\int_\Omega b(x)\abs{\nabla u}^2\phi.
        \end{equation*}
        其中对于 $u(0, x) = u_0$ 在 $\Omega$ 上几乎处处成立.
    \end{definition}
\end{frame}
% \begin{frame}
%     \begin{theorem}[Zhan2019]\label{thm:zhan}
%         如果 $p>4$, $a(x)$, $b(x)$ 满足
%         \begin{equation}\label{eq:zhan_intcondition}
%             \int_{\Omega} b^{\frac{2p}{p-4}}a^{-\frac{4}{p-4}} \leq c,
%         \end{equation}
%         且 $u_0$ 满足
%         \begin{equation}\label{eq:zhan_initdata}
%             0 \leq u_0 \in L^{\infty}(\Omega), a(x)u_0 \in W_0^{1,p}(\Omega),
%         \end{equation}
%         那么方程\eqref{eq:main} 存在满足如下条件的唯一非负弱解
%         \begin{equation*}
%             u \in L^{\infty}(Q_T), a(x)\abs{\nabla u}^p \in L^1(Q_T).
%         \end{equation*}
%         且初值具有如下性质
%         \begin{equation*}
%             \lim_{t \to 0}\int_{\Omega}\abs{u(x,t) - u_0(x)}dx = 0.
%         \end{equation*}
%     \end{theorem}
% \end{frame}
\subsection{弱解的存在唯一性定理}
\begin{frame}{\secname : \subsecname}
    \begin{theorem}\label{thm:absorb}
        考虑方程\eqref{eq:main}, 如果 $p>4$, $a(x)$, $b(x)$ 满足
        \begin{equation*}
            \int_{\Omega} b^{\frac{2p}{p-4}}a^{-\frac{4}{p-4}} \leq c,
        \end{equation*}
        且 $u_0 \in L^2(\Omega) $, 那么方程存在唯一弱解满足
        \begin{equation*}
            u \in L^p(0, T; W_0^{1,p}(a,\Omega)), \quad u \in C([0, T]; L^2(\Omega)).
        \end{equation*}
    \end{theorem}
\end{frame}
% \subsection{证明思路}
% \begin{frame}{\secname : \subsecname}

%     \begin{gather*}
% 		u_{\epsilon t}-\Div\left((a(x)+\epsilon)
% 		\left(\left|\nabla u_{\epsilon}\right|^{2}+\epsilon\right)^{\frac{p-2}{2}} \nabla u_{\epsilon}\right)
% 		+b(x)\left|\nabla u_{\epsilon}\right|^{2} = 0,(x, t) \in Q_{T} \label{eq:approximated_maineq} \\
% 		u_{\epsilon}(x, t)  = 0, \quad(x, t) \in \partial \Omega \times(0, T)\\
% 		u_{\epsilon}(x, 0)  = u_{\epsilon, 0}(x), \quad x \in \Omega
% 	\end{gather*}
%     由经典PDE理论知道上述方程具有唯一弱解 $u_\epsilon$, 且 $u_\epsilon \in C([0, T], L^2(\Omega))$.

%     结论: 初值 $u_0 \in L^{\infty}(\Omega) \cap W_0^{1, p}(a, \Omega)$ 时, 存在唯一弱解
%     \begin{equation*}
%         u \in L^{\infty}(Q_T) \cap L^p(0, T; W_0^{1,p}(a,\Omega)) \cap C([0, T]; L^2(\Omega))
%     \end{equation*}
    
% \end{frame}
% \begin{frame}{\secname : \subsecname}

%     取一列 $\Brace{u_{n, 0}}_{n=1}^{\infty} \subset C_c^{\infty}(\Omega)$,
%     当 $n \to \infty $ 时 $u_{n, 0}$ 在 $L^2(\Omega)$ 上收敛到 $u_0$.
%     \begin{equation}\label{eq:3}
% 		\begin{split}
% 			& \frac{1}{2}\int_{\Omega}\left(u_n-u_m\right)^2(t)\\
% 			+{} & \int_{0}^{t}\int_{\Omega}a(x)
% 			\left(\abs{\nabla u_n}^{p-2}\nabla u_n
% 			- \abs{\nabla u_m}^{p-2}\nabla u_m\right)
% 			\left(\nabla u_n - \nabla u_m\right)\\
% 			={} & \int_{0}^{t}\int_{\Omega}b(x)\left(\abs{\nabla u_n}^2
% 			- \abs{\nabla u_m}^2\right)\left(u_n - u_m\right)
% 			+ \frac{1}{2}\int_{\Omega}\left(u_n-u_m\right)^2(0).
% 		\end{split}
%     \end{equation}
%     \begin{equation}\label{un-umL2_0TW1p_bd_0TL2_L20}
% 		\begin{split}
% 			& \int_{\Omega}\left(u_n-u_m\right)^2(t)\\
% 			+{} & \int_{0}^{t}\int_{\Omega}a
% 			\left(\abs{\nabla u_n}^{p-2}\nabla u_n
% 			- \abs{\nabla u_m}^{p-2}\nabla u_m\right)
% 			\left(\nabla u_n - \nabla u_m\right)\\
% 			\leq{} & C\left(\int_0^t\int_{\Omega}
% 			\left(u_n-u_m\right)^2\right)^{\frac{p}{2(p-2)}}
% 			+ \int_{\Omega}\left(u_n-u_m\right)^2(0).
% 		\end{split}
% 	\end{equation}

% \end{frame}
% \begin{frame}{\secname : \subsecname}

%     对任意的 $v \in C_c^\infty(\Omega)$,
%     \begin{equation*}
% 		\begin{split}
% 			& \int_0^T\int_{\Omega}b\left(\abs{\nabla u_n}^2
% 			- \abs{\nabla u}^2\right)v\\
% 			\leq{} & C_p\left(\int_0^T\int_{\Omega}a
% 			\left(\abs{\nabla u_n}^{p-2}\nabla u_n
% 			- \abs{\nabla u}^{p-2}\nabla u\right)
% 			\left(\nabla u_n - \nabla u\right)\right)^{\frac{2}{p}}\\
% 			\times{} & \left(\int_0^T\int_{\Omega}b^{\frac{2p}{p-4}}a^{-\frac{4}{p-4}}\right)^{\frac{p-4}{2p}}
% 			\left(\int_0^T\int_{\Omega}v^2\right)^{\frac{1}{2}}.
% 		\end{split}
%     \end{equation*}
%     当 $n \to \infty$ 时, 上式右端趋于 $0$. 以上证明了弱解的存在性.

%     唯一性参见Zhan\footfullcite{Zhan2019Uniquenessa}.

% \end{frame}

\begin{frame}
    \frametitle{\secname : \subsecname}

    \begin{lemma}\label{lem:VecIneq}
        $\alpha$ 和 $\beta$ 是属于 $\R^{n}$ 的 $n$ 维向量, 如果 $p \geq 4$, 那么我们有
        \begin{equation*}
            \abs{\alpha^2 - \beta^2}^{\frac{p}{2}}
            \leq C \langle \abs{\alpha}^{p-2}\alpha - \abs{\beta}^{p-2}\beta, \alpha-\beta\rangle
        \end{equation*}
    \end{lemma}

\end{frame}

\section{全局吸引子的存在性}
\subsection{整体思路}
\begin{frame}{\secname : \subsecname}

    \begin{itemize}
        \item 连续性: 在方程解是唯一的时候,半群的连续性不难得到
        \item 耗散性: 能量估计
        \item 紧性: ?
    \end{itemize}

\end{frame}
\begin{frame}{\secname : \subsecname}

    \begin{itemize}
        \item 紧性: 对 $\forall t \geq 0$ 以及 $X$ 中任意有界集 $B$ 有, $S(t)B$ 是相对紧的;
        \item 一致紧性: 对任意有界集 $B \subset X$, $\exists t(B) > 0$ 使得,
        \begin{equation*}
            \bigcup_{t \geq t(B)} S(t)B \text{ 在 } X \text{ 中相对紧};
        \end{equation*}
        \item 渐近紧性: 对任意有界列 $\Brace{u_k} \subset X$ 和
        $\forall \Brace{t_k}, \quad t_k \to \infty$, 有 $S(t_k)u_k$ 在 $X$ 里列紧.
    \end{itemize}

    相对于直接验证紧性, 更容易去验证一致紧性.
    一般方法是证明在空间 $X‘$ 存在一致有界吸收集, 且 $X'$ 可以紧嵌入到 $X$,
    一般用 Sobolev 紧嵌入定理得到. 通常要求方程具有更高的正则性. 而对于渐近紧性,
    在缺乏紧嵌入定理的时候很有用, 一般验证方式是利用半群的分解.

\end{frame}
\subsection{\texorpdfstring{$L^2(\Omega)$}上全局吸引子的存在性}
\begin{frame}{\secname : \subsecname}

    利用方程解的适定性我们可以定义方程\eqref{eq:main}解的连续半群 $\Brace{S(t)}_{t>0}$:
    \begin{equation*}
        S(t)u_0 = u(t),
    \end{equation*}
    其中 $S(t)$ 在 $u_{0} \in L^2(\Omega)$ 上关于 $t$ 连续.

\end{frame}
\begin{frame}{\secname : \subsecname}

    \begin{theorem}\label{thm:real_absorb}
        若 $a(x)$ 在\cref{thm:absorb}里的条件下同时满足
        \begin{equation}\label{eq:a_condition}
            \int_{\Omega}a^{-\frac{2}{p-2}} < \infty,
        \end{equation}
        则半群 $\Brace{S(t)}_{t \geq 0} $ 在
        $L^2$ 和 $W_0^{1,p}(a,\Omega)$ 上分别存在有界吸收集, 即对任意有界子集
        $B \subset L^2(\Omega)$, 存在常数 $T(\norm{u_0}_2)$ 及 $\rho > 0$, 使得
        \begin{equation*}
            \norm{u(t)}_2^2 + \int_{\Omega}a\abs{\nabla u}^p \leq \rho,
        \end{equation*}
        对于所有 $t \geq T$ 和 $u_0 \in B$, 其中 $u(t) = S(t)u_0$.
    \end{theorem}

\end{frame}
\begin{frame}{\secname : \subsecname}

    再由紧嵌入定理
    $W_0^{1,2}(\Omega) \subset\subset L^2(\Omega)$
    以及
    \begin{equation}\label{eq:DL2_bd_W1pa}
		\int_{\Omega}\abs{\nabla u}^2
		= \int_{\Omega}a^{-\frac{2}{p}}a^{\frac{2}{p}}\abs{\nabla u}^2
		\leq \left(\int_{\Omega}a^{-\frac{2}{p-2}}\right)^{\frac{p-2}{p}}
		\left(\int_{\Omega}a\abs{\nabla u}^p\right)^{\frac{2}{p}}
	\end{equation}
    立即得到 $L^2(\Omega)$ 中全局吸引子的存在性.
    \begin{theorem}\label{thm:attractor_L2}
        若 $a(x)$ 同时满足条件\eqref{eq:a_condition},
        则由方程\eqref{eq:main}的弱解生成的半群 $\Brace{S(t)}_{t \geq 0}$ 在
        $L^2(\Omega)$ 存在全局吸引子 $\mathcal{A}_2$.
    \end{theorem}

\end{frame}
\subsection{\texorpdfstring{$W_0^{1,p}(\Omega)$}上全局吸引子的存在性}
\begin{frame}{\secname : \subsecname}

    \begin{theorem}\label{thm:ut_L2_bd}
        对任意有界子集 $B \subset L^2(\Omega)$,
        存在常数 $T' = T'(B) > 0$, 使得对 $\forall u_0 \in B, s \geq T'$, 有
        \begin{equation*}
            \norm{u_t(s)}_2^2 \leq M.
        \end{equation*}
    \end{theorem}

\end{frame}
% \begin{frame}{\secname : \subsecname}

%     记 $v = u_t$,
%     \begin{equation*}
% 		\begin{split}
% 			\frac{1}{2}\frac{d}{dt}\norm{v}_2^2
% 			+ \int_{\Omega}a\abs{\nabla u}^{p-2}\abs{\nabla v}^2
% 			&+ \int_{\Omega}\left(p-2\right)a\abs{\nabla u}^{p-4}\left(\nabla u
% 			\cdot \nabla v\right)^2\\
% 			&+ 2\int_{\Omega}b\nabla u \cdot \nabla v v
% 			= 0.
% 		\end{split}
% 	\end{equation*}

% \end{frame}
% \begin{frame}{\secname : \subsecname}

%     \begin{equation*}
% 		\begin{split}
% 			\abs{\int_{\Omega}b\nabla u \cdot \nabla v v}
% 			&\leq \left(\int_{\Omega}b^2\left(\nabla u
% 			\cdot \nabla v\right)^2\right)^{\frac{1}{2}}
% 			\*\norm{v}_2\\
% 			&\leq \norm{b^2 a^{-1}}_{\infty}^{\frac{1}{2}}
% 			\left(\int_{\Omega}a\abs{\nabla u}^2\abs{\nabla v}^2\right)^{\frac{1}{2}}\norm{v}_2\\
% 			&\leq \norm{b^2 a^{-1}}_{\infty}^{\frac{1}{2}}
% 			\left(\int_{\Omega}a\left(\abs{\nabla u}+1\right)^2\abs{\nabla v}^2\right)^{\frac{1}{2}}\norm{v}_2\\
% 			% &= \norm{b^2 a^{-1}}_{\infty}^{\frac{1}{2}}
% 			% \left(
% 			% \int_{\Omega}a\left(
% 			% \left(\abs{\nabla u}+1\right)^2 + 1 - 2\left(\abs{\nabla u}+1\right)
% 			% \right)\abs{\nabla v}^2
% 			% \right)^{\frac{1}{2}}\norm{v}_2\\
% 			&\leq \norm{b^2 a^{-1}}_{\infty}^{\frac{1}{2}}
% 			\left(\int_{\Omega}a\left(\abs{\nabla u}+1\right)^{p-2}\abs{\nabla v}^2\right)^{\frac{1}{2}}\norm{v}_2\\
% 			&\leq \epsilon \int_{\Omega}a\left(\abs{\nabla u}+1\right)^{p-2}\abs{\nabla v}^2
% 			+ C(\epsilon) \norm{v}_2^2,
% 		\end{split}
%     \end{equation*}

% \end{frame}
% \begin{frame}{\secname : \subsecname}

%     \begin{equation*}
% 		\begin{split}
% 			\int_{\Omega}a\abs{\nabla u}^{p-2}\abs{\nabla v}^2
% 			\geq 2^{3-p}\int_{\Omega}a\left(\abs{\nabla u}+1\right)^{p-2}\abs{\nabla v}^2
% 			- \int_{\Omega}a\abs{\nabla v}^2.
% 		\end{split}
%     \end{equation*}
%     得到
%     \begin{equation}\label{eq:DvL2_bd_vL2}
% 		\begin{split}
% 			& \frac{1}{2}\frac{d}{dt}\norm{v}_2^2
% 			+ C\int_{\Omega}a\left(\abs{\nabla u}+1\right)^{p-2}\abs{\nabla v}^2\\
% 			+{} & \int_{\Omega}\left(p-2\right)a\abs{\nabla u}^{p-4}\left(\nabla u
% 			\cdot \nabla v\right)^2
% 			\leq C\norm{v}_2^2 + C.
% 		\end{split}
% 	\end{equation}
% 	进一步
% 	\begin{equation*}
% 		\begin{split}
% 			\frac{d}{dt}\left(
% 			\norm{v}_2^2 + C\int_{\Omega}a\abs{\nabla u}^p
% 			\right)
% 			\leq C,
% 		\end{split}
%     \end{equation*}
%     对上式作 $[t, t+1]$ 积分, 应用一致 Gr\"onwall 引理.

% \end{frame}
\begin{frame}{\secname : \subsecname}

    \begin{theorem}
        若 $a(x)$ 满足\cref{thm:absorb}的假设以及条件\eqref{eq:a_condition},
        则由方程\eqref{eq:main}得到的半群 $\Brace{S(t)}_{t \geq 0}$ 在 $W_0^{1,p}(\Omega)$
        中存在全局吸引子 $\mathcal{A}_V$, 即
        $\mathcal{A}_V$ 在 $W_0^{1,p}(\Omega)$ 中紧, 不变并且在 $W_0^{1,p}(\Omega)$
        拓扑下吸引 $L^2(\Omega)$ 中的任意有界子集.
    \end{theorem}

\end{frame}
\section{全局吸引子的上半连续性}
\subsection{整体思路}
\begin{frame}{\secname : \subsecname}

    考虑全局吸引子 $\mathcal{A}_\lambda$, 当 $\lambda \to \lambda_0 \in [0, \infty)$ 时的上半连续性, 即对
    $\forall \epsilon > 0$, 存在 $\delta > 0$, 当 $\abs{\lambda - \lambda_0} < \delta$ 时, 有
    \begin{equation*}
        \dist(\mathcal{A}_\lambda, \mathcal{A}_{\lambda_0}) < \epsilon,
    \end{equation*}
    其中 $\dist(A, B) = \sup_{a \in A}\inf_{b \in B}d(a, b)$ 表示集合 $A$, $B$ 间的 Hausdorff 半距离.
    即 $\mathcal{A}_\lambda \subset \mathcal{N}(\mathcal{A}_{\lambda_0}, \epsilon)$, 因此上半连续性可以保证吸引子在参数扰动下不爆破.

\end{frame}
\begin{frame}{\secname : \subsecname}

    \begin{itemize}
        \item 对所有的半群存在一个一致有界吸收集(容易验证)
        \begin{equation*}
            \bigcup_{0 \leq \lambda < \infty} \mathcal{A}_\lambda \subset B.
        \end{equation*}
        \item 扰动半群关于扰动系数在有界集内一致收敛
        \begin{equation*}
            \sup_{x \in B}\norm{S_\lambda(t)x - S_{\lambda_0}(t)x} \to 0, \quad \lambda \to \lambda_0.
        \end{equation*}
    \end{itemize}

\end{frame}
\subsection{模型}
\begin{frame}{\secname : \subsecname}

    \begin{equation}\label{eq:turb_main}
        \begin{cases}
            u_{\lambda t} = \Div(a_{\lambda}(x)\abs{\nabla u_{\lambda}}^{p-2}\nabla u_{\lambda}) - b(x)\abs{\nabla u_{\lambda}}^2
            \quad &(x, t) \in \Omega \times \R^+,\\
            u_{\lambda}(x,0) = u_{0\lambda} \quad &x \in \Omega,\\
            u_{\lambda}(x, t) = 0 \quad &x \in \partial\Omega,
        \end{cases}
    \end{equation}
    其中, $\lambda \in [0, \infty)$. 当 $\lambda \to \lambda_0$ 时, $\norm{a_\lambda - a_{\lambda_0}}_{L^\infty(\Omega)} \to 0$.
    方程\eqref{eq:turb_main}满足方程\eqref{eq:main}的条件与假设, 并有如下假设.

\end{frame}
\begin{frame}{\secname : \subsecname}

    对 $\forall \lambda$, 存在 $M_1, M_2 > 0$, 使得
    \begin{gather}
        \int_{\Omega} b^{\frac{2p}{p-4}}a_\lambda^{-\frac{4}{p-4}} \leq M_1, \label{h1} \\
        \int_{\Omega}a_\lambda^{-\frac{2}{p-2}} \leq M_2. \label{h2}
    \end{gather}

\end{frame}
\subsection{主要定理}
\begin{frame}{\secname : \subsecname}

    为了讨论全局吸引子的上半连续性, 我们需要如下引理,
    \begin{lemma}\label{lem:slambdacontinuous}
        令 $\Brace{S_\lambda(t) \colon \lambda \in [0, \infty)}$ 是由方程 \eqref{eq:turb_main} 生成的解半群族.
        当 $\lambda \to \lambda_0 \in [0, \infty)$ 时, 在 $L^2(\Omega)$ 中, $u_{0\lambda} \to u_{0\lambda_0}$.
        则对任意固定的 $T > 0, \forall t \in [0, T]$ 有
        \begin{equation*}
            \norm{S_\lambda(t)u_{0\lambda} - S_{\lambda_0}(t)u_{0\lambda_0}}_2 \to 0
        \end{equation*}
    \end{lemma}

\end{frame}
\begin{frame}{\secname : \subsecname}

    最后我们给出本章的主要定理.
    \begin{theorem}
        对 $\forall \lambda_0 \in [0, \infty)$, 
        方程\cref{eq:turb_main}的全局吸引子 $\mathcal{A}_\lambda$ 在 $\lambda = \lambda_0$ 处是上半连续的, 即
        \begin{equation*}
            \lim_{\lambda \to \lambda_0} \dist(\mathcal{A}_\lambda, \mathcal{A}_{\lambda_0}) = 0.
        \end{equation*}
    \end{theorem}

\end{frame}

\section{致谢}
\begin{frame}{\secname}

    \begin{itemize}
        \item 感谢答辩委员会各位专家教授!
        \item 特别感谢导师马闪副教授的悉心指导和帮助!
        \item 感谢兰州大学数学与统计学院的每一位老师和同学的帮助!
        \item 感谢参加论文答辩的所有老师、同学和朋友!
    \end{itemize}

\end{frame}
\end{document}