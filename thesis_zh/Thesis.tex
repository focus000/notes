% !TEX TS-program = xelatex
% !BIB program = bibtex
% !TEX TS-program = xelatex
% !TEX TS-program = xelatex
% !Mode:: "TeX:UTF-8"   %%winedt 以utf8编码打开

%双行标题
\documentclass[twoside,longtitle]{LZUthesis}
%单行标题
%\documentclass[twoside]{LZUthesis}

\usepackage{float}
\usepackage{fancybox}
\usepackage{calc}
\usepackage{mathdots}
\usepackage{graphicx}
\usepackage{listings}

%user settings begin
\usepackage{amssymb}
\usepackage{mathtools}
\usepackage{cleveref}
\crefname{equation}{式}{式}
\crefname{figure}{图}{图}
\crefname{table}{表}{表}
\crefname{appendix}{附录}{附录}
\crefname{chapter}{章}{章}
\crefname{theorem}{定理}{定理}
\crefname{lemma}{引理}{引理}
\newcommand{\crefpairconjunction}{~和~}
\newcommand{\crefmiddleconjunction}{、}
\newcommand{\creflastconjunction}{~和~}
\newcommand{\crefpairgroupconjunction}{~和~}
\newcommand{\crefmiddlegroupconjunction}{、}
\newcommand{\creflastgroupconjunction}{~和~}
\newcommand{\crefrangeconjunction}{~}
% \usepackage{hyperref}
% \bibliographystyle{plain}
% \usepackage[backend=bibtex, style=lzubib]{biblatex}
% \addbibresource{bib/Remote.bib}
% \addbibresource{bib/DynamicalSystem.bib}
% \addbibresource{bib/background.bib}
% \addbibresource{bib/pre.bib}
% \addbibresource{bib/intro.bib}

\newtheorem{theorem}{定理}[chapter]
\newtheorem{corollary}[theorem]{推论}
\newtheorem{lemma}[theorem]{引理}
\newtheorem{proposition}[theorem]{命题}
\newtheorem{remark}[theorem]{注记}
\theoremstyle{definition}
\newtheorem{definition}[theorem]{定义}
\newtheorem{proofpart}{part}
\makeatletter
\@addtoreset{proofpart}{theorem}
\makeatother

\numberwithin{equation}{chapter}

% \providecommand*{\lemmaautorefname}{Lemma}
% \providecommand*{\corollaryautorefname}{Corollary}
% \providecommand*{\propositionautorefname}{Proposition}
% \providecommand*{\remarkautorefname}{Remark}
% \providecommand*{\definitionautorefname}{Definition}

\newcommand*\abs[1]{\lvert#1\rvert}
\newcommand*\norm[1]{\lVert#1\rVert}
\newcommand*\Brace[1]{\lbrace#1\rbrace}
% \newcommand*\innerproduct[1]{\langle#1\rangle}

\newcommand\R{\mathbb{R}}
\newcommand*\Laplace{\mathop{}\!\mathbin\bigtriangleup}

\DeclareMathOperator{\Div}{div}
\DeclareMathOperator{\dist}{dist}
%user settings end

%声明图片后缀名
\DeclareGraphicsExtensions{.pdf,.eps}

\makeatletter

% 设置图形文件的搜索路径
\graphicspath{{figures/}}

% 小节标题靠左对齐
\ctexset{section={format+={\flushleft}}}
% \CTEXsetup[format+={\flushleft}]{section}

% 取消链接的颜色(黑白打印时启用)
%\hypersetup{colorlinks=false}

%使文档居中, 打印时应注释掉
\evensidemargin 0.93 true cm
\setlength{\hoffset}{-0.3 cm}

%允许equationarray分页换行
\allowdisplaybreaks

%使字体清晰, 且透明位图不会使页面文字粗细不一
\usepackage{eso-pic}
\AddToShipoutPicture{%
\special{pdf: put @thispage <</Group << /S /Transparency /I true /CS /DeviceRGB>> >>}%
}

%页面背景色
%\definecolor{yellow}{rgb}{0.99,0.99,0.70}
%\pagecolor{yellow}

%浮动项超链接正确跳转
\usepackage[all]{hypcap}

\makeatother


\begin{document}

%分类号;一般不要求学生填写
\classification{}

%密级;申请保密后需填写
\confidential{}

%中文标题
\title{带有阻尼项的加权$p$-laplace方程}

%中文标题第二行, 题目较短时删除之, 并去除文档类选项 longtitle
\titleadd{的全局吸引子}

%英文标题
\englishtitle{Global attractors of weighted $p$-laplace}

%英文题目第二行, 题目较短时删除之, 并去除文档类选项 longtitle
\englishtitleadd{equation with damping term}

%作者汉语姓名
\author{李蕴方}

%专业
\major{数学•应用数学}

%学位
\degree{硕士}

%研究方向
\direction{无穷维动力系统及其吸引子}

%导师姓名+职称
\advisor{马闪~副教授}

%论文工作起止年月
\datebeginAndEnd{2019年9月至2020年5月}

%论文提交日期
\submitdate{}

%答辩日期
\defenddate{}

%学位授予日期
\degreedate{}

%左侧页眉
\lzuthesis{兰州大学硕士学位论文}

%生成封面
\maketitle

%生成声明页
\makestatement

%前文-罗马页码
\frontmatter\pagenumbering{Roman}

%中文摘要
\begin{abstract}
	本文主要研究了以下带有阻尼项的加权$p$-laplace方程解的存在唯一性以及解的长时间行为.
	\begin{equation*}
		\begin{alignedat}{2}
			& u_t = \Div(a(x)\abs{\nabla u}^{p-2}\nabla u) - b(x)\abs{\nabla u}^2 \quad &\text{in } \Omega \times \R^+,\\
			& u(x,0) = u_0 \quad &\text{in } \Omega,\\
			& u = 0 \quad &\text{on } \partial\Omega,
		\end{alignedat}
	\end{equation*}
	其中 $\Omega$ 是在 $\R^{n}$ 里的有界光滑开区域, 边界记为 $\partial\Omega$, $p>1$.
	$ a(x)$, $b(x) \in C^1(\bar{\Omega}) $, $b(x) \geq 0$, 在 $\Omega$ 内部 $a(x) > 0$, 在边界 $\partial\Omega$ 上 $a(x) = 0$.

	若 $a(x)$, $b(x)$ 进一步的满足一些可积性条件, 如\cref{eq:zhan_intcondition}, 在~\citep{Zhan2019Uniquenessa}里作者给出了初值满足
	\cref{eq:zhan_initdata}时非负解的存在唯一性. 本文将初值条件放宽到 $L^2(\Omega)$ 上证明了解的存在唯一性.

	随后, 本文还研究了该方程的解的长时间行为, 通过先验估计得到了 $L^2(\Omega)$ 和 $W_0^{1,p}(\Omega)$ 上全局吸引子的存在性.
\end{abstract}

%中文关键词
\keywords{全局吸引子, 全局解, 加权$p$-laplace方程}

%英文摘要
\begin{englishabstract}
	This article is devoted to study the asymptotic dynamics for a dynamical system generated by a nonlinear
	weighted $p$-laplace equation that reads
	\begin{equation}\label{eq:main}
		\begin{alignedat}{2}
			& u_t = \Div(a(x)\abs{\nabla u}^{p-2}\nabla u) - b(x)\abs{\nabla u}^2 \quad &\text{in } \Omega \times \R^+,\\
			& u(x,0) = u_0 \quad &\text{in } \Omega,\\
			& u = 0 \quad &\text{on } \partial\Omega,
		\end{alignedat}
	\end{equation}
	where $\Omega$ is a bounded open domain in $\R^{n}$ with a sufficiently smooth boundary $\partial\Omega$, $p>1$.
	$ a(x)$, $b(x) \in C^1(\bar{\Omega}) $, $b(x) \geq 0$, and $a(x) > 0$ in $\Omega$, $a(x) = 0$ on $\partial\Omega$.

	if $a(x)$, $b(x)$ have extra conditions, Zhan has proved the existence and uniqueness of solution of this equation
	which has positive initial data in $L^{\infty}(\Omega) \cap W_0^{1,p}(a,\Omega)$. In this article, we released the
	initial data condition to $L^2(\Omega)$, then give the existence and uniqueness of solution either.

	Finally, this article gives the existence of global attractor in $L^2(\Omega)$ and $W_0^{1,p}(\Omega)$.
\end{englishabstract}

%英文关键词
\englishkeywords{Global attractor, global solution, weighted $p$-laplacian}

%自动生成章节目录以及pdf书签
\tableofcontents{}


%正文部分-数字页码
\mainmatter

%正文页眉页脚样式
\pagestyle{lzu}


\chapter{绪论}
\section{研究背景}
对非线性动力系统对研究是理解自然科学里许多重要问题对核心,
其中最著名的两类问题分别是天体力学和流体力学,
对于前者往往是有限维的而后者往往是无限维的,
这里的维数是指描述给定系统所需参数的数量.
随着科技的发展, 非线性动力系统在
分子动力学, 等离子物理学和激光, 非线性光学, 燃烧, 数学经济学, 机器人控制等领域具有广泛等应用.

不同于线性系统, 非线性系统的演化很难通过直觉和简单的计算得出,
且往往解是分叉的、混沌的、对初值敏感的,
正是由于其复杂性和对某些变化的敏感性, 这类系统的演化无法仅通过解析或者数值计算来预测,
为此 S.Smale, D.Ruelle和F.Takens等人提出了吸引子的概念来描述系统的长时间行为,
并做了一系列的工作.

考虑自治方程
\begin{equation}\label{eq:autonomous}
	\begin{gathered}
		u_t = Au,\\
		u(x, 0) = u_0(x)
	\end{gathered}
\end{equation}
若存在 Banach 空间 $X$ 以及时间 $T$, 使得\cref{eq:autonomous}的解 $u \in C([0, T], X)$ 是存在唯一的,
我们可以定义\cref{eq:autonomous}解的连续半群 $S(t) \colon X \to X$ 满足:
\begin{enumerate}
	\item $S(0) = I,$
	\item $S(t + s) = S(t)S(s), \quad \forall t, s \geq 0,$
	\item $\norm{S(t)u_0 - u_0} \to 0, \quad\forall u_0 \in X$ 当 $t \to 0$.
\end{enumerate}
解半群是为了描述吸引子的前置概念, 接下来定义半群 $S(t)$ 的吸引集 $K$,
即 $\exists K \subset X$
对 $\forall B \subset X, \epsilon > 0$, $\exists T(B, \epsilon)$,
当 $t > T$ 时有 $S(t)B$ 在 $K$ 的 $\epsilon-$临域内,
其中 $B, K$ 为有界集, 也可等价的定义为
\begin{equation}
	\lim_{t \to \infty} \dist_H(S(t)B, K) = 0,
\end{equation}
其中 $\dist_H(A, B) \coloneqq \sup_{x \in A}\inf_{y \in B}d(x, y)$ 为 Hausdorff 半距离.
所谓半群 $S(t)$ 的全局吸引子即满足以下条件的 $\mathcal{A} \subset X$:
\begin{enumerate}
	\item $\mathcal{A}$ 是紧集,
	\item $\mathcal{A}$ 是严格不变的: $S(t)\mathcal{A} = \mathcal{A}, \quad\forall t \geq 0$,
	\item $\mathcal{A}$ 是半群 $S(t)$ 的吸引集.
\end{enumerate}
通过\citep{efendievAttractorsDegenerateParabolic2013b}, 我们了解到吸引子本身包含所有解轨道的极限状态,
故全局吸引子可以反应系统的长时间行为, 吸引子的存在性也是研究系统的长时间行为的基础.
关于吸引子存在性的证明有一个著名的存在性定理:
\begin{theorem}[\citep{efendievAttractorsDegenerateParabolic2013b}]\label{thm:attractorexist}
	若连续半群 $S(t): X \to X$ 具有紧的吸引集, 则存在全局吸引子 $\mathcal{A}$.
\end{theorem}
由\cref{thm:attractorexist}一般有以下证明思路:
\begin{enumerate}
	\item 验证一致紧性, 即证明在空间 $X$ 存在紧的一致有界吸收集,
	一般方法是证明在空间 $X‘$ 存在一致有界吸收集, 且 $X'$ 可以紧嵌入到 $X$,
	一般用 Sobolev 紧嵌入定理得到. 通常要求方程具有更高的正则性.
	\item 验证渐进紧性, 所谓渐进紧性, 即对任意有界列 $\Brace{u_k} \subset X$ 和
	$\forall \Brace{t_k}, \quad t_k \to \infty$, 有 $S(t_k)u_k$ 在 $X$ 里列紧.
	一般验证方式是利用半群对分解, 一般在缺乏紧嵌入定理的时候用这种方式.
\end{enumerate}

另一方面, 无限维系统是由偏微分方程去描述的, 与常微分方程不同, 偏微分方程不存在解的存在性和唯一性的一般性定理.
这使得我们在做无穷维动力系统研究时要先得到对应发展方程解的适定性.

对于经典的带有二阶微分算子的发展方程解的适定性, 正则性, 吸引子的存在性问题可参考\citep{taylorPartialDifferentialEquations2011,temamInfiniteDimensionalDynamicalSystems1997},
在此不再赘述.

对于抛物型 $p$-laplace 方程, 其在过去几十年受到了广泛的关注
(见 \citep{liuAsymptoticRegularityPLaplacian2010,zhongZ2IndexGlobal2010,acerbiRegularityResultsStationary2002,rajagopalMathematicalModelingElectrorheological2001,aboulaichNewDiffusionModels2008,guoSingularPhenomenaSolutions2015,antontsevUniquenessComparisonTheorems2013,gaoExistenceUniquenessNonexistence2016,liuNonlinearDiffusionProblem2019,guoStudyWeakSolutions2011,antontsevParabolicEquationsAnisotropic2007,constantinGlobalExistenceFully2006,constantinGlobalSolutionsQuasilinear2002},
). 由于 $p$-laplace 算子 $\Div(\abs{\nabla u}^{p-2}\nabla u)$ 是非线性的, 我们无法用谱理论去处理它, 故对于很多 laplace 算子对应的结果无法轻易的推广过来.
对于抛物型 $p$-laplace 方程的研究最早可追溯到 1985 年 DiBenedetto 和 Friedman 的研究成果以及 1986 年 Wiegner 的研究成果,
他们先后给出了
\begin{equation*}
	u_t = \Div(\abs{\nabla u}^{p - 2}\nabla u)
\end{equation*}
解在 $C^{1, \alpha}$ 的正则性, 对于方程
\begin{equation}\label{eq:plaplaceprime}
	u_t = \Div(\abs{\nabla u}^{p - 2}\nabla u) - f(u) + g,
\end{equation}
通常用广义 Galerkin 法或者非退化逼近方程法证明解的适定性, 见\citep{babinAttractorsEvolutionEquations1992a},
对于\cref{eq:plaplaceprime}解以及吸引子的一般的结果可参考\citep{efendievAttractorsDegenerateParabolic2013b}.
值得一提的是, DiBenedetto 在\citep{dibenedettoDegenerateParabolicEquations1993a}里系统的论述了一般的退化抛物方程理论,
并用 De Giorgi 迭代技巧证明了 $L^\infty$ 估计, 这在用非退化逼近方程法证明解的适定性很有用.

近年来, 由于PDEs理论的发展以及加权$p$-laplace方程在
图像去噪, 沙堆的生长/坍缩模型, 非牛顿流体模型, II型超导理论等
应用科学的许多分支的应用, 此类方程引起了人们的兴趣.
(见 \citep{aronssonFastSlowDiffusion1996,aubertMathematicalProblemsImage2006,mastorakisSolutionPLaplacianNonNewtonian2009,yinLaplacianTypeEvolution2001}).
因此, 许多学者在此类方程上做了很多工作
(\citep{cortazarExistenceSignChanging2014,musinaExistenceMultiplicityResults2009,gazziniSobolevtypeInequalityRelated2009,liLongtimeBehaviorClass2014b,maGlobalAttractorsWeighted2012a,cavalheiroWeightedSobolevSpaces2008,caldiroliVariationalDegenerateElliptic2000,leBoundaryValueProblems1998,monticelliMaximumPrinciplesWeak2009,dibenedettoDegenerateSingularParabolic1993,galClassDegenerateParabolic2012,yinEvolutionaryWeightedPLaplacian2007,Zhan2019Uniquenessa}).
最近, 有许多论文考虑了此类方程的全局吸引子.
(见 \citep{anhGlobalExistenceLongtime2008,anhGlobalAttractorMsemiflow2010,anhLongtimeBehaviorQuasilinear2009,karachaliosConvergenceAttractorsDegenerate2005,karachaliosDynamicsDegenerateParabolic2006,karachaliosGlobalAttractorsConvergence2005})

对于方程
\begin{equation}
	u_t = \Div(a(x)\abs{\nabla u}^{p-2}\nabla u) + f(u, x, t), \quad (x, t) \in Q_T = \Omega \times (0, T),
\end{equation}
文献\citep{maGlobalAttractorsWeighted2012a}证明了初值在 $L^2(\Omega)$, $f(u, x, t) = g(x) - f(u)$, 其中 $f(u)$ 满足增长性条件时解的适定性及全局吸引子的存在性.
文献\citep{zhanParabolicEquationRelated2016}证明了当
$a(x) = \rho^\alpha$, $f(u, x, t)$ 是 Lipschitz 函数, 初值在 $L^\infty(\Omega)$, 其中 $\rho(x) = dist(x, \partial \Omega), \alpha > 0, p > 1$ 时
弱解的存在唯一性. 文献\citep{Zhan2019Uniquenessa}证明了当 $f(u, x, t) = -b(x)\abs{\nabla u}^2$, 初值在 $L^\infty(\Omega)$ 时解的适定性, 但对于吸引子还未有学者研究.

\section{本文研究的问题}
本文主要研究了对于一类加权$p$-laplace方程导出的动力系统的渐进动力学行为,
对于如下模型:
\begin{equation}\label{eq:main}
	\begin{alignedat}{2}
		& u_t = \Div(a(x)\abs{\nabla u}^{p-2}\nabla u) - b(x)\abs{\nabla u}^2 \quad &\text{in } \Omega \times \R^+,\\
		& u(x,0) = u_0 \quad &\text{in } \Omega,\\
		& u = 0 \quad &\text{on } \partial\Omega,
	\end{alignedat}
\end{equation}
其中 $\Omega$ 是在 $\R^{n}$ 里的有界光滑开区域, 边界记为 $\partial\Omega$, $p>1$.
$ a(x)$, $b(x) \in C^1(\bar{\Omega}) $, $b(x) \geq 0$, 在 $\Omega$ 内部 $a(x) > 0$, 在边界 $\partial\Omega$ 上 $a(x) = 0$.

在\citep{Zhan2019Uniquenessa}中, 作者证明了初值在 $L^\infty(\Omega)$ 时正解的适定性,
本文将初值条件从\citep{Zhan2019Uniquenessa}里的$u_0 \in L^{\infty}(\Omega) \cap W_0^{1,p}(a,\Omega)$的正值
放宽到 $u_0 \in L^2(\Omega)$ 上的任意值, 并得到了相应解的存在唯一性.
这里需要构造逼近方程去
进一步的, 我们定义了 $L^2(\Omega) $ 上的解半群 $\Brace{S(t)}_{t \geq 0} $ 用以研究方程解的渐进行为.
然而, 阻尼项在一些估计上带来了很多困难, 首先在解的存在性证明里,
无法直接用 Gr\"onwall 不等式得到收敛性,
其次, 嵌入定理的缺失使得我们在 \cref{thm:ut_L2_bd} 里估计
$\norm{u_t(s)}_2^2$ 遇到了困难, 而这是证明 $W_0^{1,p}(\Omega)$ 里
全局吸引子的存在性的核心估计.

在 \cref{ch:preliminaries},
我们给出将会用到的一些结果以及技术性引理. 在
\cref{ch:Existence_and_uniqueness_of_the_weak_solution}
我们证明弱解的存在唯一性.
在 \cref{ch:existence_of_the_global_attractors}
我们给出全局吸引子在
$L^2(\Omega)$ 和 $W_0^{1,p}(\Omega)$ 上的存在性证明.

\chapter{预备知识和假设}\label{ch:preliminaries}
在本章中, 我们将介绍泛函空间和一些有用的引理,
其他文献里有用的结果也将放在这里

现在我们给出加权 sobolev 空间的定义.
$\Omega$ 是 $\R^n$ 里的有界区域, 权重
$a \colon \R^n \to [0, \infty)$
是一个非负的局部可和函数.
\begin{definition}
	带权 Lebesgue 空间 $L^p(a, \Omega)$, $1 \leq p < \infty$,
	是由局部可和函数类
	$u \colon \Omega \to \R$ 在如下范数下定义的 Banach 空间
	\begin{equation}
		\norm{u}_{L^p(a,\Omega)} =
		\left( \int_{\Omega}a\abs{u}^p \right)^{\frac{1}{p}}.
	\end{equation}
\end{definition}
\begin{definition}
	带权 Sobolev 空间 $W^{k,p}(a,\Omega)$,
	$1 \leq k < \infty$, $1 \leq p < \infty$,
	是由 $k$ 阶弱可微局部可和函数
	$u \colon \Omega \to \R$ 关于如下范数下的 Banach 空间
	\begin{equation}
		\norm{u}_{W^{k,p}(a,\Omega)} =
		\left( \int_{\Omega}a\abs{u}^p \right)^{\frac{1}{p}}
		+ \sum_{\abs{\alpha}=k}
		\left( \int_{\Omega}a\abs{D^{\alpha}u}^p \right)^{\frac{1}{p}},
	\end{equation}
	这里 $\alpha$ 是一个多指标.

	另外,
	$W_0^{k,p}(a,\Omega)$ 被定义为
	$C_0^{\infty}(\Omega)$ 在如下范数下的完备空间
	\begin{equation}
		\norm{u}_{L^p(a,\Omega)} =
		\left( \int_{\Omega}a\abs{D^{\alpha}u}^p \right)^{\frac{1}{p}}.
	\end{equation}
\end{definition}
\begin{remark}
	一般 $W_0^{k,p}(a,\Omega)$ 在
	$W^{k,p}(a,\Omega)$ 不紧, 除非 $a$ 满足 Muckenhoupt 条件, 即 $a \in A_{p^-}$. 具体细节见~\cite{goldshteinWeightedSobolevSpaces2009}.
\end{remark}
接下来我们给出一个引理用于在证明解的存在性时帮助得到收敛性.
\begin{lemma}\label{lem:VecIneq}
	$\alpha$ 和 $\beta$ 是属于 $\R^{n}$ 的 $n$ 维向量, 如果 $p \geq 4$, 那么我们有
	\begin{equation}
		\abs{\alpha^2 - \beta^2}^{\frac{p}{2}}
		\leq C \langle \abs{\alpha}^{p-2}\alpha - \abs{\beta}^{p-2}\beta, \alpha-\beta\rangle
	\end{equation}
\end{lemma}
\begin{proof}
	断言当 $p \geq 1 $ 时
	\begin{equation}\label{eq:p_ineq}
		\abs{\alpha-\beta}^p \leq C\abs{\abs{\alpha}^{p-1}\alpha - \abs{\beta}^{p-1}\beta},
	\end{equation}
	首先 $n = 1$ 且 $\alpha$, $\beta \geq 0$ 时, 我们有
	\begin{equation}
		\abs{\alpha-\beta}^p \leq \abs{\alpha^p - \beta^p}.
	\end{equation}
	不是一般性, 假设 $\alpha \leq \beta$, 令
	\begin{equation}
		f(x) = (\beta-\alpha+x)^p - x^p,
	\end{equation}
	那么
	\begin{equation}
		f'(x) = p\left((\beta-\alpha+x)^{p-1} - x^{p-1}\right) \geq 0,
	\end{equation}
	于是得到
	\begin{equation}
		\left(\beta-\alpha\right)^p = f(0) \leq f(\alpha) = \beta^p - \alpha^p.
	\end{equation}
	另一方面, 应用 jensen 不等式我们得到
	\begin{equation}
		\left(\alpha+\beta\right)^p \leq 2^{p-1}\left(\alpha^p + \beta^p\right),
	\end{equation}
	以上我们证明了 $n = 1$ 时的 \cref{eq:p_ineq},
	对于 $n > 1$, 我们需要用到余弦定理, 令 $\gamma$ 是 $\alpha$ 和 $\beta$ 夹角, 我们有
	\begin{equation}
		\begin{split}
			\left(\abs{\alpha-\beta}^p\right)^2
			&= \left(\alpha^2 + \beta^2 - 2 \abs{\alpha}\abs{\beta}\cos{\gamma}\right)^p\\
			&\leq \left(\alpha^2+\beta^2\right)^p - 2^p\abs{\alpha}^p\abs{\beta}^p\cos^p{\gamma}\\
			&\leq 2^{p-1}\left(\alpha^{2p} + \beta^{2p} - 2\abs{\alpha}^p\abs{\beta}^p\cos^p{\gamma}\right)\\
			&\leq \left(2^{p-1}+C\right)\left(\alpha^{2p} + \beta^{2p}\right)
			- \left(2^{p-1}\cos^p{\gamma}+C\right)2\abs{\alpha}^p\abs{\beta}^p\\
			&\leq \left(2^{p-1}+C\right)\left(\alpha^{2p} + \beta^{2p} - 2\abs{\alpha}^p\abs{\beta}^p\cos{\gamma}\right)\\
			&= \left(2^{p-1}+C\right)\abs{\abs{\alpha}^{p-1}\alpha - \abs{\beta}^{p-1}\beta}^2,
		\end{split}
	\end{equation}
	事实上
	\begin{equation}
		\begin{split}
			& 2^{p-1}\cos^p{\gamma} + C - \left(2^{p-1}+C\right)\cos{\gamma}\\
			={} & 2^{p-1}\left(\cos^p{\gamma} - \cos{\gamma}\right) + C\left(1-\cos{\gamma}\right)
			\geq 0
		\end{split}
	\end{equation}
	当 $C$ 足够大时. 以上我们完整的证明了 \cref{eq:p_ineq}.

	另一方面, 我们有
	\begin{equation}
		\abs{\abs{\beta}^{p-1}\beta - \abs{\alpha}^{p-1}\alpha} \leq p\abs{\beta-\alpha}\int_0^1 \abs{\alpha + t(\beta - \alpha)}^{p-1}dt.
	\end{equation}
	注意到, 如果 $p \geq 2$, 我们有
	\begin{equation}\label{eq:VecIneq_1}
		\langle \abs{\alpha}^{p-2}\alpha - \abs{\beta}^{p-2}\beta, \alpha-\beta\rangle
		\geq \abs{\alpha-\beta}^2\int_0^1 \abs{\beta + t(\alpha - \beta)}^{p-2}dt,
	\end{equation}
	我们可以在~\cite{lindqvistNotesStationaryPLaplace2019} 找到~\cref{eq:VecIneq_1} 的证明.
	因此当 $p \geq 4$ 时
	\begin{equation}
		\begin{split}
			\abs{\alpha^2 - \beta^2}^{\frac{p}{2}}
			&\leq 2^{\frac{p}{2}}\abs{\alpha-\beta}^{\frac{p}{2}}
			\left(\int_0^1 \abs{\beta + t(\alpha - \beta)}dt\right)^{\frac{p}{2}}\\
			&\leq C\frac{p-2}{2}2^{\frac{p}{2}}\abs{\alpha-\beta}^2
			\left(\int_0^1 \abs{\beta + t(\alpha - \beta)}^{\frac{p-4}{2}}dt\right)\\
			&\times\left(\int_0^1 \abs{\beta + t(\alpha - \beta)}dt\right)^{\frac{p}{2}}\\
			&\leq C\frac{p-2}{2}2^{\frac{p}{2}}\abs{\alpha-\beta}^2
			\int_0^1 \abs{\beta + t(\alpha - \beta)}^{p-2}dt\\
			&\leq C \langle \abs{\alpha}^{p-2}\alpha - \abs{\beta}^{p-2}\beta, \alpha-\beta\rangle
		\end{split}
	\end{equation}
\end{proof}

最后, 我们需要如下定理帮助我们证明弱解的存在唯一性,
\begin{theorem}\cite[定理 1.3 和 1.6]{Zhan2019Uniquenessa}\label{thm:zhan}
	如果 $p>4$, $a(x)$, $b(x)$ 满足
	\begin{equation}\label{eq:zhan_intcondition}
		\int_{\Omega} b^{\frac{2p}{p-4}}a^{-\frac{4}{p-4}} \leq c,
	\end{equation}
	且 $u_0$ 满足
	\begin{equation}\label{eq:zhan_initdata}
		0 \leq u_0 \in L^{\infty}(\Omega), a(x)u_0 \in W_0^{1,p}(\Omega),
	\end{equation}
	那么 \cref{eq:main} 存在满足如下条件的唯一非负弱解
	\begin{equation}
		u \in L^{\infty}(Q_T), a(x)\abs{\nabla u}^p \in L^1(Q_T).
	\end{equation}
	且初值具有如下性质
	\begin{equation}
		\lim_{t \to 0}\int_{\Omega}\abs{u(x,t) - u_0(x)}dx = 0.
	\end{equation}
\end{theorem}

\chapter{弱解的存在唯一性}\label{ch:Existence_and_uniqueness_of_the_weak_solution}
本章我们证明在 $L^2(\Omega)$ 下弱解的存在唯一性.
由于初值在特定条件下方程已经有了适定性 (见 \cref{thm:zhan}),
我们可以直接使用这些初值逼近 $u_0 \in L^2$.
\begin{theorem}\label{thm:absorb}
	如果 $p>4$ 在~\cref{eq:main}, $a(x)$, $b(x)$ 满足
	\begin{equation}
		\int_{\Omega} b^{\frac{2p}{p-4}}a^{-\frac{4}{p-4}} \leq c,
	\end{equation}
	且 $u_0 \in L^2(\Omega) $, 那么方程存在弱解满足
	\begin{equation}
		u \in L^p(0, T; W_0^{1,p}(a,\Omega)), \quad u \in C([0, T]; L^2(\Omega)).
	\end{equation}
\end{theorem}

\begin{proof}
	首先, 我们基于 {\cite{Zhan2019Uniquenessa}} 证明初值 $u_0$ 满足如下条件时解的存在性
	\begin{equation}\label{initial_data_condition_tmp}
		u_0 \in L^{\infty}(\Omega) \cap W_0^{1, p}(a, \Omega).
	\end{equation}
	为此先构造如下渐进方程
	\begin{gather}
		u_{\epsilon t}-\Div\left((a(x)+\epsilon)
		\left(\left|\nabla u_{\epsilon}\right|^{2}+\epsilon\right)^{\frac{p-2}{2}} \nabla u_{\epsilon}\right)
		+b(x)\left|\nabla u_{\epsilon}\right|^{2} = 0,(x, t) \in Q_{T} \label{eq:approximated_maineq} \\
		u_{\epsilon}(x, t)  = 0, \quad(x, t) \in \partial \Omega \times(0, T)\\
		u_{\epsilon}(x, 0)  = u_{\epsilon, 0}(x), \quad x \in \Omega
	\end{gather}
	我们由PDE经典理论知道上述方程具有唯一弱解 $u_\epsilon$, 且 $u_\epsilon \in C([0, T], L^2(\Omega))$.

	通过与磨光子作卷积,
	可以找到一列 $u_{\epsilon,0} \in C_0^\infty(\Omega) $, $au_{\epsilon,0} \to au_0 $ 在 $W_0^{1,p}(\Omega) $,
	且 $\norm{u_{\epsilon, 0}}_{\infty}$ 一致有界,
	由极大值原理 $\abs{u_{\epsilon}} \leq C$,
	其中 $C$ 只与 $\norm{u_0}_{\infty} $ 有关.

	将 \cref{eq:approximated_maineq} 与 $u_\epsilon$ 作乘积, 并作 $Q_T$ 上的积分, 我们得到
	\begin{equation}
		\begin{split}
			\frac{1}{2} \int_{\Omega} u_{\epsilon}^{2}
			&+\iint_{Q_{T}}(a(x)+\epsilon)\left(\left|\nabla u_{\epsilon}\right|^{2}+\epsilon\right)^{\frac{p-2}{2}}\left|\nabla u_{\epsilon}\right|^{2}\\
			&+\iint_{Q_{T}} b(x)\left|\nabla u_{\epsilon}\right|^{2} u_{\epsilon}  =\frac{1}{2} \int_{\Omega} u_{\epsilon, 0}^{2}.
		\end{split}
	\end{equation}
	下面对每一项依次作估计, 事实上
	\begin{equation}
		\begin{split}
			\abs{\int_0^T\int_{\Omega} b\abs{\nabla u_{\epsilon}}^2u_{\epsilon}}
			&\leq C\int_0^T\left( \int_{\Omega} b^{\frac{p}{p-2}}a^{-\frac{2}{p-2}}\abs{u_{\epsilon}}^{\frac{p}{p-2}} \right)^{\frac{p-2}{p}}
			\left(  \int_{\Omega} a\abs{\nabla u_{\epsilon}}^p \right)^{\frac{2}{p}}\\
			&\leq C\int_0^T\left(\int_{\Omega}\abs{u_{\epsilon}}^2\right)^{\frac{p}{2(p-2)}} + \frac{1}{2}\int_0^T\int_{\Omega} a\abs{\nabla u_{\epsilon}}^p\\
			&\leq \eta\int_0^T\int_{\Omega}\abs{u_{\epsilon}}^2 + \frac{1}{2}\int_0^T\int_{\Omega} a\abs{\nabla u_{\epsilon}}^p + C(\eta)
		\end{split}
	\end{equation}
	我们有
	\begin{equation}
		\begin{split}
			\frac{1}{2} \int_{\Omega} u_{\epsilon}^{2}
			&+\iint_{Q_{T}}(a(x)+\epsilon)\left(\left|\nabla u_{\epsilon}\right|^{2}+\epsilon\right)^{\frac{p-2}{2}}\left|\nabla u_{\epsilon}\right|^{2}\\
			&\leq \frac{1}{2}\iint_{Q_T}a\abs{\nabla u_\epsilon}^p + C.
		\end{split}
	\end{equation}
	同时
	\begin{equation}
		\begin{split}
			\iint_{Q_T}a\abs{\nabla u_\epsilon}^p
			\leq \iint_{Q_{T}}(a(x)+\epsilon)\left(\left|\nabla u_{\epsilon}\right|^{2}+\epsilon\right)^{\frac{p-2}{2}}\left|\nabla u_{\epsilon}\right|^{2}
			\leq \frac{1}{2}\iint_{Q_T}a\abs{\nabla u_\epsilon}^p + C
		\end{split}
	\end{equation}
	于是得到
	\begin{equation}\label{eq:iint_adu}
		\begin{split}
			\iint_{Q_T}a\abs{\nabla u_\epsilon}^p \leq C.
		\end{split}
	\end{equation}
	事实上, \cref{eq:iint_adu} 就是 \cite[定理 1.3 式(2.5)]{Zhan2019Uniquenessa},
	剩下的证明与~\cite[thm 1.3]{Zhan2019Uniquenessa} 的证明相似, 在此略去并给出如下结论:
	\cref{eq:main} 在初值满足~\eqref{initial_data_condition_tmp} 时具有唯一弱解, 记为 $u$,
	且 $u \in C([0, T], L^2(\Omega))$.

	接下来我们将初值条件放宽到 $L^2(\Omega)$ 上.
	选择 $C_c^{\infty}(\Omega) \supset \Brace{u_{n, 0}}_{n=1}^{\infty} $
	在 $L^2(\Omega) $ 收敛到 $u_0$ 当 $n \to \infty $. $\forall u_{n, 0}$,
	由第一部分的结论, 存在唯一的弱解 $u_n$ 满足 \cref{eq:main}.
	于是我们有
	\begin{equation}
		\begin{split}
			& \frac{1}{2}\int_{\Omega}\left(u_n-u_m\right)^2(t)\\
			+{} & \int_{0}^{t}\int_{\Omega}a(x)
			\left(\abs{\nabla u_n}^{p-2}\nabla u_n
			- \abs{\nabla u_m}^{p-2}\nabla u_m\right)
			\left(\nabla u_n - \nabla u_m\right)\\
			={} & \int_{0}^{t}\int_{\Omega}b(x)\left(\abs{\nabla u_n}^2
			- \abs{\nabla u_m}^2\right)\left(u_n - u_m\right)
			+ \frac{1}{2}\int_{\Omega}\left(u_n-u_m\right)^2(0).
		\end{split}
	\end{equation}
	由 \cref{lem:VecIneq}, 我们有
	\begin{equation}
		\begin{split}
			& \int_{0}^{t}\int_{\Omega}b\left(\abs{\nabla u_n}^2
			- \abs{\nabla u_m}^2\right)\left(u_n - u_m\right)\\
			\leq{} & \left(\int_0^t\int_{\Omega}a\left(\abs{\nabla u_n}^2
			- \abs{\nabla u_m}^2\right)^{\frac{p}{2}}\right)^{\frac{2}{p}}
			\left(\int_0^t\int_{\Omega}\left(ba^{-\frac{2}{p}}
			\left(u_n-u_m\right)\right)^{\frac{p}{p-2}}\right)^{\frac{p-2}{p}}\\
			\leq{} & C\left(\int_0^t\int_{\Omega}a
			\left(\abs{\nabla u_n}^{p-2}\nabla u_n
			- \abs{\nabla u_m}^{p-2}\nabla u_m\right)
			\left(\nabla u_n - \nabla u_m\right)\right)^{\frac{2}{p}}\\
			\times{} & \left(\int_0^t\int_{\Omega}b^{\frac{2p}{p-4}}a^{-\frac{4}{p-4}}\right)^{\frac{p-4}{2p}}
			\left(\int_0^t\int_{\Omega}\left(u_n-u_m\right)^2\right)^{\frac{1}{2}}\\
			\leq{} & \frac{1}{2}\int_0^t\int_{\Omega}a
			\left(\abs{\nabla u_n}^{p-2}\nabla u_n
			- \abs{\nabla u_m}^{p-2}\nabla u_m\right)
			\left(\nabla u_n - \nabla u_m\right)\\
			+{} & C\left(\int_0^t\int_{\Omega}\left(u_n-u_m\right)^2\right)^{\frac{p}{2(p-2)}}.
		\end{split}
	\end{equation}
	联系上述不等式我们得到
	\begin{equation}\label{un-umL2_0TW1p_bd_0TL2_L20}
		\begin{split}
			& \int_{\Omega}\left(u_n-u_m\right)^2(t)\\
			+{} & \int_{0}^{t}\int_{\Omega}a
			\left(\abs{\nabla u_n}^{p-2}\nabla u_n
			- \abs{\nabla u_m}^{p-2}\nabla u_m\right)
			\left(\nabla u_n - \nabla u_m\right)\\
			\leq{} & C\left(\int_0^t\int_{\Omega}
			\left(u_n-u_m\right)^2\right)^{\frac{p}{2(p-2)}}
			+ \int_{\Omega}\left(u_n-u_m\right)^2(0).
		\end{split}
	\end{equation}
	当 $p=4$ 时, 通过 Gr\"onwall's 不等式有
	\begin{equation}
		\begin{split}
			\int_{\Omega}\left( u_n-u_m \right)^2(t)
			\leq \int_{\Omega}\left(u_n-u_m\right)^2(0)e^{Ct} ,
		\end{split}
	\end{equation}
	对几乎处处 $0 \leq t \leq T$. 当 $p>4$ 时,
	无法直接使用 Gr\"onwall's 不等式,
	因为 $\frac{p}{2(p-2)}<1$. 接下来用反证法证明当
	\begin{equation}
		\int_{\Omega}\left(u_n-u_m\right)^2(0) \to 0
	\end{equation}
	时
	\begin{equation}
		\int_0^t\int_{\Omega}\left(u_n-u_m\right)^2 \to 0.
	\end{equation}
	假设
	\begin{equation*}
		\int_0^t\int_{\Omega}\left(u_n-u_m\right)^2 \not\to 0,
	\end{equation*}
	即存在 $\Brace{u_n}$ 的子列, 同样记为 $\Brace{u_n}$ , 存在 $c_0 > 0, N > 0$ 使得
	\begin{equation*}
		\int_0^t\int_{\Omega}\left(u_n-u_m\right)^2 > c_0, \quad n, m > N.
	\end{equation*}
	于是有
	\begin{equation*}
		\left(\int_0^t\int_{\Omega}\left(u_n-u_m\right)^2\right)^{\frac{p}{2(p-2)}}
		\leq c_0^{\frac{4-p}{2p-2}}\int_0^t\int_{\Omega}\left(u_n-u_m\right)^2,
	\end{equation*}
	再由\cref{un-umL2_0TW1p_bd_0TL2_L20}得:
	\begin{equation}
		\int_{\Omega}\left(u_n-u_m\right)^2(t)
		\leq C\int_0^t\int_{\Omega}\left(u_n-u_m\right)^2
		+ \int_{\Omega}\left(u_n-u_m\right)^2(0).
	\end{equation}
	再通过 Gr\"onwall's 不等式, $n, m \to \infty$ 时:
	\begin{equation}
		\int_{\Omega}\left( u_n-u_m \right)^2(t)
		\leq \int_{\Omega}\left(u_n-u_m\right)^2(0)e^{Ct} \to 0.
	\end{equation}
	由控制收敛定理得到
	\begin{equation}
		\int_0^t\int_{\Omega}\left(u_n-u_m\right)^2 \to 0, \quad n, m \to \infty,
	\end{equation}
	导出矛盾. 对于几乎处处 $0 \leq t \leq T$, 让 $n,m \to \infty$, 有
	\begin{equation}\label{cauchy_in_L2}
		\norm{u_n(t,\cdot)-u_m(t,\cdot)}_{L^2(\Omega)} \to 0,
	\end{equation}
	以及
	\begin{equation}\label{cauchy_in_W1pa}
		\int_{0}^{t}\int_{\Omega}a
		\left(\abs{\nabla u_n}^{p-2}\nabla u_n
		- \abs{\nabla u_m}^{p-2}\nabla u_m\right)
		\left(\nabla u_n - \nabla u_m\right)
		\to 0,
	\end{equation}
	注意到
	\begin{equation}
		\begin{split}
			& \int_0^T\int_{\Omega}a\abs{\nabla u_n - \nabla u_m}^p\\
			\leq{} & 2^{p-2}\int_{0}^{T}\int_{\Omega}a
			\left(\abs{\nabla u_n}^{p-2}\nabla u_n
			- \abs{\nabla u_m}^{p-2}\nabla u_m\right)
			\left(\nabla u_n - \nabla u_m\right),
		\end{split}
	\end{equation}
	因此存在 $u \in L^p(0, T; W_0^{1,p}(a,\Omega))
		\cap L^2(0, T; L^2(\Omega))$ 使得 $n \to \infty$ 时
	\begin{equation}
		u_{n} \to u
	\end{equation}
	在 $L^p(0, T; W_0^{1,p}(a,\Omega))\cap L^2(0, T; L^2(\Omega))$.
	当 $n \to \infty$, 由 $u_n \in C([0, T], L^2(\Omega))$, 得到
	\begin{equation}
		\lim_{t' \to t}\norm{u(t,\cdot)-u(t',\cdot)}_{L^2(\Omega)}=0.
	\end{equation}
	即
	\begin{equation*}
		u \in C([0, T]; L^2(\Omega)).
	\end{equation*}

	最后我们考虑阻尼项, 应用 H\"older 不等式,
	对任意的 $v \in C_c^\infty$,
	\begin{equation}
		\begin{split}
			& \int_0^T\int_{\Omega}b\left(\abs{\nabla u_n}^2
			- \abs{\nabla u}^2\right)v\\
			\leq{} & C_p\left(\int_0^T\int_{\Omega}a
			\left(\abs{\nabla u_n}^{p-2}\nabla u_n
			- \abs{\nabla u}^{p-2}\nabla u\right)
			\left(\nabla u_n - \nabla u\right)\right)^{\frac{2}{p}}\\
			\times{} & \left(\int_0^T\int_{\Omega}b^{\frac{2p}{p-4}}a^{-\frac{4}{p-4}}\right)^{\frac{p-4}{2p}}
			\left(\int_0^T\int_{\Omega}v^2\right)^{\frac{1}{2}} \to 0
		\end{split}
	\end{equation}
	当 $n \to \infty$, 以上我们完成了整个证明.
\end{proof}
通过~\cite{Zhan2019Uniquenessa}里相同的方法可以得到唯一性.
现在我们可以定义解半群 $\Brace{S(t)}_{t>0}$,
通过
\begin{equation}
	S(t)u_0 = u(t),
\end{equation}
其中 $S(t)$ 在 $u_{0} \in L^2(\Omega)$ 关于 $t$ 连续.
\chapter{全局吸引子的存在性}\label{ch:existence_of_the_global_attractors}
\begin{theorem}
	半群 $\Brace{S(t)}_{t \geq 0} $ 在
	$L^2$ 和 $W_0^{1,p}(a,\Omega)$ 上分别存在有界吸收集, 即对任意有界子集
	$B \subset L^2(\Omega)$, 存在常数 $T(\norm{u_0}_2)$ 及 $\rho > 0$, 使得
	\begin{equation}
		\norm{u(t)}_2^2 + \int_{\Omega}a\abs{\nabla u}^p \leq \rho,
	\end{equation}
	对于所有 $t \geq T$ 和 $u_0 \in B$, 其中 $u(t) = S(t)u_0$.
\end{theorem}
\begin{proof}
	将 \cref{eq:main} 与 $u$ 相乘并在 $\Omega$ 上积分,
	得到
	\begin{equation}\label{eq:operate_u}
		\frac{1}{2}\frac{d}{dt}\norm{u}_2^2
		+ \int_\Omega a\abs{\nabla u}^p
		+ \int_\Omega b\abs{\nabla u}^2u = 0,
	\end{equation}
	因为 $p \geq 4$, 由 H\"older 不等式,
	\begin{equation}
		\int_{\Omega}\abs{\nabla u}^2
		= \int_{\Omega}a^{-\frac{2}{p}}a^{\frac{2}{p}}\abs{\nabla u}^2
		\leq \left(\int_{\Omega}a^{-\frac{2}{p-2}}\right)^{\frac{p-2}{p}}
		\left(\int_{\Omega}a\abs{\nabla u}^p\right)^{\frac{2}{p}}
	\end{equation}
	再由 Poincar\`e 不等式, 得到
	\begin{equation}
		\left(\int_{\Omega}\abs{u}^2\right)^{\frac{p}{2}}
		\leq C\int_{\Omega}a\abs{\nabla u}^p,
	\end{equation}
	其中 $C$ 只与 $\Omega$ 和 $p$ 有关. 对于第三项, 通过
	H\"older 不等式和 Young 不等式得到有界估计,
	\begin{equation}\label{eq:absorb_damping_u}
		\begin{split}
			\abs{\int_{\Omega}b\abs{\nabla u}^2 u}
			&\leq \left(\int_{\Omega}a\abs{\nabla u}^p\right)^{\frac{2}{p}}
			\left(\int_{\Omega}\abs{b a^{-\frac{2}{p}} u}^{\frac{p}{p-2}}\right)^{\frac{p-2}{p}}\\
			&\leq \frac{1}{2}\int_{\Omega}a\abs{\nabla u}^p
			+ 2^{\frac{p-2}{2}}\int_{\Omega}\abs{b a^{-\frac{2}{p}} u}^{\frac{p}{p-2}}\\
			&\leq \frac{1}{2}\int_{\Omega}a\abs{\nabla u}^p
			+ 2^{\frac{p-2}{2}}\left(\int_{\Omega}b^{\frac{2p}{p-4}}a^{-\frac{4}{p-4}}\right)^{\frac{p-4}{2(p-2)}}
			\left(\int_{\Omega}u^2\right)^{\frac{p}{2(p-2)}}\\
			&\leq \frac{1}{2}\int_{\Omega}a\abs{\nabla u}^p
			+ \epsilon \left(\int_{\Omega}u^2\right)^{\frac{p}{2}}
			+ C_{\epsilon}\left(\int_{\Omega}b^{\frac{2p}{p-4}}a^{-\frac{4}{p-4}}\right)^{\frac{p-4}{2(p-3)}},
		\end{split}
	\end{equation}
	令 $\epsilon$ 足够小, 并联系以上的估计, 得到
	\begin{equation}\label{eq:DL2_L2_bd}
		\frac{d}{dt}\norm{u}_2^2 + \norm{u}_2^p \leq C
	\end{equation}
	其中 $C$ 只与 $u$ 有关. 由 Gr\"onwall 不等式, 可以得到 $L^2(\Omega)$ 上的有界吸收集,
	即 $\exists \rho_0$ 和
	$T_0 = T_0(\norm{u_0}_2)$ 使得
	\begin{equation}\label{eq:L2_bd}
		\norm{u(t)}_2^2 \leq \rho_0 \text{ for } t \geq T_0.
	\end{equation}
	应用 \cref{eq:operate_u,eq:absorb_damping_u,eq:L2_bd}, 有
	\begin{equation}\label{eq:DL2_W1pa_bd}
		\frac{d}{dt}\norm{u}_2^2 + \int_{\Omega}a \abs{\nabla u}^p \leq C.
	\end{equation}
	将 \cref{eq:main} 与 $u_{t}$ 相乘并在 $\Omega$ 上积分得,
	\begin{equation}\label{eq:operate_ut}
		\norm{u_t}_2^2
		+ \frac{1}{p}\frac{d}{dt}\int_{\Omega}a \abs{\nabla u}^p
		+ \int_{\Omega}b \abs{\nabla u}^2 u_t
		= 0,
	\end{equation}
	像 \cref{eq:absorb_damping_u} 一样估计第三项, 可以得到
	\begin{equation}\label{eq:utL2_DW1pa_bd_W1pa}
		C\norm{u_t}_2^2
		+ \frac{d}{dt}\int_{\Omega}a \abs{\nabla u}^p
		\leq \int_{\Omega}a \abs{\nabla u}^p
		+ C,
	\end{equation}
	\cref{eq:DL2_W1pa_bd} $+$ \cref{eq:utL2_DW1pa_bd_W1pa} 得到
	\begin{equation}\label{eq:utL2_DW1pa_DL2}
		C\norm{u_t}_2^2
		+ \frac{d}{dt}\left(\int_{\Omega}a \abs{\nabla u}^p
		+ \norm{u}_2^2\right)
		\leq C.
	\end{equation}
	应用 \cref{eq:DL2_L2_bd}, 有
	\begin{equation}
		\frac{d}{dt}\norm{u}_2^2 + \norm{u}_2^2 \leq C
	\end{equation}
	因此
	\begin{equation}\label{eq:int_DL2_L2_bd}
		\norm{u(t+1)}_2^2
		+ \int_t^{t+1}\norm{u}_2^2
		\leq C + \norm{u(t)}_2^2,
	\end{equation}
	类似的, 应用 \cref{eq:DL2_W1pa_bd}, 有
	\begin{equation}\label{eq:int_DL2_W1pa_bd}
		\norm{u(t+1)}_2^2
		+ \int_t^{t+1}\int_{\Omega}a \abs{\nabla u}^p
		\leq C + \norm{u(t)}_2^2,
	\end{equation}
	联系 \cref{eq:int_DL2_L2_bd,eq:int_DL2_W1pa_bd}, 有
	\begin{equation}\label{eq:int_W1pa_L2}
		\int_t^{t+1}\int_{\Omega}a \abs{\nabla u}^p
		+ \norm{u}_2^2 \leq C \text{ for } t \geq T_0.
	\end{equation}
	应用 \cref{eq:utL2_DW1pa_DL2,eq:int_W1pa_L2}, 由一致 Gr\"onwall 引理, 我们得到 $W_0^{1,p}(a,\Omega)$ 上的有界吸收集,
	即 $\exists \rho$ 和 $T \geq T_{0}$ 使得
	\begin{equation}\label{eq:uL2_W1pa_bd}
		\norm{u(t)}_2^2 + \int_{\Omega}a\abs{\nabla u}^p \leq \rho \text{ for } t \geq T,
	\end{equation}
	以上完成了整个证明.
\end{proof}
因为 $p \geq 4$, 由紧嵌入定理
$W_0^{1,p}(\Omega) \subset\subset L^2(\Omega)$
以及 \cref{thm:absorb} 立即得到 $L^2(\Omega)$ 中全局吸引子的存在性.
\begin{theorem}\label{thm:attractor_L2}
	由~\cref{eq:main}的弱解生成的半群 $\Brace{S(t)}_{t \geq 0}$ 在
	$L^2(\Omega)$ 存在全局吸引子 $\mathcal{A}_2$.
\end{theorem}
接下来证明 $W_0^{1,p}(\Omega)$ 中全局吸引子的存在性. 首先, 我们需要得到 $u_t$ 在 $L^2(\Omega)$ 中的有界性.
然而, 当 $\abs{\nabla u}$ 足够小时, 很难证明 $u_t$ 的一致有界性, 因此我们考虑 $\abs{\nabla u}+1$.
\begin{theorem}\label{thm:ut_L2_bd}
	对任意有界子集 $B \subset L^2(\Omega)$,
	存在常数 $T' = T'(B) > 0$, 使得
	\begin{equation}
		\norm{u_t(s)}_2^2 \leq M, \quad \forall u_0 \in B \text{ and } s \geq T'.
	\end{equation}
\end{theorem}
\begin{proof}
	% \begin{proofpart}
	% 	First we consider the situation of $\abs{\nabla u} \to 0$ as $t \to t_0$,
	% 	where $t_0 \in \R^{+}\cup\Brace{+\infty}$, without loss of generality,
	% 	we assume $\abs{\nabla u} \leq 1$.
	% 	If we multiply \cref{eq:main} by any fixed $\phi \in C_c^{\infty}(\Omega)$
	% 	and integrate over $\Omega$, then take limits, we have,
	% 	\begin{equation}
	% 		\begin{split}
	% 			\lim_{t \to t_0}\int_{\Omega}u_t \phi
	% 			= - \lim_{t \to t_0}\int_{\Omega}a\abs{\nabla u}^{p-2} \nabla u
	% 			\cdot \nabla \phi
	% 			- \lim_{t \to t_0}\int_{\Omega}b\abs{\nabla u}^2 \phi
	% 			= 0
	% 		\end{split}
	% 	\end{equation}
	% \end{proofpart}
	求 \cref{eq:main} 关于 $t$ 的导数, 记 $v = u_t$, 有
	\begin{equation}
		\begin{split}
			v_t
			= \Div\left(a\abs{\nabla u}^{p-2}\nabla v\right)
			+ \Div\left(\left(p-2\right)a \abs{\nabla u}^{p-4}\left(\nabla u \cdot \nabla v\right)\nabla u\right)
			- 2b\nabla u \cdot \nabla v.
		\end{split}
	\end{equation}
	将上述方程与 $v$ 相乘并在 $\Omega$ 上积分, 得到
	\begin{equation}
		\begin{split}
			\frac{1}{2}\frac{d}{dt}\norm{v}_2^2
			+ \int_{\Omega}a\abs{\nabla u}^{p-2}\abs{\nabla v}^2
			&+ \int_{\Omega}\left(p-2\right)a\abs{\nabla u}^{p-4}\left(\nabla u
			\cdot \nabla v\right)^2\\
			&+ 2\int_{\Omega}b\nabla u \cdot \nabla v v
			= 0.
		\end{split}
	\end{equation}
	下面估计阻尼项,
	\begin{equation}
		\begin{split}
			\int_{\Omega}b\nabla u \cdot \nabla v v
			&\leq \left(\int_{\Omega}b^2\left(\nabla u
			\cdot \nabla v\right)^2\right)^{\frac{1}{2}}
			\*\norm{v}_2\\
			&\leq \norm{b^2 a^{-1}}_{\infty}^{\frac{1}{2}}
			\left(\int_{\Omega}a\abs{\nabla u}^2\abs{\nabla v}^2\right)^{\frac{1}{2}}\norm{v}_2\\
			&\leq \norm{b^2 a^{-1}}_{\infty}^{\frac{1}{2}}
			\left(\int_{\Omega}a\left(\abs{\nabla u}+1\right)^2\abs{\nabla v}^2\right)^{\frac{1}{2}}\norm{v}_2\\
			% &= \norm{b^2 a^{-1}}_{\infty}^{\frac{1}{2}}
			% \left(
			% \int_{\Omega}a\left(
			% \left(\abs{\nabla u}+1\right)^2 + 1 - 2\left(\abs{\nabla u}+1\right)
			% \right)\abs{\nabla v}^2
			% \right)^{\frac{1}{2}}\norm{v}_2\\
			&\leq \norm{b^2 a^{-1}}_{\infty}^{\frac{1}{2}}
			\left(\int_{\Omega}a\left(\abs{\nabla u}+1\right)^{p-2}\abs{\nabla v}^2\right)^{\frac{1}{2}}\norm{v}_2\\
			&\leq \epsilon \int_{\Omega}a\left(\abs{\nabla u}+1\right)^{p-2}\abs{\nabla v}^2
			+ C(\epsilon) \norm{v}_2^2,
		\end{split}
	\end{equation}
	对于第二项
	\begin{equation}
		\begin{split}
			\int_{\Omega}a\abs{\nabla u}^{p-2}\abs{\nabla v}^2
			\geq 2^{3-p}\int_{\Omega}a\left(\abs{\nabla u}+1\right)^{p-2}\abs{\nabla v}^2
			- \int_{\Omega}a\abs{\nabla v}^2.
		\end{split}
	\end{equation}
	因此我们有
	\begin{equation}\label{eq:DvL2_bd_vL2}
		\begin{split}
			& \frac{1}{2}\frac{d}{dt}\norm{v}_2^2
			+ C\int_{\Omega}a\left(\abs{\nabla u}+1\right)^{p-2}\abs{\nabla v}^2\\
			+{} & \int_{\Omega}\left(p-2\right)a\abs{\nabla u}^{p-4}\left(\nabla u
			\cdot \nabla v\right)^2
			\leq C\norm{v}_2^2 + C.
		\end{split}
	\end{equation}
	现在应用 \cref{eq:utL2_DW1pa_bd_W1pa,eq:uL2_W1pa_bd,eq:DvL2_bd_vL2},
	得到
	\begin{equation}
		\begin{split}
			\frac{d}{dt}\left(
			\norm{v}_2^2 + C\int_{\Omega}a\abs{\nabla u}^p
			\right)
			\leq C,
		\end{split}
	\end{equation}
	接下来积分 \cref{eq:utL2_DW1pa_DL2} 从 $t$ 到 $t+1$, 有
	\begin{equation}
		\begin{split}
			\int_{\Omega}a\abs{\nabla u(t+1)}^p
			+ \norm{u(t+1)}_2^2
			+ C\int_t^{t+1}\norm{v}_2^2
			\leq C + \int_{\Omega}a\abs{\nabla u(t)}^p
			+ \norm{u(t)}_2^2,
		\end{split}
	\end{equation}
	因此当 $t$ 足够大, 有
	\begin{equation}
		\begin{split}
			\int_t^{t+1}\left(
			\norm{v}_2^2 + C\int_{\Omega}a\abs{\nabla u}^p
			\right) \leq C,
		\end{split}
	\end{equation}
	由一致 Gr\"onwall 引理得, 存在 $M > 0$ 使得
	\begin{equation}
		\begin{split}
			\norm{v(t)}_2^2
			+ \int_{\Omega}a\abs{\nabla u}^p \leq M \text{ for } t \geq T' \geq T,
		\end{split}
	\end{equation}
	以上完成了证明.
\end{proof}
现在证明
$W_0^{1,p}(\Omega)$ 中全局吸引子的存在性.
\begin{theorem}
	由~\cref{eq:main}得到的半群 $\Brace{S(t)}_{t \geq 0}$ 在 $W_0^{1,p}(\Omega)$
	存在全局吸引子 $\mathcal{A}_V$, 即
	$\mathcal{A}_V$ 在 $W_0^{1,p}(\Omega)$ 紧, 不变并且在 $W_0^{1,p}(\Omega)$
	拓扑下吸引 $L^2(\Omega)$ 的任意有界子集.
\end{theorem}
\begin{proof}
	只需证明 $\Brace{S(t)}_{t \geq 0}$ 在 $W_0^{1,p}(\Omega)$ 里是渐进紧的.
	由~\cref{thm:absorb}, 令 $B_0$ 是在 $W_0^{1,p}(\Omega)$ 的有界吸收集,
	接下来验证对于任意序列
	$\Brace{u_{0n}}_{n=1}^{\infty} \subset B_0$, $\Brace{u_n(t_n)}_{n=1}^{\infty}$
	在 $W_0^{1,p}(\Omega)$ 存在收敛子列.

	事实上由 \cref{thm:attractor_L2}, 我们有
	$\Brace{u_n(t_n)}_{n=1}^{\infty}$ 在 $L^2(\Omega)$ 里是预紧的.
	从 $\Brace{u_n(t_n)}_{n=1}^{\infty}$ 取子列在 $L^2(\Omega)$ 上是 Cauchy 列,
	同样记为 $\Brace{u_n(t_n)}_{n=1}^{\infty}$. 由 $p \geq 4$ 以及
	H\"older 不等式, 有
	\begin{equation*}
		\begin{split}
			& \int_{\Omega}a\abs{\nabla u_m(t_m) - \nabla u_n(t_n)}^p\\
			\leq{} & C\int_{\Omega}a
			\left(\abs{\nabla u_m(t_m)}^{p-2}\nabla u_m(t_m)
			- \abs{\nabla u_n(t_n)}^{p-2}\nabla u_n(t_n)\right)
			\*\nabla\left(u_m(t_m) - u_n(t_n)\right)\\
			={} & C\left(\int_{\Omega}\left(u_n-u_m\right)_t\left(u_m-u_n\right)
			- \int_{\Omega}b\left(\abs{\nabla u_m}^2 - \abs{\nabla u_n}^2\right)
			\left(u_m - u_n\right)\right)\\
			\leq{} & C\norm{\left(u_n-u_m\right)_t}_2\norm{u_m-u_n}_2\\
			+{} & C\left(\int_{\Omega}a\left(\abs{\nabla u_m}^p
			+ \abs{\nabla u_n}^p\right)\right)^{\frac{2}{p}}
			\left(b^{\frac{2p}{p-4}}a^{-\frac{4}{p-4}}\right)^{\frac{p-4}{2p}}
			\norm{u_m-u_n}_2 \to 0
		\end{split}
	\end{equation*}
	当 $m$, $n \to \infty$. 应用 \cref{thm:absorb}, 半群 $\Brace{S(t)}_{t \geq 0}$
	存在一个全局吸引子 $\mathcal{A}_V$ 在 $W_0^{1,p}(\Omega)$.
\end{proof}

%附录部分
\appendix

%自动生成参考文献列表, 需要模板根目录下有:bib/thesis.bib
\bibliographystyle{amsplain}
\bibliography{bib/thesis}
% \printbibliography

%致谢
\begin{thanks}
	首先, 衷心的感谢我的导师马闪副教授! 感谢马老师一直以来对我学习的鼓励和支持!
	感谢她在我读硕士这三年对我孜孜不倦的教诲和指导.
	在毕业论文撰写期间, 马老师总是极具耐心地指导我, 并一丝不苟地帮我修改,
	从论文课题的选择到论文写作定稿的过程, 老师都倾注了大量的心血和时间.

	其次, 感谢我的同学、同门、以及师兄弟姐妹们三年里对我的宽容以及对我的帮助和建议.
	感谢 LZUthesis 模版的作者帮我节省了很多排版的精力

	最后感谢答辩委员会和参加论文评阅的专家和老师们为本论文给予的宝贵建议与
	悉心指导.

\end{thanks}

\end{document}
