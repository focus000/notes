% !TEX TS-program = xelatex
% !BIB program = bibtex
% !TEX TS-program = xelatex
% !TEX TS-program = xelatex
% !Mode:: "TeX:UTF-8"   %%winedt 以utf8编码打开

%双行标题
\documentclass[oneside,longtitle]{LZUthesis}
%单行标题
%\documentclass[twoside]{LZUthesis}

\usepackage{float}
\usepackage{fancybox}
\usepackage{calc}
\usepackage{mathdots}
\usepackage{graphicx}
\usepackage{listings}
\usepackage{enumitem}

%user settings begin
\usepackage{amssymb}
\usepackage{mathtools}
\usepackage{cleveref}
\crefname{equation}{}{}
\crefname{figure}{图}{图}
\crefname{table}{表}{表}
\crefname{appendix}{附录}{附录}
\crefname{chapter}{章}{章}
\crefname{theorem}{定理}{定理}
\crefname{lemma}{引理}{引理}
\newcommand{\crefpairconjunction}{~和~}
\newcommand{\crefmiddleconjunction}{、}
\newcommand{\creflastconjunction}{~和~}
\newcommand{\crefpairgroupconjunction}{~和~}
\newcommand{\crefmiddlegroupconjunction}{、}
\newcommand{\creflastgroupconjunction}{~和~}
\newcommand{\crefrangeconjunction}{~}
% \usepackage{hyperref}
% \bibliographystyle{plain}
% \usepackage[backend=bibtex, style=lzubib]{biblatex}
% \addbibresource{bib/Remote.bib}
% \addbibresource{bib/DynamicalSystem.bib}
% \addbibresource{bib/background.bib}
% \addbibresource{bib/pre.bib}
% \addbibresource{bib/intro.bib}

% \newtheorem{theorem}{定理}[chapter]
% \newtheorem{corollary}[theorem]{推论}
% \newtheorem{lemma}[theorem]{引理}
% \newtheorem{proposition}[theorem]{命题}
% \newtheorem{remark}[theorem]{注记}
% \theoremstyle{definition}
% \newtheorem{definition}[theorem]{定义}
\newtheorem{proofpart}{part}
\makeatletter
\@addtoreset{proofpart}{theorem}
\makeatother

\numberwithin{equation}{chapter}

% \providecommand*{\lemmaautorefname}{Lemma}
% \providecommand*{\corollaryautorefname}{Corollary}
% \providecommand*{\propositionautorefname}{Proposition}
% \providecommand*{\remarkautorefname}{Remark}
% \providecommand*{\definitionautorefname}{Definition}

\newcommand*\abs[1]{\lvert#1\rvert}
\newcommand*\norm[1]{\lVert#1\rVert}
\newcommand*\Brace[1]{\lbrace#1\rbrace}
% \newcommand*\innerproduct[1]{\langle#1\rangle}

\newcommand\R{\mathbb{R}}
\newcommand*\Laplace{\mathop{}\!\mathbin\bigtriangleup}

\DeclareMathOperator{\Div}{div}
\DeclareMathOperator{\dist}{dist}
%user settings end

%声明图片后缀名
\DeclareGraphicsExtensions{.pdf,.eps}

\makeatletter

% 设置图形文件的搜索路径
\graphicspath{{figures/}}

% 小节标题靠左对齐
\ctexset{section={format+={\flushleft}}}
% \CTEXsetup[format+={\flushleft}]{section}

% 取消链接的颜色(黑白打印时启用)
%\hypersetup{colorlinks=false}

%使文档居中, 打印时应注释掉
% \evensidemargin 0.93 true cm
% \setlength{\hoffset}{-0.3 cm}

%允许equationarray分页换行
\allowdisplaybreaks

%使字体清晰, 且透明位图不会使页面文字粗细不一
\usepackage{eso-pic}
\AddToShipoutPicture{%
\special{pdf: put @thispage <</Group << /S /Transparency /I true /CS /DeviceRGB>> >>}%
}

%页面背景色
%\definecolor{yellow}{rgb}{0.99,0.99,0.70}
%\pagecolor{yellow}

%浮动项超链接正确跳转
\usepackage[all]{hypcap}

\makeatother


\begin{document}

%分类号;一般不要求学生填写
\classification{0193}
\serialnumber{10730}

%密级;申请保密后需填写
\confidential{}

%中文标题
\title{带有阻尼项的加权$p$-Laplace方程}

%中文标题第二行, 题目较短时删除之, 并去除文档类选项 longtitle
\titleadd{全局吸引子的存在性}

%英文标题
\englishtitle{Global attractors of weighted $p$-Laplace}

%英文题目第二行, 题目较短时删除之, 并去除文档类选项 longtitle
\englishtitleadd{equation with a damping term}

%作者汉语姓名
\author{李蕴方}

%专业
\major{数学•应用数学}

%学位
\degree{硕士}

%研究方向
\direction{无穷维动力系统}

%导师姓名+职称
\advisor{马闪~副教授}

%论文工作起止年月
\datebeginAndEnd{2019年9月至2020年5月}

%论文提交日期
\submitdate{}

%答辩日期
\defenddate{}

%学位授予日期
\degreedate{}

%左侧页眉
\lzuthesis{兰州大学硕士学位论文}

%生成封面
\maketitle

%生成声明页
\makestatement

%前文-罗马页码
\frontmatter\pagenumbering{Roman}

%中文摘要
\begin{abstract}
	本文主要研究了以下带有阻尼项的加权$p$-Laplace方程解的存在唯一性以及解的长时间行为.
	\begin{equation*}
		\begin{cases}
			u_t = \Div(a(x)\abs{\nabla u}^{p-2}\nabla u) - b(x)\abs{\nabla u}^2 \quad &(x, t) \in \Omega \times \R^+,\\
			u(x,0) = u_0 \quad &x \in \Omega,\\
			u(x, t) = 0 \quad &x \in \partial\Omega,
		\end{cases}
	\end{equation*}
	其中 $\Omega$ 是在 $\R^{n}$ 里的有界光滑开区域, 边界记为 $\partial\Omega$, $p>1$.
	$ a(x)$, $b(x) \in C^1(\bar{\Omega}) $,
	% $b(x) \geq 0$,
	在 $\Omega$ 内部 $a(x) > 0$, 在边界 $\partial\Omega$ 上 $a(x) = 0$.
	而且, $a(x), b(x)$ 满足可积性条件:
	\begin{equation*}
		\int_{\Omega} b^{\frac{2p}{p-4}}a^{-\frac{4}{p-4}} < \infty,
	\end{equation*}
	当 $u_0 \in L^2(\Omega) $ 时, 证明了方程存在唯一弱解满足
	\begin{equation*}
		u \in L^p(0, T; W_0^{1,p}(a,\Omega)), \quad u \in C([0, T]; L^2(\Omega)).
	\end{equation*}

	进一步, 还研究了带有阻尼项的加权$p$-Laplace方程全局吸引子的存在性. 通过渐近先验估计方法分别得到了 $L^2(\Omega)$ 和 $W_0^{1,p}(\Omega)$ 上全局吸引子的存在性.
	最后讨论了扰动加权项时全局吸引子的上半连续性.
\end{abstract}

%中文关键词
\keywords{全局吸引子, 弱解, 加权$p$-Laplace方程, 上半连续性}

%英文摘要
\begin{englishabstract}
	This article is devoted to study the asymptotic behavior for a dynamical system generated by a nonlinear
	weighted $p$-Laplace equation that reads
	\begin{equation*}
		\begin{cases}
			u_t = \Div(a(x)\abs{\nabla u}^{p-2}\nabla u) - b(x)\abs{\nabla u}^2 \quad &(x, t) \in \Omega \times \R^+,\\
			u(x,0) = u_0 \quad &x \in \Omega,\\
			u(x, t) = 0 \quad &x \in \partial\Omega,
		\end{cases}
	\end{equation*}
	where $\Omega$ is a bounded open domain in $\R^{n}$ with a sufficiently smooth boundary $\partial\Omega$, $p>1$,
	$ a(x)$, $b(x) \in C^1(\bar{\Omega}) $, 
	% $b(x) \geq 0$, 
	and $a(x) > 0$ in $\Omega$, $a(x) = 0$ on $\partial\Omega$ which satisfy
	\begin{equation*}
		\int_{\Omega} b^{\frac{2p}{p-4}}a^{-\frac{4}{p-4}} < \infty
	\end{equation*}
	as $u_0 \in L^2(\Omega) $, then it has a unique weak solution satisfies
	\begin{equation*}
	u \in L^p(0, T; W_0^{1,p}(a,\Omega)), \quad u \in C([0, T]; L^2(\Omega)).
	\end{equation*}

	Next, using a time dependent prior estimate, the existence of global attractor in $L^2(\Omega)$ and $W_0^{1,p}(\Omega)$ is given.
	Finally, upper semi-continuity of global attractors is considered.
\end{englishabstract}

%英文关键词
\englishkeywords{Global attractor, weak solution, weighted $p$-Laplacian, upper semi-continuity}

%自动生成章节目录以及pdf书签
\tableofcontents{}


%正文部分-数字页码
\mainmatter

%正文页眉页脚样式
\pagestyle{lzu}


\chapter{绪论}
\section{研究背景}
动力系统是一门用于描述给定状态空间随时间变化情况的数学分支.
通常认为 H. Poincare 是动力系统之父.
到 19 世纪末, 他首次指出, 如果所研究的系统涉及非线性相互作用,
即使是非常简单的系统, 其长时间行为也可能是不可预测的.
自 Poincare 以来,
许多数学家和物理学家在发展当今的动力系统概念上做出了重要贡献,
如 Lyapunov 引入了 Lyapunov 泛函和 Lyapunov 指数的概念,
发展了动力系统稳定性理论, Birkhoff 则将泛函分析的方法应用到
常微分动力系统理论研究中, 并对于拓扑动力系统建立了大范围的理论框架.
多年以后, A.A. Markov 总结了 Birkhoff 的理论, 正式提出了动力系统的抽象概念.
动力系统的研究是理解自然科学里许多重要问题的核心,
其在分子动力学, 等离子物理学和激光, 非线性光学, 燃烧现象, 数学经济学, 机器人控制等领域具有广泛等应用,
而其中最著名的两类问题分别是天体力学和流体力学,
由于用于描述天体运行系统的相空间是有限维的, 而描述流体运动相空间是无限维的,
我们说前者是有限维动力系统, 用常微分方程描述, 后者是无穷维动力系统, 用偏微分方程描述.
天体力学方面, 动力系统出现了著名的 Kolmogorov-Arnold-Moser 理论.
而在流体力学方面, 则有 S.Smale 提出的吸引子的概念以及 D.Ruelle 和 F.Takens 对湍流的解释,
即耗散系统的混沌行为可以解释成当时间
$t \to \infty$ 时存在一个结构复杂的吸引子, 例如其内部结构可以是 Cantor 集的笛卡尔积,
而这种复杂结构是我们观察到混沌现象的原因之一, 因此无穷维动力系统的研究具有重要意义.

事实上, 吸引子是无穷维动力系统研究的核心问题之一.
一般而言, 从相空间的任意有界集出发的轨道最终都趋向于吸引子,
这样的集合包含了系统所有可能的极限状态.
所以研究吸引子的几何结构可以揭示对应系统的重要信息,
因此证明吸引子的存在性是无穷维动力系统的一个基本且重要的问题之一.

为了描述系统状态对参数对敏感性, 通常用吸引子关于参数的连续性来描述.
而吸引子的连续性又可进一步分为上半连续性和下半连续性, 上半连续性主要刻画了吸引子在参数扰动下不会爆破,
而下半连续性主要刻画了吸引子在参数扰动下不会消失(见\citep{robinsonInfiniteDimensionalDynamicalSystems2001a}).

本文讨论了带有加权 $p$-Laplace 算子的发展方程所生成的无穷维动力系统,
这类系统具有丰富的物理背景.
事实上, 当 $1 < p < 4/3$ 时, $p$-Laplace方程可以用来描述冰川运动,
当 $p = 3/2$ 时, $p$-Laplace方程可以描述多孔介质渗流现象,
当 $p = 2$ 时, $p$-Laplace方程可以描述牛顿流体反应扩散现象,
当 $p > 2$ 时, $p$-Laplace方程可以描述非线性耗散, 低功耗材料, 非牛顿流体运动等.
而对于带有权函数的 $p$-Laplace 算子更能刻画材质的非均匀分布导致局部具有不同的物理规律的现象(见\citep{lionsMathematicalTopicsFluid1996}),
更符合现实情况. 同时这类方程在图像识别, 图像恢复, 图像去噪等方面也有广泛的应用.
因此 $p$-Laplace 类型的方程具有很好的应用背景.

带权 $p$-Laplace 在数学上获得了广泛的关注, 许多学者在此类方程上做了很多工作.
在\citep{cortazarExistenceSignChanging2014}中,
作者使用了打靶法找到了方程 $\Div(a\abs{\nabla u}^{p-2}\nabla u) + bf(u) = 0$ 的径向解.
在\citep{musinaExistenceMultiplicityResults2009}中,
作者考虑了一类临界增长, 带有 Hardy 位势的加权$p$-Laplace椭圆方程非负解的存在性.
在\citep{cavalheiroWeightedSobolevSpaces2008}中,
作者证明了方程$-\Div(\omega(x)\mathcal{A}(x, u, \nabla u)) = f(x)$熵解的存在性.
在\citep{caldiroliVariationalDegenerateElliptic2000}中,
作者用变分法证明了方程 $-\Div(a\nabla) = \lambda u + g(x, u)$ 正解的存在性.
在\citep{leBoundaryValueProblems1998}中,
作者考虑了一类退化半线性椭圆方程的边值问题.
在\citep{yinEvolutionaryWeightedPLaplacian2007}中,
作者考虑了一类边值退化的加权 $p$-Laplace 发展方程解的适定性.
文献\citep{gazziniSobolevtypeInequalityRelated2009}推广了一类用于估计带权 $p$-Laplace 方程所对应能量泛函的 Sobolev 不等式,
文献\citep{liLongtimeBehaviorClass2014b,maGlobalAttractorsWeighted2012a,galClassDegenerateParabolic2012}证明了一类带权 $p$-Laplace 方程全局吸引子的存在性或吸引子的维数.
文献\citep{monticelliMaximumPrinciplesWeak2009}证明了一类带权 $p$-Laplace 方程的弱极值原理.
文献\citep{dengInfinitelyManySolutions2019}证明了一类$p$-Laplace椭圆方程具有无穷多个解.
文献\citep{chenMultipleSolutionsFractional2020}讨论了带有凹凸非线性项的分数阶$p$-Laplace方程的多解问题.
文献\citep{oanhMultiplicitySolutionsNonlocal2020}考虑了非局部$p$-Laplace方程的多解问题.

本文主要考虑了如下偏微分方程解和吸引子的存在性问题,
\begin{equation*}
	u_t = \Div(a(x)\abs{\nabla u}^{p-2}\nabla u) - b(x)\abs{\nabla u}^2,
\end{equation*}
下面我们将分别回顾吸引子的存在性和带有 $p$-Laplace 算子的发展方程的研究概况.

\section{无穷维动力系统全局吸引子存在性研究概况}
考虑一般的自治系统
\begin{equation}\label{eq:autonomous}
	\begin{cases}
		u_t = Au,\\
		u(x, 0) = u_0(x).
	\end{cases}
\end{equation}
若存在 Banach 空间 $X$ 以及时间 $T$, 使得方程\eqref{eq:autonomous}的解 $u \in C([0, T], X)$ 是存在唯一的,
可定义方程\eqref{eq:autonomous}解的连续半群 $S(t) \colon X \to X$ 满足:
\begin{enumerate}[itemindent = -1em]
	\item $S(0) = I,$
	\item $S(t + s) = S(t)S(s), \quad \forall t, s \geq 0,$
	\item $\norm{S(t)u_0 - u_0} \to 0, \quad\forall u_0 \in X$ 当 $t \to 0$.
\end{enumerate}
解半群是为了描述吸引子的基本概念, 
动力系统有能量保守系统和能量耗散系统之分. 本文主要就能量耗散系统
的动力学渐近行为进行研究. 在这里, 系统的能量耗散性由吸收集的存在性来刻画,
为了简化计算, 我们用紧的吸引集, 即全局吸引子刻画系统的长时间行为, 对于吸引集 $K$,
即 $\exists K \subset X$
对 $\forall B \subset X, \epsilon > 0$, $\exists T(B, \epsilon)$,
当 $t > T$ 时有 $S(t)B$ 在 $K$ 的 $\epsilon$-邻域内,
其中 $B, K$ 为有界集, 也可等价的定义为
\begin{equation*}
	\lim_{t \to \infty} \dist_H(S(t)B, K) = 0,
\end{equation*}
其中 $\dist_H(A, B) \coloneqq \sup_{x \in A}\inf_{y \in B}d(x, y)$ 为 Hausdorff 半距离.
所谓半群 $S(t)$ 的全局吸引子即满足以下条件的 $\mathcal{A} \subset X$:
\begin{enumerate}[itemindent = -1em]
	\item $\mathcal{A}$ 是紧集,
	\item $\mathcal{A}$ 是严格不变的: $S(t)\mathcal{A} = \mathcal{A}, \quad\forall t \geq 0$,
	\item $\mathcal{A}$ 是半群 $S(t)$ 的吸引集.
\end{enumerate}
我们了解到, 吸引子本身包含所有解轨道的极限状态,
故全局吸引子可以反应系统的长时间行为, 吸引子的存在性也是研究系统的长时间行为的基础.
关于吸引子存在性的证明有一个著名的存在性定理:
\begin{theorem}[\citep{efendievAttractorsDegenerateParabolic2013b}]\label{thm:attractorexist}
	若连续半群 $S(t): X \to X$ 具有紧的吸引集, 则存在全局吸引子 $\mathcal{A}$.
\end{theorem}
根据\cref{thm:attractorexist}的描述, 一般证明吸引子的存在性需要验证三个条件:
连续性, 耗散性和紧性. 在方程解是唯一的时候, 半群的连续性不难得到,
耗散性一般通过能量估计得到, 而对于紧性, 一般验证下列条件之一.
\begin{enumerate}[itemindent = -1em]
	\item 紧性 (\citep{temamInfiniteDimensionalDynamicalSystems1997}): 对 $\forall t \geq 0$ 以及 $X$ 中任意有界集 $B$ 有, $S(t)B$ 是相对紧的;
	\item 一致紧性 (\citep{babinAttractorsEvolutionEquations1992a,robinsonInfiniteDimensionalDynamicalSystems2001a}): 对任意有界集 $B \subset X$, $\exists t(B) > 0$ 使得,
	\begin{equation*}
		\bigcup_{t \geq t(B)} S(t)B \text{ 在 } X \text{ 中相对紧};
	\end{equation*}
	\item 渐近紧性 (\citep{ladyzhenskayaAttractorsSemigroupsEvolution1991}): 对任意有界列 $\Brace{u_k} \subset X$ 和
	$\forall \Brace{t_k}, \quad t_k \to \infty$, 有 $S(t_k)u_k$ 在 $X$ 里列紧.
\end{enumerate}

由于一般的 Banach 空间不是局部紧的, 相对于条件一, 更多的是去验证条件二.
一般方法是证明在空间 $X‘$ 存在一致有界吸收集, 且 $X'$ 可以紧嵌入到 $X$,
一般用 Sobolev 紧嵌入定理得到. 通常要求方程具有更高的正则性. 而对于条件三,
在缺乏紧嵌入定理的时候很有用, 一般验证方式是利用半群的分解.

\section{带有 $p$-Laplace 算子的发展方程研究概况}
对于抛物型 $p$-Laplace 方程, 由于 $p$-Laplace 算子 $\Div(\abs{\nabla u}^{p-2}\nabla u)$ 是非线性的, 我们无法用谱理论去处理它, 故对于很多 laplace 算子对应的结果无法轻易的推广过来.
对于抛物型 $p$-Laplace 方程通常用广义 Galerkin 法或者非退化逼近方程法证明解的适定性, 见\citep{babinAttractorsEvolutionEquations1992a},
对于\cref{eq:plaplaceprime}, 文献\citep{efendievAttractorsDegenerateParabolic2013b}给出了解的适定性以及吸引子的存在性以及吸引子的维数估计.
值得一提的是, DiBenedetto 在\citep{dibenedettoDegenerateParabolicEquations1993a}里系统的论述了一般的退化抛物方程理论,
并用 De Giorgi 迭代技巧证明了 $L^\infty$ 估计, 这在用非退化逼近方程法证明解的适定性很有用.
对于方程
\begin{equation}\label{eq:plaplaceprime}
	u_t = \Div(\abs{\nabla u}^{p - 2}\nabla u) - f(u) + g,
\end{equation}
文献\citep{carvalhoGlobalAttractorsProblems1999}考虑了方程$\frac{du}{dt}+A(u(t))+B(u(t))=0$全局吸引子的存在性, 其中$A$是极大单调算子.
作为应用, 文献\citep{cholewaGlobalAttractorsAbstract2000a}则考虑了$p$-Laplace发展方程解的适定性以及全局吸引子的存在性.
文献\citep{yangGlobalAttractorsPLaplacian2007} 通过验证渐近紧性证明了$p$-Laplace发展方程$W_0^{1, p}$上全局吸引子的存在性.
文献\citep{simsenLaplacianDifferentialInclusions2009} 考虑了一类$p$-Laplace微分包含耦合系统全局解的存在性, 以及全局吸引子的存在性和上半连续性.
文献\citep{liuAsymptoticRegularityPLaplacian2010}证明了当 $f$ 满足多项式增长条件时方程解的适定性和吸引子的存在性以及吸引子的维数.
文献\citep{zhongZ2IndexGlobal2010}考虑了非线性项为 $\abs{u}^{r-2}u - \lambda\abs{u}^{s-2}u + f(x, u)$ 的 $p$-Laplace 方程的吸引子的存在性并用$Z_2$指标定理证明了吸引子是无限维的.
文献\citep{simsenLaplacianParabolicProblems2014a}考虑了$p_s(x)$-Laplace发展方程全局吸引子的存在性以及关于$s$的上半连续性.
文献\citep{guoSingularPhenomenaSolutions2015}考虑了带有 $p(x)$-laplace 算子, 非线性项为 $\abs{u}^{r-2}u$, 其中 $r > 1$ 时解的奇异性.
文献\citep{gaoExistenceUniquenessNonexistence2016}则考虑了带有 $p(x, t)$-laplace 算子, 非线性项为 $\Div(b(x, t)\nabla u) + f(u)$ 时局部弱解的存在性以及全局弱解的不存在性.
文献\citep{liuExistenceUpperSemicontinuity2017b}讨论了非自治的$p$-Laplace方程拉回吸引子的上半连续性

对于以下加权$p$-Laplace方程
\begin{equation*}
	u_t = \Div(a(x)\abs{\nabla u}^{p-2}\nabla u) + f(u, \nabla u, x, t), \quad (x, t) \in Q_T = \Omega \times (0, T),
\end{equation*}
文献\citep{constantinGlobalSolutionsQuasilinear2002}讨论了更一般的加权项是 $a(t, x, u), p = 2$时全局吸引子的存在性.
文献\citep{karachaliosConvergenceAttractorsDegenerate2005}证明了一类带权 Ginzburg-Landau 方程全局吸引子的存在性.
文献\citep{anhGlobalExistenceLongtime2008}考虑了在任意边界下加权反应扩散方程解的适定性及全局吸引子的存在性.
文献\citep{anhLongtimeBehaviorQuasilinear2009}考虑了加权$p$-Laplace发展方程解的适定性即全局吸引子的存在性.
文献\citep{maGlobalAttractorsWeighted2012a}证明了初值在 $L^2(\Omega)$, $f(u, x, t) = g(x) - f(u)$, 其中 $f(u)$ 满足增长性条件时解的适定性及全局吸引子的存在性.
文献\citep{simsenExistenceUpperSemicontinuity2014}考虑了非自治的加权$p$-Laplace发展方程拉回吸引子的上半连续性.
文献\citep{zhanParabolicEquationRelated2016}证明了当
$a(x) = \rho^\alpha$, $f(u, x, t)$ 是 Lipschitz 函数, 初值在 $L^\infty(\Omega)$, 其中 $\rho(x) = dist(x, \partial \Omega), \alpha > 0, p > 1$ 时
弱解的存在唯一性.
文献\citep{Zhan2019Uniquenessa}证明了当 $f(u, x, t) = -b(x)\abs{\nabla u}^2$,
且 $p > 4$ 或者 $n > 2, \frac{4}{n+2} \leq p \leq \frac{4}{n-2}$,
非负初值 $u_0 \in L^\infty(\Omega), a(x)\nabla u_0 \in W_0^{1, p}(\Omega)$ 时存在唯一弱正解,
这里 $a(x) > 0$ 在 $\Omega$ 内部, $a(x) = 0$ 在边界 $\partial\Omega$ 上,
% $b(x) \geq 0$ 
且 $\int_{\Omega} b^{\frac{2p}{p-4}}a^{-\frac{4}{p-4}} \leq c$.
但对于吸引子还未有学者研究.

\section{本文研究的问题}
在\citep{Zhan2019Uniquenessa}中, 作者讨论了带有阻尼项的加权$p$-Laplace方程非负初值
$u_0 \in L^\infty(\Omega), a(x)\nabla u_0 \in W_0^{1, p}(\Omega)$ 时正解的适定性,
本文为了考虑方程初值更一般时方程所生成的无穷维动力系统全局吸引子的存在性, 首先讨论了初值 $u_0 \in L^2(\Omega)$ 时解的适定性问题,
继而讨论了在 $L^2(\Omega)$ 和 $W_0^{1, p}(\Omega)$ 上全局吸引子的存在性, 最后考虑了扰动权函数和初值时全局吸引子的上半连续性.
本文主要遇到的难点在于用渐近紧性验证$W_0^{1, p}(\Omega)$ 上全局吸引子的存在性,
为了得到渐近紧性, 我们用算子分解的技巧得到了 $u_t(s)$ 的 $L^2$ 估计.

在第二章中, 首先, 给出了带权 Lebesgue-Sobolev 空间的基本概念, 方程弱解的定义,
以及其他讨论方程解的适定性所要用到的引理.
之后, 讨论了方程解的适定性的主要结果. 为接下来讨论全局吸引子做准备.
在第三章, 我们给出全局吸引子在
$L^2(\Omega)$ 和 $W_0^{1,p}(\Omega)$ 上的存在性证明.
在第四章, 讨论了全局吸引子的上半连续性.

\chapter{带有阻尼项的加权$p$-Laplace方程弱解的适定性}\label{ch:Existence_and_uniqueness_of_the_weak_solution}
本章主要讨论如下带有阻尼项的加权$p$-Laplace方程弱解的适定性问题,
\begin{equation}\label{eq:main}
	\begin{cases}
		u_t = \Div(a(x)\abs{\nabla u}^{p-2}\nabla u) - b(x)\abs{\nabla u}^2 \quad &(x, t) \in \Omega \times \R^+,\\
		u(x,0) = u_0 \quad &x \in \Omega,\\
		u = 0 \quad &x \in \partial\Omega,
	\end{cases}
\end{equation}
其中, $\Omega$ 是 $\R^{n}$ 里的有界光滑开区域, 边界记为 $\partial\Omega$. $p>4$.
$ a(x)$, $b(x) \in C^1(\bar{\Omega}) $
% , $b(x) \geq 0$
. 在 $\Omega$ 内部 $a(x) > 0$, 在边界 $\partial\Omega$ 上 $a(x) = 0$.
\section{预备知识和假设}\label{ch:preliminaries}
通常, 通过 Sobolev 空间 $W^{k, p}$ 研究椭圆和抛物方程是自然的,
然而, 对于带权$p$-Laplace方程, 由于权函数带有奇异性, 在加权 Sobolev 空间中讨论是自然的.
因此, 我们首先引入研究方程\eqref{eq:main}合适的加权 Sobolev 空间.

设 $\Omega$ 是 $\R^n$ 里的有界区域, 权重
$a \colon \R^n \to [0, \infty)$
是一个非负的局部可和函数.
\begin{definition}
	带权 Lebesgue 空间 $L^p(a, \Omega)$, $1 \leq p < \infty$,
	是由局部可和函数类
	$u \colon \Omega \to \R$ 在如下范数下定义的 Banach 空间
	\begin{equation*}
		\norm{u}_{L^p(a,\Omega)} =
		\left( \int_{\Omega}a\abs{u}^p \right)^{\frac{1}{p}}.
	\end{equation*}
\end{definition}
\begin{definition}
	带权 Sobolev 空间 $W^{k,p}(a,\Omega)$,
	$1 \leq k < \infty$, $1 \leq p < \infty$,
	是由 $k$ 阶弱可微局部可和函数
	$u \colon \Omega \to \R$ 关于如下范数下的 Banach 空间
	\begin{equation*}
		\norm{u}_{W^{k,p}(a,\Omega)} =
		\left( \int_{\Omega}a\abs{u}^p \right)^{\frac{1}{p}}
		+ \sum_{\abs{\alpha}=k}
		\left( \int_{\Omega}a\abs{D^{\alpha}u}^p \right)^{\frac{1}{p}},
	\end{equation*}
	这里 $\alpha$ 是一个多重指标.

	另外,
	$W_0^{k,p}(a,\Omega)$ 被定义为
	$C_0^{\infty}(\Omega)$ 在如下范数下的完备空间
	\begin{equation*}
		\norm{u}_{L^p(a,\Omega)} =
		\left( \int_{\Omega}a\abs{D^{\alpha}u}^p \right)^{\frac{1}{p}}.
	\end{equation*}
\end{definition}
\begin{remark}
	一般 $W_0^{k,p}(a,\Omega)$ 在
	$W^{k,p}(a,\Omega)$ 不稠密, 除非 $a$ 满足 Muckenhoupt 条件. 具体细节见~\cite{goldshteinWeightedSobolevSpaces2009}.
\end{remark}
为了证明方程初值$u \in L^2(\Omega)$时解的存在性, 我们给出如下引理,
\begin{lemma}
	$\alpha$ 和 $\beta$ 是属于 $\R^{n}$ 的 $n$ 维向量, 如果 $p \geq 1$, 那么我们有
	\begin{equation}\label{eq:p_ineq}
		\abs{\alpha-\beta}^p \leq C\abs{\abs{\alpha}^{p-1}\alpha - \abs{\beta}^{p-1}\beta}.
	\end{equation}
\end{lemma}
\begin{proof}
	当 $n = 1$ 时, \cref{eq:p_ineq} 可写成如下形式,
	\begin{equation*}
		\abs{\alpha-\beta}^p \leq \abs{\alpha^p - \beta^p}.
	\end{equation*}

	首先, $\alpha \beta \geq 0$ 时, 
	不失一般性, 假设 $0 \leq \alpha \leq \beta$. 令
	\begin{equation*}
		f(x) = (\beta-\alpha+x)^p - x^p.
	\end{equation*}
	应知
	\begin{equation*}
		f'(x) = p\left((\beta-\alpha+x)^{p-1} - x^{p-1}\right) \geq 0,
	\end{equation*}
	于是可得
	\begin{equation*}
		\left(\beta-\alpha\right)^p = f(0) \leq f(\alpha) = \beta^p - \alpha^p.
	\end{equation*}
	其次, $\alpha \beta \leq 0$ 时,
	应用 Jensen 不等式我们得到
	\begin{equation*}
		\left(\abs{\alpha}+\abs{\beta}\right)^p \leq 2^{p-1}\left(\abs{\alpha}^p + \abs{\beta}^p\right),
	\end{equation*}
	以上我们证明了 $n = 1$ 时的\cref{eq:p_ineq}.
	
	当 $n > 1$ 时, 由余弦定理, 令 $\gamma$ 是 $\alpha$ 和 $\beta$ 夹角, 我们有
	\begin{equation*}
		\begin{split}
			\left(\abs{\alpha-\beta}^p\right)^2
			&= \left(\alpha^2 + \beta^2 - 2 \abs{\alpha}\abs{\beta}\cos{\gamma}\right)^p\\
			&\leq \left(\alpha^2+\beta^2\right)^p - 2^p\abs{\alpha}^p\abs{\beta}^p\cos^p{\gamma}\\
			&\leq 2^{p-1}\left(\alpha^{2p} + \beta^{2p} - 2\abs{\alpha}^p\abs{\beta}^p\cos^p{\gamma}\right)\\
			&\leq \left(2^{p-1}+C\right)\left(\alpha^{2p} + \beta^{2p}\right)
			- \left(2^{p-1}\cos^p{\gamma}+C\right)2\abs{\alpha}^p\abs{\beta}^p\\
			&\leq \left(2^{p-1}+C\right)\left(\alpha^{2p} + \beta^{2p} - 2\abs{\alpha}^p\abs{\beta}^p\cos{\gamma}\right)\\
			&= \left(2^{p-1}+C\right)\abs{\abs{\alpha}^{p-1}\alpha - \abs{\beta}^{p-1}\beta}^2.
		\end{split}
	\end{equation*}
	事实上当 $C$ 足够大时
	\begin{equation*}
		\begin{split}
			& 2^{p-1}\cos^p{\gamma} + C - \left(2^{p-1}+C\right)\cos{\gamma}\\
			={} & 2^{p-1}\left(\cos^p{\gamma} - \cos{\gamma}\right) + C\left(1-\cos{\gamma}\right)
			\geq 0.
		\end{split}
	\end{equation*}
	以上完整的证明了\cref{eq:p_ineq}.
\end{proof}
\begin{lemma}\label{lem:VecIneq}
	$\alpha$ 和 $\beta$ 是属于 $\R^{n}$ 的 $n$ 维向量, 如果 $p \geq 4$, 那么我们有
	\begin{equation*}
		\abs{\alpha^2 - \beta^2}^{\frac{p}{2}}
		\leq C \langle \abs{\alpha}^{p-2}\alpha - \abs{\beta}^{p-2}\beta, \alpha-\beta\rangle
	\end{equation*}
\end{lemma}
\begin{proof}
	易证下面的不等式成立,
	\begin{equation}\label{eq:VecIneq_0}
		\abs{\abs{\beta}^{p-1}\beta - \abs{\alpha}^{p-1}\alpha} \leq p\abs{\beta-\alpha}\int_0^1 \abs{\alpha + t(\beta - \alpha)}^{p-1}dt.
	\end{equation}
	在~\cite{lindqvistNotesStationaryPLaplace2019}中, 作者证明了当 $p \geq 2$ 时, 
	\begin{equation}\label{eq:VecIneq_1}
		\langle \abs{\alpha}^{p-2}\alpha - \abs{\beta}^{p-2}\beta, \alpha-\beta\rangle
		\geq \abs{\alpha-\beta}^2\int_0^1 \abs{\beta + t(\alpha - \beta)}^{p-2}dt.
	\end{equation}
	成立.
	因此当 $p \geq 4$ 时, 结合 \cref{eq:p_ineq,eq:VecIneq_0,eq:VecIneq_1} 有
	\begin{equation*}
		\begin{split}
			\abs{\alpha^2 - \beta^2}^{\frac{p}{2}}
			&\leq 2^{\frac{p}{2}}\abs{\alpha-\beta}^{\frac{p}{2}}
			\left(\int_0^1 \abs{\beta + t(\alpha - \beta)}dt\right)^{\frac{p}{2}}\\
			&\leq C\frac{p-2}{2}2^{\frac{p}{2}}\abs{\alpha-\beta}^2
			\left(\int_0^1 \abs{\beta + t(\alpha - \beta)}^{\frac{p-4}{2}}dt\right)\\
			&\times\left(\int_0^1 \abs{\beta + t(\alpha - \beta)}dt\right)^{\frac{p}{2}}\\
			&\leq C\frac{p-2}{2}2^{\frac{p}{2}}\abs{\alpha-\beta}^2
			\int_0^1 \abs{\beta + t(\alpha - \beta)}^{p-2}dt\\
			&\leq C \langle \abs{\alpha}^{p-2}\alpha - \abs{\beta}^{p-2}\beta, \alpha-\beta\rangle.
		\end{split}
	\end{equation*}
\end{proof}

\section{弱解的存在唯一性}
本节我们讨论方程\eqref{eq:main}弱解的存在唯一性. 首先给出弱解的定义
\begin{definition}
	函数 $u(x, t)$ 是方程\eqref{eq:main}在 $[0, T]$ 上的弱解, 如果
	\begin{equation*}
		u \in C([0, T]; L^2(\Omega))\cap L^p(0, T; W_0^{1,p}(a,\Omega))
	\end{equation*}
	且对于 $\forall \phi \in L^p(0, T; W_0^{1,p}(a,\Omega))\cap L^{\frac{p}{p-2}}(Q_T)$ 有
	\begin{equation*}
		\int_0^T <u_t, \phi> + \int_0^T\int_\Omega a(x)\abs{\nabla u}^{p - 2}\nabla u \cdot \nabla \phi
		+ \int_0^T\int_\Omega b(x)\abs{\nabla u}^2\phi = 0.
	\end{equation*}
	其中对于 $u(0, x) = u_0$ 在 $\Omega$ 上几乎处处成立.
\end{definition}

% 陈述进展
在\cite{Zhan2019Uniquenessa}中作者给出了方程~\eqref{eq:main}正解的适定性结果,
% 为了方便证明方程\eqref{eq:main}解的适定性,
% \cite{Zhan2019Uniquenessa}中的相关结论将在证明里用到, 在此列出,
% \begin{theorem}\cite[定理 1.3 和 1.6]{Zhan2019Uniquenessa}\label{thm:zhan}
% 	如果 $p>4$, $a(x)$, $b(x)$ 满足
% 	\begin{equation}\label{eq:zhan_intcondition}
% 		\int_{\Omega} b^{\frac{2p}{p-4}}a^{-\frac{4}{p-4}} \leq c,
% 	\end{equation}
% 	且 $u_0$ 满足
% 	\begin{equation}\label{eq:zhan_initdata}
% 		0 \leq u_0 \in L^{\infty}(\Omega), a(x)u_0 \in W_0^{1,p}(\Omega),
% 	\end{equation}
% 	那么方程\eqref{eq:main} 存在满足如下条件的唯一非负弱解
% 	\begin{equation*}
% 		u \in L^{\infty}(Q_T), a(x)\abs{\nabla u}^p \in L^1(Q_T).
% 	\end{equation*}
% 	且初值具有如下性质
% 	\begin{equation*}
% 		\lim_{t \to 0}\int_{\Omega}\abs{u(x,t) - u_0(x)}dx = 0.
% 	\end{equation*}
% \end{theorem}
% 相对于以上关于方程正解的适定性结果,
本文将会讨论初值 $u_0 \in L^2(\Omega)$ 时一般解的适定性问题
\begin{theorem}\label{thm:absorb}
	如果 $p>4$, $a(x)$, $b(x)$ 满足
	\begin{equation}\label{initial_data_condition_tmp}
		\int_{\Omega} b^{\frac{2p}{p-4}}a^{-\frac{4}{p-4}} < \infty,
	\end{equation}
	而且 $u_0 \in L^2(\Omega) $ 时, 方程\eqref{eq:main}存在唯一弱解且满足
	\begin{equation*}
		u \in L^p(0, T; W_0^{1,p}(a,\Omega)), \quad u \in C([0, T]; L^2(\Omega)).
	\end{equation*}
\end{theorem}

\begin{proof}
	首先, 我们证明当初值 $u_0 \in L^{\infty}(\Omega) \cap W_0^{1, p}(a, \Omega)$ 时解的存在性.
	% \begin{equation}\label{initial_data_condition_tmp}
	% 	u_0 \in L^{\infty}(\Omega) \cap W_0^{1, p}(a, \Omega).
	% \end{equation}

	取一列 $u_{\epsilon,0} \in C_0^\infty(\Omega) $, $u_{\epsilon,0} \to u_0 $ 在 $W_0^{1,p}(a, \Omega) $,
	且 $\norm{u_{\epsilon, 0}}_{\infty}$ 一致有界,
	由\citep{taylorPartialDifferentialEquations2011}可知, 方程
	% 花括号
	\begin{equation}\label{eq:approximated_maineq}
		\begin{cases}
			u_{\epsilon t}-\Div\left((a(x)+\epsilon)
			\left(\left|\nabla u_{\epsilon}\right|^{2}+\epsilon\right)^{\frac{p-2}{2}} \nabla u_{\epsilon}\right)
			+b(x)\left|\nabla u_{\epsilon}\right|^{2} = 0, \quad (x, t) \in Q_{T}, \\
			u_{\epsilon}(x, t)  = 0, \quad (x, t) \in \partial \Omega \times(0, T), \notag \\
			u_{\epsilon}(x, 0)  = u_{\epsilon, 0}(x), \quad x \in \Omega. \notag
		\end{cases}
	\end{equation}
	具有唯一弱解 $u_\epsilon$, 且 $u_\epsilon \in C([0, T], L^2(\Omega))$.
	由极值原理知 $\abs{u_{\epsilon}} \leq C\norm{u_0}_{\infty}$.

	将逼近方程与 $u_\epsilon$ 作乘积, 并在 $Q_T$ 上的积分得到
	\begin{equation}\label{eq:1}
		\begin{split}
			\frac{1}{2} \int_{\Omega} u_{\epsilon}^{2}
			&+\iint_{Q_{T}}(a(x)+\epsilon)\left(\left|\nabla u_{\epsilon}\right|^{2}+\epsilon\right)^{\frac{p-2}{2}}\left|\nabla u_{\epsilon}\right|^{2}\\
			&+\iint_{Q_{T}} b(x)\left|\nabla u_{\epsilon}\right|^{2} u_{\epsilon}  =\frac{1}{2} \int_{\Omega} u_{\epsilon, 0}^{2}.
		\end{split}
	\end{equation}

	其中
	\begin{equation}\label{eq:2}
		\begin{split}
			\abs{\int_0^T\int_{\Omega} b\abs{\nabla u_{\epsilon}}^2u_{\epsilon}}
			&\leq C\int_0^T\left( \int_{\Omega} b^{\frac{p}{p-2}}a^{-\frac{2}{p-2}}\abs{u_{\epsilon}}^{\frac{p}{p-2}} \right)^{\frac{p-2}{p}}
			\left(  \int_{\Omega} a\abs{\nabla u_{\epsilon}}^p \right)^{\frac{2}{p}}\\
			&\leq C\int_0^T\left(\int_{\Omega}\abs{u_{\epsilon}}^2\right)^{\frac{p}{2(p-2)}} + \frac{1}{2}\int_0^T\int_{\Omega} a\abs{\nabla u_{\epsilon}}^p\\
			&\leq \eta\int_0^T\int_{\Omega}\abs{u_{\epsilon}}^2 + \frac{1}{2}\int_0^T\int_{\Omega} a\abs{\nabla u_{\epsilon}}^p + C(\eta).
		\end{split}
	\end{equation}
	结合\cref{eq:1,eq:2}, 可得
	\begin{equation*}
		\begin{split}
			\frac{1}{2} \int_{\Omega} u_{\epsilon}^{2}
			&+\iint_{Q_{T}}(a(x)+\epsilon)\left(\left|\nabla u_{\epsilon}\right|^{2}+\epsilon\right)^{\frac{p-2}{2}}\left|\nabla u_{\epsilon}\right|^{2}\\
			&\leq \frac{1}{2}\iint_{Q_T}a\abs{\nabla u_\epsilon}^p + C.
		\end{split}
	\end{equation*}
	另外,
	\begin{equation*}
		\begin{split}
			\iint_{Q_T}a\abs{\nabla u_\epsilon}^p
			\leq \iint_{Q_{T}}(a(x)+\epsilon)\left(\left|\nabla u_{\epsilon}\right|^{2}+\epsilon\right)^{\frac{p-2}{2}}\left|\nabla u_{\epsilon}\right|^{2}
			\leq \frac{1}{2}\iint_{Q_T}a\abs{\nabla u_\epsilon}^p + C.
		\end{split}
	\end{equation*}
	即
	\begin{equation}\label{eq:iint_adu}
		\begin{split}
			\iint_{Q_T}a\abs{\nabla u_\epsilon}^p \leq C.
		\end{split}
	\end{equation}
	运用 \cite{Zhan2019Uniquenessa} 中方法容易得到 $u_\epsilon$ 收敛, 且收敛到方程\eqref{eq:main}的解.
	因此, 当初值满足~\eqref{initial_data_condition_tmp} 时方程\eqref{eq:main} 具有唯一弱解, 记为 $u$,
	且 $u \in C([0, T] \cap L^2(\Omega)) \cap L^{\infty}(Q_T) \cap  W_0^{1,p}(a, \Omega)$.

	下证当初值 $u_0 \in L^2(\Omega)$ 时弱解的存在唯一性.

	取一列 $\Brace{u_{n, 0}}_{n=1}^{\infty} \subset C_c^{\infty}(\Omega)$ 一致有界,
	当 $n \to \infty $ 时, $u_{n, 0}$ 在 $L^2(\Omega)$ 上收敛到 $u_0$.
	由上述的证明, 方程\eqref{eq:main}存在唯一的弱解 $u_n$.
	于是,
	\begin{equation}\label{eq:3}
		\begin{split}
			& \frac{1}{2}\int_{\Omega}\left(u_n-u_m\right)^2(t)\\
			+{} & \int_{0}^{t}\int_{\Omega}a(x)
			\left(\abs{\nabla u_n}^{p-2}\nabla u_n
			- \abs{\nabla u_m}^{p-2}\nabla u_m\right)
			\left(\nabla u_n - \nabla u_m\right)\\
			={} & \int_{0}^{t}\int_{\Omega}b(x)\left(\abs{\nabla u_n}^2
			- \abs{\nabla u_m}^2\right)\left(u_n - u_m\right)
			+ \frac{1}{2}\int_{\Omega}\left(u_n-u_m\right)^2(0).
		\end{split}
	\end{equation}
	由$a, b$的假设, Young不等式, Holder不等式, 结合引理\ref{lem:VecIneq}可得,
	\begin{equation}\label{eq:4}
		\begin{split}
			& \int_{0}^{t}\int_{\Omega}b\left(\abs{\nabla u_n}^2
			- \abs{\nabla u_m}^2\right)\left(u_n - u_m\right)\\
			\leq{} & \left(\int_0^t\int_{\Omega}a\left(\abs{\nabla u_n}^2
			- \abs{\nabla u_m}^2\right)^{\frac{p}{2}}\right)^{\frac{2}{p}}
			\left(\int_0^t\int_{\Omega}\left(ba^{-\frac{2}{p}}
			\left(u_n-u_m\right)\right)^{\frac{p}{p-2}}\right)^{\frac{p-2}{p}}\\
			\leq{} & C\left(\int_0^t\int_{\Omega}a
			\left(\abs{\nabla u_n}^{p-2}\nabla u_n
			- \abs{\nabla u_m}^{p-2}\nabla u_m\right)
			\left(\nabla u_n - \nabla u_m\right)\right)^{\frac{2}{p}}\\
			\times{} & \left(\int_0^t\int_{\Omega}b^{\frac{2p}{p-4}}a^{-\frac{4}{p-4}}\right)^{\frac{p-4}{2p}}
			\left(\int_0^t\int_{\Omega}\left(u_n-u_m\right)^2\right)^{\frac{1}{2}}\\
			\leq{} & \frac{1}{2}\int_0^t\int_{\Omega}a
			\left(\abs{\nabla u_n}^{p-2}\nabla u_n
			- \abs{\nabla u_m}^{p-2}\nabla u_m\right)
			\left(\nabla u_n - \nabla u_m\right)\\
			+{} & C\left(\int_0^t\int_{\Omega}\left(u_n-u_m\right)^2\right)^{\frac{p}{2(p-2)}}.
		\end{split}
	\end{equation}
	进而结合\cref{eq:3,eq:4}可得
	\begin{equation}\label{un-umL2_0TW1p_bd_0TL2_L20}
		\begin{split}
			& \int_{\Omega}\left(u_n-u_m\right)^2(t)\\
			+{} & \int_{0}^{t}\int_{\Omega}a
			\left(\abs{\nabla u_n}^{p-2}\nabla u_n
			- \abs{\nabla u_m}^{p-2}\nabla u_m\right)
			\left(\nabla u_n - \nabla u_m\right)\\
			\leq{} & C\left(\int_0^t\int_{\Omega}
			\left(u_n-u_m\right)^2\right)^{\frac{p}{2(p-2)}}
			+ \int_{\Omega}\left(u_n-u_m\right)^2(0).
		\end{split}
	\end{equation}

	接下来, 证明当 $p>4$, 对于几乎处处的 $0 \leq t \leq T$, $n, m \to \infty$ 时,
	有
	\begin{equation*}
		\left(\int_0^t\int_{\Omega}
			\left(u_n-u_m\right)^2\right)^{\frac{p}{2(p-2)}} \to 0.
	\end{equation*}
	只需证明
	\begin{equation*}
		\int_{\Omega}\left(u_n-u_m\right)^2(0) \to 0
	\end{equation*}
	时
	\begin{equation*}
		\int_0^t\int_{\Omega}\left(u_n-u_m\right)^2 \to 0.
	\end{equation*}
	假设,
	\begin{equation}\label{eq:assum}
		\int_0^t\int_{\Omega}\left(u_n-u_m\right)^2 \not\to 0,
	\end{equation}
	即, 存在 $\Brace{u_n}$ 的子列, 仍记为 $\Brace{u_n}$ , 以及 $c_0 > 0, N > 0$, 使得
	\begin{equation*}
		\int_0^t\int_{\Omega}\left(u_n-u_m\right)^2 > c_0, \quad n, m > N.
	\end{equation*}
	于是有
	\begin{equation*}
		\left(\int_0^t\int_{\Omega}\left(u_n-u_m\right)^2\right)^{\frac{p}{2(p-2)}}
		\leq c_0^{\frac{4-p}{2p-2}}\int_0^t\int_{\Omega}\left(u_n-u_m\right)^2.
	\end{equation*}
	将上式代入\cref{un-umL2_0TW1p_bd_0TL2_L20}可得
	\begin{equation*}
		\int_{\Omega}\left(u_n-u_m\right)^2(t)
		\leq C\int_0^t\int_{\Omega}\left(u_n-u_m\right)^2
		+ \int_{\Omega}\left(u_n-u_m\right)^2(0).
	\end{equation*}
	由 Gr\"onwall's 不等式可知, $n, m \to \infty$ 时,
	\begin{equation*}
		\int_{\Omega}\left( u_n-u_m \right)^2(t)
		\leq \int_{\Omega}\left(u_n-u_m\right)^2(0)e^{Ct} \to 0.
	\end{equation*}
	由 Lebesgue 控制收敛定理易得
	\begin{equation*}
		\int_0^t\int_{\Omega}\left(u_n-u_m\right)^2 \to 0, \quad n, m \to \infty,
	\end{equation*}
	与假设\eqref{eq:assum}矛盾. 对于几乎处处 $0 \leq t \leq T$, 让 $n,m \to \infty$, 有
	\begin{equation}\label{cauchy_in_L2}
		\norm{u_n(t,\cdot)-u_m(t,\cdot)}_{L^2(\Omega)} \to 0,
	\end{equation}
	和
	\begin{equation}\label{cauchy_in_W1pa}
		\int_{0}^{t}\int_{\Omega}a
		\left(\abs{\nabla u_n}^{p-2}\nabla u_n
		- \abs{\nabla u_m}^{p-2}\nabla u_m\right)
		\left(\nabla u_n - \nabla u_m\right)
		\to 0.
	\end{equation}
	注意到
	\begin{equation*}
		\begin{split}
			& \int_0^T\int_{\Omega}a\abs{\nabla u_n - \nabla u_m}^p\\
			\leq{} & 2^{p-2}\int_{0}^{T}\int_{\Omega}a
			\left(\abs{\nabla u_n}^{p-2}\nabla u_n
			- \abs{\nabla u_m}^{p-2}\nabla u_m\right)
			\left(\nabla u_n - \nabla u_m\right),
		\end{split}
	\end{equation*}
	因此, 存在 $u \in L^p(0, T; W_0^{1,p}(a,\Omega))
		\cap L^2(0, T; L^2(\Omega))$, 使得当 $n \to \infty$ 时,
	\begin{equation*}
		u_{n} \to u.
	\end{equation*}
	由 $u_n \in C([0, T], L^2(\Omega))$, 容易得到
	\begin{equation*}
		\lim_{t' \to t}\norm{u(t,\cdot)-u(t',\cdot)}_{L^2(\Omega)}=0.
	\end{equation*}
	即
	\begin{equation*}
		u \in C([0, T]; L^2(\Omega)).
	\end{equation*}

	最后,
	对任意的 $v \in C_c^\infty(\Omega)$, 由 H\"older 不等式可得,
	\begin{equation*}
		\begin{split}
			& \int_0^T\int_{\Omega}b\left(\abs{\nabla u_n}^2
			- \abs{\nabla u}^2\right)v\\
			\leq{} & C_p\left(\int_0^T\int_{\Omega}a
			\left(\abs{\nabla u_n}^{p-2}\nabla u_n
			- \abs{\nabla u}^{p-2}\nabla u\right)
			\left(\nabla u_n - \nabla u\right)\right)^{\frac{2}{p}}\\
			\times{} & \left(\int_0^T\int_{\Omega}b^{\frac{2p}{p-4}}a^{-\frac{4}{p-4}}\right)^{\frac{p-4}{2p}}
			\left(\int_0^T\int_{\Omega}v^2\right)^{\frac{1}{2}}.
		\end{split}
	\end{equation*}
	当 $n \to \infty$ 时, 由\eqref{cauchy_in_W1pa}可知上式右端趋于 $0$.
	
	由~\cite{Zhan2019Uniquenessa}的方法容易得到该方程的解的唯一性.
	这样我们完成了整个证明.
\end{proof}

现在我们可以定义方程\eqref{eq:main}解的连续半群 $\Brace{S(t)}_{t>0}$:
\begin{equation*}
	S(t)u_0 = u(t),
\end{equation*}
其中 $S(t)$ 在 $u_{0} \in L^2(\Omega)$ 上关于 $t$ 连续.
\chapter{全局吸引子的存在性}\label{ch:existence_of_the_global_attractors}
本章主要讨论方程\cref{eq:main}生成的解半群在$L^2(\Omega)$和$W_0^{1,2}(\Omega)$上全局吸引子的存在性.
\section{$L^2(\Omega)$上全局吸引子的存在性}
下面证明 $L^2(\Omega)$ 上全局吸引子的存在性.
首先, 需要得到半群在$L^2(\Omega)$ 和 $W_0^{1,p}(a,\Omega)$ 上分别存在有界吸收集

\begin{theorem}\label{thm:real_absorb}
	若 $a(x)$ 在\cref{thm:absorb}里的条件下同时满足
	\begin{equation}\label{eq:a_condition}
		\int_{\Omega}a^{-\frac{2}{p-2}} < \infty,
	\end{equation}
	则半群 $\Brace{S(t)}_{t \geq 0} $ 在
	$L^2$ 和 $W_0^{1,p}(a,\Omega)$ 上分别存在有界吸收集, 即对任意有界子集
	$B \subset L^2(\Omega)$, 存在常数 $T(\norm{u_0}_2)$ 及 $\rho > 0$, 使得
	对于所有 $t \geq T$ 和 $u_0 \in B$, 都有
	\begin{equation*}
		\norm{u(t)}_2^2 + \int_{\Omega}a\abs{\nabla u}^p \leq \rho,
	\end{equation*}
	其中 $u(t) = S(t)u_0$.
\end{theorem}
\begin{proof}
	将 方程\eqref{eq:main} 与 $u$ 相乘并在 $\Omega$ 上积分,
	得到
	\begin{equation}\label{eq:operate_u}
		\frac{1}{2}\frac{d}{dt}\norm{u}_2^2
		+ \int_\Omega a\abs{\nabla u}^p
		+ \int_\Omega b\abs{\nabla u}^2u = 0,
	\end{equation}
	因为 $p \geq 4$, 由 H\"older 不等式,
	\begin{equation}\label{eq:DL2_bd_W1pa}
		\int_{\Omega}\abs{\nabla u}^2
		= \int_{\Omega}a^{-\frac{2}{p}}a^{\frac{2}{p}}\abs{\nabla u}^2
		\leq \left(\int_{\Omega}a^{-\frac{2}{p-2}}\right)^{\frac{p-2}{p}}
		\left(\int_{\Omega}a\abs{\nabla u}^p\right)^{\frac{2}{p}}
	\end{equation}
	再由 Poincar\`e 不等式知,
	\begin{equation*}
		\left(\int_{\Omega}\abs{u}^2\right)^{\frac{p}{2}}
		\leq C\int_{\Omega}a\abs{\nabla u}^p,
	\end{equation*}
	其中 $C$ 只与 $\Omega$ 和 $p$ 有关. 对于\cref{eq:operate_u}第三项, 运用
	H\"older 不等式和 Young 不等式得到,
	\begin{equation}\label{eq:absorb_damping_u}
		\begin{split}
			\abs{\int_{\Omega}b\abs{\nabla u}^2 u}
			&\leq \left(\int_{\Omega}a\abs{\nabla u}^p\right)^{\frac{2}{p}}
			\left(\int_{\Omega}\abs{b a^{-\frac{2}{p}} u}^{\frac{p}{p-2}}\right)^{\frac{p-2}{p}}\\
			&\leq \frac{1}{2}\int_{\Omega}a\abs{\nabla u}^p
			+ 2^{\frac{p-2}{2}}\int_{\Omega}\abs{b a^{-\frac{2}{p}} u}^{\frac{p}{p-2}}\\
			&\leq \frac{1}{2}\int_{\Omega}a\abs{\nabla u}^p
			+ 2^{\frac{p-2}{2}}\left(\int_{\Omega}b^{\frac{2p}{p-4}}a^{-\frac{4}{p-4}}\right)^{\frac{p-4}{2(p-2)}}
			\left(\int_{\Omega}u^2\right)^{\frac{p}{2(p-2)}}\\
			&\leq \frac{1}{2}\int_{\Omega}a\abs{\nabla u}^p
			+ \epsilon \left(\int_{\Omega}u^2\right)^{\frac{p}{2}}
			+ C_{\epsilon}\left(\int_{\Omega}b^{\frac{2p}{p-4}}a^{-\frac{4}{p-4}}\right)^{\frac{p-4}{2(p-3)}}.
		\end{split}
	\end{equation}
	令 $\epsilon$ 足够小, 由\cref{eq:operate_u}-\cref{eq:absorb_damping_u} 得到
	\begin{equation}\label{eq:DL2_L2_bd}
		\frac{d}{dt}\norm{u}_2^2 + \norm{u}_2^p \leq C
	\end{equation}
	其中 $C$ 只与 $u$ 有关. 由 Gr\"onwall 不等式, 可以得到 $L^2(\Omega)$ 上的有界吸收集,
	即 $\exists \rho_0$ 和
	$T_0 = T_0(\norm{u_0}_2)$ 使得
	\begin{equation}\label{eq:L2_bd}
		\norm{u(t)}_2^2 \leq \rho_0 , \quad t \geq T_0.
	\end{equation}
	应用\cref{eq:operate_u,eq:absorb_damping_u,eq:L2_bd}, 有
	\begin{equation}\label{eq:DL2_W1pa_bd}
		\frac{d}{dt}\norm{u}_2^2 + \int_{\Omega}a \abs{\nabla u}^p \leq C.
	\end{equation}
	将方程\eqref{eq:main} 与 $u_{t}$ 相乘并在 $\Omega$ 上积分得,
	\begin{equation}\label{eq:operate_ut}
		\norm{u_t}_2^2
		+ \frac{1}{p}\frac{d}{dt}\int_{\Omega}a \abs{\nabla u}^p
		+ \int_{\Omega}b \abs{\nabla u}^2 u_t
		= 0,
	\end{equation}
	类似于\cref{eq:absorb_damping_u}估计\cref{eq:operate_ut}第三项, 可以得到
	\begin{equation*}
			\abs{\int_{\Omega}b\abs{\nabla u}^2 u_t}
			\leq \frac{1}{2}\int_{\Omega}a\abs{\nabla u}^p
			+ \epsilon \left(\int_{\Omega}u_t^2\right)^{\frac{p}{2}}
			+ C_{\epsilon}\left(\int_{\Omega}b^{\frac{2p}{p-4}}a^{-\frac{4}{p-4}}\right)^{\frac{p-4}{2(p-3)}}.
	\end{equation*}
	将其带入\cref{eq:operate_ut}有,
	\begin{equation}\label{eq:utL2_DW1pa_bd_W1pa}
		C\norm{u_t}_2^2
		+ \frac{d}{dt}\int_{\Omega}a \abs{\nabla u}^p
		\leq \int_{\Omega}a \abs{\nabla u}^p
		+ C,
	\end{equation}
	结合\cref{eq:DL2_W1pa_bd}, \cref{eq:utL2_DW1pa_bd_W1pa} 得到
	\begin{equation}\label{eq:utL2_DW1pa_DL2}
		C\norm{u_t}_2^2
		+ \frac{d}{dt}\left(\int_{\Omega}a \abs{\nabla u}^p
		+ \norm{u}_2^2\right)
		\leq C.
	\end{equation}
	应用\cref{eq:DL2_L2_bd}, 有
	\begin{equation*}
		\frac{d}{dt}\norm{u}_2^2 + \norm{u}_2^2 \leq C.
	\end{equation*}
	因此
	\begin{equation}\label{eq:int_DL2_L2_bd}
		\norm{u(t+1)}_2^2
		+ \int_t^{t+1}\norm{u}_2^2
		\leq C + \norm{u(t)}_2^2.
	\end{equation}
	类似的, 应用\cref{eq:DL2_W1pa_bd}, 有
	\begin{equation}\label{eq:int_DL2_W1pa_bd}
		\norm{u(t+1)}_2^2
		+ \int_t^{t+1}\int_{\Omega}a \abs{\nabla u}^p
		\leq C + \norm{u(t)}_2^2,
	\end{equation}
	联系\cref{eq:int_DL2_L2_bd,eq:int_DL2_W1pa_bd}, 有
	\begin{equation}\label{eq:int_W1pa_L2}
		\int_t^{t+1}\int_{\Omega}a \abs{\nabla u}^p
		+ \norm{u}_2^2 \leq C , \quad t \geq T_0.
	\end{equation}
	应用\cref{eq:utL2_DW1pa_DL2,eq:int_W1pa_L2}, 由一致 Gr\"onwall 引理, 我们得到 $W_0^{1,p}(a,\Omega)$ 上的有界吸收集,
	即 $\exists \rho$ 和 $T \geq T_{0}$ 使得
	\begin{equation}\label{eq:uL2_W1pa_bd}
		\norm{u(t)}_2^2 + \int_{\Omega}a\abs{\nabla u}^p \leq \rho , \quad t \geq T.
	\end{equation}
	以上完成了整个证明.
\end{proof}

由\cref{thm:real_absorb}可知, 在$W_0^{1, 2}(\Omega)$中的有界吸收集存在,
再由紧嵌入定理
$W_0^{1,2}(\Omega) \subset\subset L^2(\Omega)$ 以及 \cref{eq:DL2_bd_W1pa}, 
立即得到 $L^2(\Omega)$ 中全局吸引子的存在性.
\begin{theorem}\label{thm:attractor_L2}
	若 $a(x)$ 同时满足条件\eqref{eq:a_condition},
	则由方程\eqref{eq:main}的弱解生成的半群 $\Brace{S(t)}_{t \geq 0}$ 在
	$L^2(\Omega)$ 存在全局吸引子 $\mathcal{A}_2$.
\end{theorem}
\section{$W_0^{1,p}(\Omega)$上全局吸引子的存在性}
接下来证明 $W_0^{1,p}(\Omega)$ 中全局吸引子的存在性. 首先, 给出 $u_t$ 在 $L^2(\Omega)$ 中的有界性.
\begin{theorem}\label{thm:ut_L2_bd}
	对任意有界子集 $B \subset L^2(\Omega)$,
	存在常数 $T' = T'(B) > 0$, 使得对 $\forall u_0 \in B, s \geq T'$, 有
	\begin{equation*}
		\norm{u_t(s)}_2^2 \leq M.
	\end{equation*}
\end{theorem}
\begin{proof}
	% \begin{proofpart}
	% 	First we consider the situation of $\abs{\nabla u} \to 0$ as $t \to t_0$,
	% 	where $t_0 \in \R^{+}\cup\Brace{+\infty}$, without loss of generality,
	% 	we assume $\abs{\nabla u} \leq 1$.
	% 	If we multiply方程\eqref{eq:main} by any fixed $\phi \in C_c^{\infty}(\Omega)$
	% 	and integrate over $\Omega$, then take limits, we have,
	% 	\begin{equation}
	% 		\begin{split}
	% 			\lim_{t \to t_0}\int_{\Omega}u_t \phi
	% 			= - \lim_{t \to t_0}\int_{\Omega}a\abs{\nabla u}^{p-2} \nabla u
	% 			\cdot \nabla \phi
	% 			- \lim_{t \to t_0}\int_{\Omega}b\abs{\nabla u}^2 \phi
	% 			= 0
	% 		\end{split}
	% 	\end{equation}
	% \end{proofpart}
	对方程\eqref{eq:main} 求关于 $t$ 的导数, 记 $v = u_t$, 有
	\begin{equation*}
		\begin{split}
			v_t
			= \Div\left(a\abs{\nabla u}^{p-2}\nabla v\right)
			+ \Div\left(\left(p-2\right)a \abs{\nabla u}^{p-4}\left(\nabla u \cdot \nabla v\right)\nabla u\right)
			- 2b\nabla u \cdot \nabla v.
		\end{split}
	\end{equation*}
	将上述方程与 $v$ 相乘并在 $\Omega$ 上积分, 得到
	\begin{equation*}
		\begin{split}
			\frac{1}{2}\frac{d}{dt}\norm{v}_2^2
			+ \int_{\Omega}a\abs{\nabla u}^{p-2}\abs{\nabla v}^2
			&+ \int_{\Omega}\left(p-2\right)a\abs{\nabla u}^{p-4}\left(\nabla u
			\cdot \nabla v\right)^2\\
			&+ 2\int_{\Omega}b\nabla u \cdot \nabla v v
			= 0.
		\end{split}
	\end{equation*}
	其中,
	\begin{equation*}
		\begin{split}
			\abs{\int_{\Omega}b\nabla u \cdot \nabla v v}
			&\leq \left(\int_{\Omega}b^2\left(\nabla u
			\cdot \nabla v\right)^2\right)^{\frac{1}{2}}
			\*\norm{v}_2\\
			&\leq \norm{b^2 a^{-1}}_{\infty}^{\frac{1}{2}}
			\left(\int_{\Omega}a\abs{\nabla u}^2\abs{\nabla v}^2\right)^{\frac{1}{2}}\norm{v}_2\\
			&\leq \norm{b^2 a^{-1}}_{\infty}^{\frac{1}{2}}
			\left(\int_{\Omega}a\left(\abs{\nabla u}+1\right)^2\abs{\nabla v}^2\right)^{\frac{1}{2}}\norm{v}_2\\
			% &= \norm{b^2 a^{-1}}_{\infty}^{\frac{1}{2}}
			% \left(
			% \int_{\Omega}a\left(
			% \left(\abs{\nabla u}+1\right)^2 + 1 - 2\left(\abs{\nabla u}+1\right)
			% \right)\abs{\nabla v}^2
			% \right)^{\frac{1}{2}}\norm{v}_2\\
			&\leq \norm{b^2 a^{-1}}_{\infty}^{\frac{1}{2}}
			\left(\int_{\Omega}a\left(\abs{\nabla u}+1\right)^{p-2}\abs{\nabla v}^2\right)^{\frac{1}{2}}\norm{v}_2\\
			&\leq \epsilon \int_{\Omega}a\left(\abs{\nabla u}+1\right)^{p-2}\abs{\nabla v}^2
			+ C(\epsilon) \norm{v}_2^2,
		\end{split}
	\end{equation*}
	和
	\begin{equation*}
		\begin{split}
			\int_{\Omega}a\abs{\nabla u}^{p-2}\abs{\nabla v}^2
			\geq 2^{3-p}\int_{\Omega}a\left(\abs{\nabla u}+1\right)^{p-2}\abs{\nabla v}^2
			- \int_{\Omega}a\abs{\nabla v}^2.
		\end{split}
	\end{equation*}
	于是可得,
	\begin{equation}\label{eq:DvL2_bd_vL2}
		\begin{split}
			& \frac{1}{2}\frac{d}{dt}\norm{v}_2^2
			+ C\int_{\Omega}a\left(\abs{\nabla u}+1\right)^{p-2}\abs{\nabla v}^2\\
			+{} & \int_{\Omega}\left(p-2\right)a\abs{\nabla u}^{p-4}\left(\nabla u
			\cdot \nabla v\right)^2
			\leq C\norm{v}_2^2 + C.
		\end{split}
	\end{equation}
	现在应用\cref{eq:utL2_DW1pa_bd_W1pa,eq:uL2_W1pa_bd,eq:DvL2_bd_vL2},
	得到
	\begin{equation*}
		\begin{split}
			\frac{d}{dt}\left(
			\norm{v}_2^2 + C\int_{\Omega}a\abs{\nabla u}^p
			\right)
			\leq C.
		\end{split}
	\end{equation*}
	接下来, 对\cref{eq:utL2_DW1pa_DL2}做从 $t$ 到 $t+1$的积分, 有
	\begin{equation*}
		\begin{split}
			\int_{\Omega}a\abs{\nabla u(t+1)}^p
			+ \norm{u(t+1)}_2^2
			+ C\int_t^{t+1}\norm{v}_2^2
			\leq C + \int_{\Omega}a\abs{\nabla u(t)}^p
			+ \norm{u(t)}_2^2,
		\end{split}
	\end{equation*}
	因此, 当 $t$ 足够大时, 有
	\begin{equation*}
		\begin{split}
			\int_t^{t+1}\left(
			\norm{v}_2^2 + C\int_{\Omega}a\abs{\nabla u}^p
			\right) \leq C,
		\end{split}
	\end{equation*}
	由一致 Gr\"onwall 引理得, 存在 $M > 0, T' \geq T$, 使得
	\begin{equation*}
		\begin{split}
			\norm{v(t)}_2^2
			+ \int_{\Omega}a\abs{\nabla u}^p \leq M , \quad t \geq T',
		\end{split}
	\end{equation*}
	以上完成了证明.
\end{proof}
现在证明
$W_0^{1,p}(\Omega)$ 中全局吸引子的存在性.
\begin{theorem}
	若 $a(x)$ 满足\cref{thm:absorb}的假设以及条件\eqref{eq:a_condition},
	则由方程\eqref{eq:main}得到的半群 $\Brace{S(t)}_{t \geq 0}$ 在 $W_0^{1,p}(\Omega)$
	中存在全局吸引子 $\mathcal{A}_V$, 即
	$\mathcal{A}_V$ 在 $W_0^{1,p}(\Omega)$ 中紧, 不变并且在 $W_0^{1,p}(\Omega)$
	拓扑下吸引 $L^2(\Omega)$ 中的任意有界子集.
\end{theorem}
\begin{proof}
	为了得到吸引子的存在性,
	只需证明 $\Brace{S(t)}_{t \geq 0}$ 在 $W_0^{1,p}(\Omega)$ 里是渐近紧的.
	由\cref{thm:real_absorb}, 令 $B_0$ 是在 $W_0^{1,p}(\Omega)$ 的有界吸收集,
	接下来验证对于任意序列
	$\Brace{u_{0n}}_{n=1}^{\infty} \subset B_0$, $\Brace{u_n(t_n)}_{n=1}^{\infty}$
	在 $W_0^{1,p}(\Omega)$ 存在收敛子列.

	事实上, 由\cref{thm:attractor_L2}, 以及$W_0^{1,2}(\Omega) \subset\subset L^2(\Omega)$,
	容易知道
	$\Brace{u_n(t_n)}_{n=1}^{\infty}$ 在 $L^2(\Omega)$ 里是预紧的.
	从 $\Brace{u_n(t_n)}_{n=1}^{\infty}$ 取子列在 $L^2(\Omega)$ 上是 Cauchy 列,
	仍记为 $\Brace{u_n(t_n)}_{n=1}^{\infty}$. 由 $p \geq 4$ 以及
	H\"older 不等式, 有
	\begin{equation*}
		\begin{split}
			& \int_{\Omega}a\abs{\nabla u_m(t_m) - \nabla u_n(t_n)}^p\\
			\leq{} & C\int_{\Omega}a
			\left(\abs{\nabla u_m(t_m)}^{p-2}\nabla u_m(t_m)
			- \abs{\nabla u_n(t_n)}^{p-2}\nabla u_n(t_n)\right)
			\*\nabla\left(u_m(t_m) - u_n(t_n)\right)\\
			={} & C\left(\int_{\Omega}\left(u_n-u_m\right)_t\left(u_m-u_n\right)
			- \int_{\Omega}b\left(\abs{\nabla u_m}^2 - \abs{\nabla u_n}^2\right)
			\left(u_m - u_n\right)\right)\\
			\leq{} & C\norm{\left(u_n-u_m\right)_t}_2\norm{u_m-u_n}_2\\
			+{} & C\left(\int_{\Omega}a\left(\abs{\nabla u_m}^p
			+ \abs{\nabla u_n}^p\right)\right)^{\frac{2}{p}}
			\left(b^{\frac{2p}{p-4}}a^{-\frac{4}{p-4}}\right)^{\frac{p-4}{2p}}
			\norm{u_m-u_n}_2.
		\end{split}
	\end{equation*}
	当 $m$, $n \to \infty$ 时, 可知
	\begin{equation*}
		\int_{\Omega}a\abs{\nabla u_m(t_m) - \nabla u_n(t_n)}^p \to 0.
	\end{equation*}
	即得半群是渐近紧的.

	结合\cref{thm:real_absorb}, 方程\cref{eq:main}生成的解半群 $\Brace{S(t)}_{t \geq 0}$ 在 $W_0^{1, p}(\Omega)$ 中
	存在一个全局吸引子 $\mathcal{A}_V$.
\end{proof}

\chapter{全局吸引子的上半连续性}
本章, 我们考虑全局吸引子 $\mathcal{A}_\lambda$, 当 $\lambda \to \lambda_0 \in [0, \infty)$ 时的上半连续性, 即对
$\forall \epsilon > 0$, 存在 $\delta > 0$, 当 $\abs{\lambda - \lambda_0} < \delta$ 时, 有
\begin{equation*}
	\dist(\mathcal{A}_\lambda, \mathcal{A}_{\lambda_0}) < \epsilon,
\end{equation*}
其中 $\dist(A, B) = \sup_{a \in A}\inf_{b \in B}d(a, b)$ 表示集合 $A$, $B$ 间的 Hausdorff 半距离.
即 $\mathcal{A}_\lambda \subset \mathcal{N}(\mathcal{A}_{\lambda_0}, \epsilon)$, 因此上半连续性可以保证吸引子在参数扰动下不爆破.

考虑方程\cref{eq:main}的扰动方程如下
\begin{equation}\label{eq:turb_main}
	\begin{cases}
		u_{\lambda t} = \Div(a_{\lambda}(x)\abs{\nabla u_{\lambda}}^{p-2}\nabla u_{\lambda}) - b(x)\abs{\nabla u_{\lambda}}^2
		\quad &(x, t) \in \Omega \times \R^+,\\
		u_{\lambda}(x,0) = u_{0\lambda} \quad &x \in \Omega,\\
		u_{\lambda}(x, t) = 0 \quad &x \in \partial\Omega,
	\end{cases}
\end{equation}
其中, $\lambda \in [0, \infty)$. 当 $\lambda \to \lambda_0$ 时, $\norm{a_\lambda - a_{\lambda_0}}_{L^\infty(\Omega)} \to 0$.
方程\eqref{eq:turb_main}满足方程\eqref{eq:main}的条件与假设, 并有如下假设.

对 $\forall \lambda$, 存在 $M_1, M_2 > 0$, 使得
\begin{gather}
	\int_{\Omega} b^{\frac{2p}{p-4}}a_\lambda^{-\frac{4}{p-4}} \leq M_1, \label{h1} \\
	\int_{\Omega}a_\lambda^{-\frac{2}{p-2}} \leq M_2. \label{h2}
\end{gather}

由定理\ref{thm:attractor_L2}易知, 对 $\forall \lambda$,
方程\eqref{eq:turb_main}生成的解半群 $S_{\lambda}(t)$ 都存在 $L^2(\Omega)$ 上的全局吸引子 $\mathcal{A}_\lambda$.

为了讨论全局吸引子的上半连续性, 我们需要如下引理,
\begin{lemma}\label{lem:slambdacontinuous}
	令 $\Brace{S_\lambda(t) \colon \lambda \in [0, \infty)}$ 是由方程 \eqref{eq:turb_main} 生成的解半群族.
	当 $\lambda \to \lambda_0 \in [0, \infty)$ 时, 在 $L^2(\Omega)$ 中, $u_{0\lambda} \to u_{0\lambda_0}$.
	则对任意固定的 $T > 0, \forall t \in [0, T]$ 有
	\begin{equation*}
		\norm{S_\lambda(t)u_{0\lambda} - S_{\lambda_0}(t)u_{0\lambda_0}}_2 \to 0
	\end{equation*}
\end{lemma}
\begin{proof}
	由方程\cref{eq:turb_main}知,
	\begin{equation*}
		(u_{\lambda} - u_{\lambda_0})_t
		= \Div(a_{\lambda}(x)\abs{\nabla u_{\lambda}}^{p-2}\nabla u_{\lambda} - a_{\lambda_0}(x)\abs{\nabla u_{\lambda_0}}^{p-2}\nabla u_{\lambda_0})
		- b(x)(\abs{\nabla u_{\lambda}}^2 - \abs{\nabla u_{\lambda_0}}^2).
	\end{equation*}
	用 $u_{\lambda} - u_{\lambda_0}$ 与上式作用得,
	\begin{equation}
		\begin{split}
			& \frac{1}{2}\int_{\Omega}\left(u_{\lambda}-u_{\lambda_0}\right)^2(t)\\
			+{} & \int_{0}^{t}\int_{\Omega}a_\lambda(x)
			\left(\abs{\nabla u_{\lambda}}^{p-2}\nabla u_{\lambda}
			- \abs{\nabla u_{\lambda_0}}^{p-2}\nabla u_{\lambda_0}\right)
			\left(\nabla u_{\lambda} - \nabla u_{\lambda_0}\right)\\
			+{} & \int_{0}^{t}\int_{\Omega}(a_\lambda - a_{\lambda_0})(\abs{\nabla u_{\lambda_0}}^{p - 2}\nabla u_{\lambda_0}
			(\nabla u_{\lambda} - \nabla u_{\lambda_0})) \\
			={} & \int_{0}^{t}\int_{\Omega}b(x)\left(\abs{\nabla u_{\lambda}}^2
			- \abs{\nabla u_{\lambda_0}}^2\right)\left(u_{\lambda} - u_{\lambda_0}\right)
			+ \frac{1}{2}\int_{\Omega}\left(u_{\lambda}-u_{\lambda_0}\right)^2(0).
		\end{split}
	\end{equation}
	其中
	\begin{equation*}
		\begin{split}
			& \int_{0}^{t}\int_{\Omega}(a_\lambda - a_{\lambda_0})(\abs{\nabla u_{\lambda_0}}^{p - 2}\nabla u_{\lambda_0}
			(\nabla u_{\lambda} - \nabla u_{\lambda_0})) \\
			\leq{} & \norm{a_\lambda - a_{\lambda_0}}_{L^\infty(\Omega)}
			\left(
				\int_0^t\int_{\Omega}\abs{\nabla u_{\lambda_0}}^{p-1}\abs{\nabla u_\lambda}
				+ \int_0^t\int_{\Omega}\abs{\nabla u_{\lambda_0}}^p
			\right).
		\end{split}
	\end{equation*}
	由\cref{eq:iint_adu,eq:DL2_bd_W1pa,h2}知, 当 $\lambda \to \lambda_0$ 时, 上式右端趋于$0$.
	根据假设\cref{h1}以及\cref{cauchy_in_L2}的证明方法类似可证, 对于几乎处处的 $t \in [0, T]$, 当 $\lambda \to \lambda_0$ 时, 有
	\begin{equation}
		\norm{u_\lambda(t,\cdot)-u_{\lambda_0}(t,\cdot)}_{L^2(\Omega)} \to 0.
	\end{equation}
	以上完成了整个定理的证明.
\end{proof}

最后我们给出本章的主要定理.
\begin{theorem}
	对 $\forall \lambda_0 \in [0, \infty)$, 
	方程\cref{eq:turb_main}的全局吸引子 $\mathcal{A}_\lambda$ 在 $\lambda = \lambda_0$ 处是上半连续的, 即
	\begin{equation*}
		\lim_{\lambda \to \lambda_0} \dist(\mathcal{A}_\lambda, \mathcal{A}_{\lambda_0}) = 0.
	\end{equation*}
\end{theorem}
\begin{proof}
	由假设\cref{h1,h2}以及\cref{thm:real_absorb}知, 存在 $L^2(\Omega)$ 里的有界集 $B$, 使得
	\begin{equation*}
		\bigcup_{0 \leq \lambda < \infty} \mathcal{A}_\lambda \subset B.
	\end{equation*}
	于是, 对 $\forall \epsilon > 0$, 存在 $t_0 > 0$, 当 $t > t_0$ 时, 有
	\begin{equation*}
		S_{\lambda_0}(t)B \subset \mathcal{N}(\mathcal{A}_{\lambda_0}, \epsilon / 2).
	\end{equation*}
	另外, 由引理\ref{lem:slambdacontinuous}以及定理\ref{thm:real_absorb}可知,
	存在足够小的 $\eta(\epsilon) > 0$, 以及 $t_1 > t_0$, 使得 $\abs{\lambda - \lambda_0} < \eta(\epsilon)$, 以及 $t > t_1$ 时, 有
	\begin{equation*}
		\sup_{x \in B}\norm{S_\lambda(t)x - S_{\lambda_0}(t)x}_{L^2(\Omega)} < \epsilon / 2.
	\end{equation*}
	由于$\mathcal{A}_\lambda \subset B$, 于是有
	\begin{equation*}
		\mathcal{A}_\lambda = S_\lambda(t)\mathcal{A}_\lambda \subset S_\lambda(t)B \subset \mathcal{N}(\mathcal{A}_{\lambda_0}, \epsilon),
	\end{equation*}
	即 $\dist(\mathcal{A}_\lambda, \mathcal{A}_{\lambda_0}) \leq \epsilon$.
	% \begin{equation*}
	% 	\begin{split}
	% 		\dist(\mathcal{A}_\lambda, \mathcal{A}_{\lambda_0})
			
	% 	\end{split}
	% \end{equation*}
\end{proof}

% \begin{theorem}
% 	假设对于每一个 $\eta \in [0, \eta_0)$, 半群 $S_\eta$ 都存在全局吸引子 $\mathcal{A}_\eta$, 且满足以下条件,
% 	\begin{enumerate}
% 		\item 存在一个有界集 $B$, 使得
% 		\begin{equation*}
% 			\cup_{0 \leq \eta < \eta_0} \mathcal{A}_\eta \subset B,
% 		\end{equation*}
% 		\item  
% 	\end{enumerate}
% \end{theorem}

%附录部分
\appendix

%自动生成参考文献列表, 需要模板根目录下有:bib/thesis.bib
\bibliographystyle{amsplain}
\bibliography{bib/thesis}
% \printbibliography

%致谢
\begin{thanks}
	首先, 衷心的感谢我的导师马闪副教授! 感谢马老师一直以来对我学习的鼓励和支持!
	感谢她在我读硕士这三年对我孜孜不倦的教诲和指导.
	在毕业论文撰写期间, 马老师总是极具耐心地指导我, 并一丝不苟地帮我修改,
	从论文课题的选择到论文写作定稿的过程, 老师都倾注了大量的心血和时间.

	其次, 感谢我的同学、同门、以及师兄弟姐妹们三年里对我的宽容以及对我的帮助和建议.
	感谢 LZUthesis 模版的作者帮我节省了很多排版的精力.

	最后感谢答辩委员会和参加论文评阅的专家和老师们为本论文给予的宝贵建议与
	悉心指导.
	\begin{flushright}
		李蕴方\\
		2020年5月于兰州大学
	\end{flushright}

\end{thanks}

\end{document}
